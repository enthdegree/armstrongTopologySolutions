\documentclass{book}
\usepackage[utf8]{inputenc}
\usepackage{amsfonts, amsthm, amsmath, amssymb, mathrsfs,mathabx}
\usepackage{enumerate}
\usepackage{hyperref}
\usepackage{units}
\usepackage{mathtools}
\usepackage{xcolor}
\usepackage{graphicx}
\graphicspath{ {images/} }
\usepackage{caption}
\usepackage{subcaption}
\usepackage{tikz}
\usepackage{tikz-3dplot}
\usepackage{standalone}
\usepackage[numbers]{natbib}

%defined functions and such
\DeclareMathOperator*{\Rcal}{\mathcal{R}}
\DeclareMathOperator*{\Q}{\mathbb{Q}}
\DeclareMathOperator*{\R}{\mathbb{R}}
\DeclareMathOperator*{\C}{\mathbb{C}}
\DeclareMathOperator*{\N}{\mathbb{N}}
\DeclareMathOperator*{\Aut}{\text{Aut}}
\DeclareMathOperator*{\Int}{\text{Int}}
\DeclareMathOperator*{\Out}{\text{Out}}
\DeclareMathOperator*{\E}{\mathbb{E}}
\DeclareMathOperator*{\Ocal}{\mathcal{O}}
\DeclareMathOperator*{\Ft}{\text{Fr}}
\DeclareMathOperator*{\Image}{\text{Im}}
\DeclareMathOperator*{\F}{\mathcal{F}}
\DeclareMathOperator*{\Diam}{\text{Diam}}
\DeclareMathOperator*{\RP}{\R\text{P}}
\DeclareMathOperator*{\Z}{\mathbb{Z}}
\DeclareMathOperator*{\Orb}{\text{Orb}}
\DeclareMathOperator*{\Stab}{\text{stab}}
\DeclareMathOperator*{\card}{\text{card}}
\DeclareMathOperator*{\Star}{\text{star}}
\DeclareMathOperator*{\car}{\text{carrier}}
\DeclareMathOperator*{\KB}{\text{KB}}
\DeclareMathOperator*{\MB}{\text{MB}}

%new commands
\newenvironment{Lemma}[1][Lemma]
{\proof[#1]\leftskip=1cm\rightskip=1cm}
{\endproof} %this make a lemma indented in proof
\newcommand{\bij}{\mathrel{\hookrightarrow\hspace{-1.8ex}\to}} %creates a bijection arrow
\newcommand*\productop{\mathbin{\Pi}} %disjoint union 
\newcommand\inner[2]{\langle #1, #2 \rangle}
\newcommand\innerone[1]{\langle #1 \rangle}
\newcommand\red[1]{{\color{red} #1}}

%NEW ENVIRONMENTS
\newenvironment{acknowledgements}%
{\cleardoublepage\thispagestyle{empty}\null\vfill\begin{center}%
\bfseries Acknowledgements\end{center}}%
{\vfill\null}

%counters and reformating: 
\setcounter{section}{-1}
\setcounter{chapter}{-1}

\title{Armstrong Topology Solutions}
\author{George Blikas \\ Gregory Grant \\ David Hong \\ Grant Baker \\ Kristy Ong  \\ \red{Christian Chapman}}
\date{April 2016 \red{(September 2023)}}

\begin{document}

%creates title and the empty page style allows 
%for the format of the blank page after
%maketitle to actually be blank (w/o numbering).
\maketitle
\thispagestyle{empty} 
\tableofcontents

\begin{acknowledgements}
    George B: Much thanks to all the contributors of this book. Special thanks goes to Gregory Grant, and particularly in helping establish chapter \hyperref[chapter:intro]{\ref{chapter:intro}}, section \hyperref[sec:productspaces]{\ref{sec:productspaces}}, and problems $10$, $11$, and $12$ of section $4.2$, all of which greatly sped-up the processes of this project. 
\end{acknowledgements}

\chapter{Preliminaries}
\section{Definitions}

\subsection{Functions}
\begin{enumerate}[(1)]
    \item Let $f:A \rightarrow B$ be a function. Then, 
        \begin{itemize}
            \item $f$ is onto if, and only if, there exists a function $g: B \rightarrow A$ such that $fg = 1_B$.
            \item $f$ is one-to-one if, and only if, there exists a function $g: B \rightarrow A$ such that $gf = 1_A$ (provided $A$ is non-empty). 
            \item $f$ Is in 1-1 correspondence if there exists a function $g: B \rightarrow A$ such that $fg = 1_B$ and $gf = 1_A$. In this case, $g$ is unique and is called the inverse function of $f$, typically denoted $f^{-1}$. 
        \end{itemize}
\end{enumerate}


\section{Extra Proofs and Lemmas}

\subsection{Group Theory \& Abstract Algebra}
\begin{enumerate}[(1)]
    \item Split Exact Sequence: \label{sec:splitexact}
        Given an exact sequence of abeilean groups and homomorphisms, 
        \[0 \xrightarrow[]{} G \xrightarrow{\theta} H \xrightarrow{\phi} K \xrightarrow[]{} 0, \]
        and a homomorphism $\psi: K \rightarrow H$, such that $\phi\psi \equiv 1_K$, we have $H \cong G \oplus K$. 
        \begin{proof}
            Define $\alpha: G \oplus K \rightarrow H$ by $\alpha(g \oplus k) = \theta(g) + \psi(k)$: it is easy to see that $\alpha$ is a homomorphism. Also, $\alpha$ is (1-1), for if $\alpha(g \oplus k) = 0$, we have 
            \[0 = \phi(\theta(g) + \psi(k)) = \phi\psi(k) = k;\]
            but, then $\theta(g) = 0$, so that $g = 0$ since $\theta$ is one-to-one. 
            \par Moreover, $\alpha$ is onto, since given $h \in H$, we have 
            \[\phi(h - \psi\phi(h)) = \phi(h) - \phi\psi\phi(h) = 0\]
            Thus, there exists $g \in G$ such that $h - \psi\phi(k) = \theta(g)$, that is, 
            \[h = \theta(g) + \psi\phi(h) = \alpha(g \oplus \psi(h))\]
        \end{proof}
\end{enumerate}


\subsection{Limit Points, Closure \& Density}
\begin{enumerate}[(1)]

    \item Let $(X, \tau)$ be a topological space, and $A \subset X$. We show that $\overline{A} = A \cup L_A$, where $L_A$ is the set of all accumulation points of $A$:
        \begin{proof} If $\overline{A} = A$, then by Theorem $2.2$, pg. $29$, $L_A \subset A$ and we are done. So, suppose that $A$ is not closed. Then, $A$ does not contain all of its limit points; consequently, $L_A \neq \emptyset$. Further, the set $B = A \cup L_A$ is closed by Theorem $2.2$, pg. $29$. 
            \par To conclude the proof, we show that any closed set $C$ containing $A$, contains $B$. Indeed; Let $C \subset X$ be closed, such that $A \subset C$. Now, for $a \in L_A$ and open set $O_a$ containing $a$, we have that $A \cap (O_a - \{a\}) \neq \emptyset$. But, as $A \subset C$, we have $C \cap (O_a - \{a\}) \neq \emptyset$. Therefore, $a \subset C$. So, $L_A \subset C$, implying $A \cup L_A \subset C$. By Theorem $2.3$, pg. $30$, $\overline{A} = B = A \cup L_A$.
        \end{proof}

    \item Let $(X, \tau)$ be a topological space, and $A \subset X$. We prove that $L_A$, defined above, contains all limit points of sequences contained in $A$:\footnote{This assumes the general definition of limit points of a set}
        \begin{proof} Let $\{a_n\}_{n=1}^\infty \subset A$, such that $a_n \rightarrow a$. Then, $a \in L_{\{a_n: n \in \mathbb{N}\}}$ and every neighbourhood of $a$ contains a point of $\{a_n: n \in \mathbb{N}\} - \{a\} = B$. As $\{a_n: n \in \mathbb{N}\} \subset A$, every neighbourhood of $a$ has a point in $A - \{a\}$; $a \in L_A$. 
        \end{proof}

    \item Let $(X, \tau)$ be a topological space, and $A \subset X$ such that $\overline{A} = X$. We prove that $A \cap O \neq \emptyset$, for all $O \neq \emptyset$, $O \in \tau$: 
        \begin{proof} Suppose that for some $O \in \tau$, $O \neq \emptyset$, $A \cap O = \emptyset$. By a previous lemma, we have $\overline{A} = (L_A - A) \cup O^c$, implying $O^c = A$. But then we have $X = O^c = \overline{O^c} = \overline{A}$, as $O$ is open. But, contrarily, this implies that $(O^c)^c = O = X^c = \emptyset$.  
        \end{proof}

    \item We prove that the intersection of a closed set and a compact set is always compact: 
        \begin{proof} Let $(X, \tau)$ be a topological space. Let $H,K \subset X$, such that $H$ is closed and $K$ is compact. Consider $H \cap K$. Now, if $\{O_{\alpha}\}_{\alpha}$ is an open cover of $H \cap K$, then $K \subset \bigcup_{\alpha \in N} O_\alpha \cup (X - H)$. But, since $H$ is closed, $X - H$ is open. In conclusion, as 
            $$H \cap K \subset \bigcup_{\alpha \in N} O_\alpha \cup (X - H)$$
            such a finite subcover of $H \cap K$ exists. 
        \end{proof}

    \item We prove that if $(X, d)$ is a metric space with the induced topology, then $C \subset X$ is closed if, and only if, whenever $\{a_n\}_{n=1}^\infty$ is a sequence in $C$, with $\{a_n\} \rightarrow L$, we have $L \in C$: 
        \begin{proof} 
            \begin{enumerate}
                    $ $\newline
                \item[] ($\implies$): Suppose, to the contrary, that $\overline{C} = C$, but $L \notin C$. Thus, there exists some $\epsilon > 0$, such that $B_\epsilon(L) \cap C = \emptyset$, as $L$ is not a limit point of $C$. But then, $a_n \notin B_\epsilon(L)$, for all $n \geq N$, $N \in \mathbb{N}$ is sufficiently large; A contradiction. 
                \item[] ($\impliedby$): Suppose, to the contrary, that $\overline{C} \neq C$. Then, by extra lemma\footnote{reference this}, there is some $l \in L_C$, such that $L \notin C$. Thus, for each $n \in \mathbb{N}$, we pick $a_n \in B_{\nicefrac{1}{n}}(L) \cap C \neq \emptyset$. Consequently, $\{a_n\}_{n=1}^\infty$ is a sequence in $C$ such that $a_n \rightarrow L \in C$, by hypothesis; A contradiction.
            \end{enumerate}
        \end{proof}

\end{enumerate}



\subsection{Separation}
\begin{enumerate}[(1)]

    \item We show that a compact $T_2$ space $T_3$. Consequently, we show that it is $T_4$: 
        \begin{proof} Let $X$ be the compact $T_2$ space. We first show that $X$ is $T_3$: 
            \par Let $A \subset X$, such that $A$ is closed. Then $A$ is compact. Further, let $b \in X$ such that $b \notin A$. Now, for each $a \in A$, there exists an open set $O_a$, and some open set $O_b^a$, such that $O_a \cap O_b^a = \emptyset$. It follows that $A \subset \bigcup_{a \in N} O_a \subset \bigcup_a O_a$ and that $\bigcup_{a \in N} O_a = O_A$ is open. Further, $b \in \bigcap_{a \in N} O_b^a = O_b$, is open. And by construction, we have $O_A \cap O_b = \emptyset$. So, $X$ is $T_3$. 
            \par To show that $X$ is $T_4$, let $A, B \subset X$, be disjoint and closed. Then, arguing as above, we have two disjoint open subsets $O_B \in \tau$, $O_A \in \tau$ with $O_A \cap O_B = \emptyset$. 
        \end{proof}

\end{enumerate}



\subsection{Compactness}
\begin{enumerate}[(1)]

    \item We show that a homeomorphism between locally compact $T_2$ spaces, $X$, $Y$, extends to a homeomorphism between the Alexandroff compactifications; in other-words, locally compact homeomorphic $T_2$ spaces have homeomorphic one point compactifications. 
        \begin{proof} Suppose that $f: X \rightarrow Y$ is the homeomorphism. Define $g: X \cup \{\infty_1\} \rightarrow Y \cup \{\infty_2\}$ as follows: 
            $$ g(x) = 
            \begin{cases} 
                f(x) & x \neq \infty_1 \\
                \infty_2 & x = \infty_1
            \end{cases}
            $$
            It is clear that as $f$ is a bijection, so is $g$. We show that $g$ is a homeomorphism: 
            \par Suppose that $U$ is open in $Y$. Then, $g^{-1}[U] = f^{-1}[U]$, which is open in $X \cup \{\infty_1\}$. Now, if ${\Ocal}$ is open in $Y \cup \{\infty_2\}$, we have ${\Ocal} = K \cup \{\infty_2\}$, where $K \subset Y$ is compact. Then, 
            $$g^{-1}[{\Ocal}] = g^{-1}[K \cup \{\infty_2\}] = g^{-1}[K] \cup g^{-1}[\{\infty_2\}] =g^{-1}[K] \cup \{\infty_1\}$$
            which is open in $X \cup \{\infty_1\}$. 
            \par Consequently, as $g$ is one-to-one and onto, since $f$ is, $g$ is a homeomorphism by Theorem $3.7$, pg. $48$.
        \end{proof}

    \item We show that if $(X, \tau)$ is a topological space, and $S \subset X$ is compact, then if $W$ is a neighbourhood of $S$, there exists another neighbourhood $G$, such that 
        $$S \subset G \subset W$$
        \begin{proof} By exercise $21$, section $3$ pg. $55$, and the fact that $X \cup \{y\} \cong X$, the result follows.  
        \end{proof}

\end{enumerate}

\subsection{Connectedness}
\begin{enumerate}[(1)]
    \item We show that if $X$ is locally connected, then every connected component of $X$ is open in $X$; hence $X$ is the disjoint union of its connected components:
        \begin{proof} Let $x \in Y$, where $Y$ is a connected component of $X$. By definition, $x$ is contained in some open connected subset $U$ of $X$. Since $Y$ is a maximal connected set containing $x$, we have $x \in U \subset Y$. This shows that $Y$ is open in $X$.
        \end{proof}
\end{enumerate}


\subsection{Quotient Maps \& Quotient Topology}
\begin{enumerate}[(1)]
    \item We show that if $q: X \rightarrow Y$ is a quotient map\footnote{Armstrong does not do a good job describing what the topology on $Y$ is. A simple exercise shows that by letting ${\Ocal}$ be open in $Y$ whenever $q^{-1}[{\Ocal}]$ is open in $X$, we have a topology on $Y$; call this $\tau_{Y}$. Further, Armstrong does not do an adequate job describing what a quotient map is: $q: X \rightarrow Y$ is a quotient map if it is onto, continuous with respect to $\tau_Y$, and such that $g^{-1}[{\Ocal}]$ is open in $X$ implies ${\Ocal}$ open in $Y$. We can summarize by saying that $q$ is onto, and such that ${\Ocal}$ is open in $Y$ iff $g^{-1}[{\Ocal}]$ is open in $X$.}, then the topology of $Y$ is the largest which makes $q$ continuous: 
        \begin{proof} Suppose that $\tau$ was some other topology on $Y$, such that $q$ was continuous. We conclude by showing that $\tau \subset \tau_Y$:
            \par Indeed; if ${\Ocal} \in \tau$, then $q^{-1}[{\Ocal}]$ is open in $X$. But then, $q[q^{-1}[{\Ocal}]] = {\Ocal} \in \tau_Y$. Thus, $\tau \subset \tau_Y$. 
        \end{proof}

    \item Suppose that $(X, \tau)$ is a topological space, and $A$ a non-empty set. We show that if $f: X \rightarrow A$, then the quotient topology on $A$, $\tau_A$ is indeed a topology: 
        \begin{proof} By definition, as noted above, ${\Ocal}$ is open in $A$, if $f^{-1}[{\Ocal}]$ is open in $X$:
            \begin{enumerate}
                \item As $f^{-1}[\emptyset] = \emptyset$, and $f^{-1}[A] = X$, $\emptyset, A \in \tau_A$.
                \item As $$f^{-1}[\bigcup_{j \in J} {\Ocal}_j] = \bigcup_{j \in J} f^{-1}[{\Ocal}_j],$$ we see that $\bigcup_{j \in J} {\Ocal}_j \in \tau_A$.
                \item Lastly, $$f^{-1}[\bigcap_{j =1 }^\infty {\Ocal}_j] = \bigcap_{j =1}^\infty f^{-1}[{\Ocal}_j]$$ So, $\bigcap_{j =1 }^\infty {\Ocal}_j \in \tau_A$. 
            \end{enumerate}
        \end{proof}

    \item We show that projective real n-space, $\mathbb{P}^n$, is Hausdorff\footnote{Credit for the initial idea of the proof goes to Brain M. Scott. In addtion, we use the notation $B(z,z')$ for the open ball of radius $z'$, centered at $z$.}: 
        \begin{proof} Let $q:S^n \rightarrow \mathbb{P}^n$ be the quotient map, and suppose that $u,v \in \mathbb{P}^n$, $u \neq v$; there are such $x,y \in S^n$, such that $q^{-1}[\{u\}] = \{x, -x\}$, and $q^{-1}[\{v\}] = \{y, -y\}$. We exhibit an $\epsilon \in \mathbb{R}$ such that the open neighbourhoods of radius $\epsilon$, centered at $u,v$ are disjoint: 
            \par Let $$\epsilon = \frac{1}{2} \min\{ ||x - y ||, ||x +y||\},$$ and set $U = B(x,\epsilon)$, and $V = B(y, \epsilon)$. It follows form the construction of $U,V$, that $U,V,-U,-V$ are pairwise disjoint, and open neighbourhoods of $x,y,-x,-y \in S^n$. Moreover, as noted above, $q^{-1}(q[U]) = U \cup -U$, and $q^{-1}(q[V]) = V \cup -V$. To conclude, we show that $q[U] \cap q[V]$ are disjoint neighbourhoods for $u,v \in \mathbb{P}^n$. 
            \par Indeed; 
            \begin{align*}
                q[U] \cap q[V] & = q[B(x, \epsilon) \cap S^n] \cap q[B(y, \epsilon) \cap S^n] \\
                & = q[B(x, \epsilon)] \cap q[B(y, \epsilon)] \cap \mathbb{P}^n \\
                & = B(u, q(\epsilon)) \cap B(v, q(\epsilon)) \cap \mathbb{P}^n \\
                & = B_1 \cap B_2 \cap \mathbb{P}^n 
            \end{align*}
            Now, if $B_1 \cap B_2 \neq \emptyset$, then we would have $q^{-1}[B_1 \cap B_2] = U \cap V \neq \emptyset$, contary to assumption. Thus, 
            $$q[U] \cap q[V] = \emptyset \cap \mathbb{P}^n = \emptyset$$
        \end{proof}

\end{enumerate}


\subsection{Topological Groups} \label{chap:sec:sub:topologicalgroups}
\begin{enumerate}[(1)]

    \item Suppose that $(G, \tau, *)$ is a topological group. Fixing $x \in G$, we show that $x{\Ocal}\in \tau$ if, and only if, ${\Ocal} \in \tau$: 
        \begin{proof} 
            $ $\newline
            \begin{itemize}
                \item [] ($\implies$): Suppose that $x{\Ocal} \in \tau$. It follows from pg. $75$, that $L_{x^{-1}}: G \rightarrow G$ is a homeomorphism and an open map: thus, $L_{x^{-1}}[{\Ocal}] \in \tau$.
                \item [] ($\impliedby$): Suppose that ${\Ocal} \in \tau$. Then, as noted on pg. $75$, $L_x: G \rightarrow G$ is a homeomorphism and an open map: thus, $L_x[{\Ocal}] = x{\Ocal} \in \tau$.  
            \end{itemize}
        \end{proof}

    \item Suppose that $(G, \tau, *)$ is a topological group. We show that ${\Ocal}$ is a neighbourhood of $x \in G$, if, and only if, $x^{-1}{\Ocal}$ is a neighbourhood of $e$: 
        \begin{proof} Suppose that $x \in G$, and without loss of generality, ${\Ocal}$ is an open set containing $x$. Then, by a previous lemma\footnote{reference this}, $L_{x^{-1}}[{\Ocal}] = x^{-1}{\Ocal} \in \tau$, and contains $x^{-1}x = e$.
            \par By considering $L_x$, similar logic shows the reverse implication. 
        \end{proof}

    \item We show that $\R$, with the Euclidean topology and addition is a topological group: 
        \begin{proof} The fact that $(\R, +)$ is a group is clear. To conclude, we show that $+ \equiv m:\R \rightarrow \R$, and $-1 \equiv i: \R \rightarrow \R$ are continuous:
            \par Now, $i(x) = -x$, which is polynomial, and so continuous. And, as $m(x,y) = x + y = P_1(x,y) + P_2(x,y)$, where $P_i$ is the projection mapping from $\R^{2}$, $m$ is continuous; it is the sum of two continuous functions. 
            \par A similar argument show that $(\R^{n}, \tau, +)$ is a topological group.
        \end{proof}

    \item Let $(G, \tau, m)$ be a topological group. We show that the topological automorphisms of $(G, \tau, m)$, for a subgroup of $\text(Aut)(G)$. We denote the set of all topological automorphism of $G$ by $\Aut_\tau (G)$. 
        \begin{proof} It is well known that $(\Aut (G), \circ)$ is a group. We show that $(\Aut_\tau (G), \circ)$ is a group: 
            \par As the composition of automorphism/homeomorphisms is another automorphism/homeomorphism, the fact that $(\Aut_\tau (G), \circ )$ is closed is clear. To conclude, as $f \in \Aut_\tau(G)$ is a homeomorphism, $f^{-1} \in \Aut_\tau(G)$. Furthermore, $e: (G, \tau) \rightarrow (G, \tau)$ given by $e(x) = x$ is a homeomorphism. So, $e \in \Aut_\tau(G)$. 
            \par By the subgroup test, 
            $$\Aut_\tau(G) \leq \Aut(G)$$
        \end{proof}

    \item Suppose that $(G, \tau, m)$ is a $T_2$, finite, topological group. We claim $\tau = \mathcal{P}(G)$: 
        \begin{proof} As $G$ is $T_2$, $\{x\}$ is open for each $x \in G$. Thus, as the union of open sets is open, it follows that $\tau = \mathcal{P}(G)$. 
        \end{proof}

    \item We claim that all topological groups of order $2$ are topologically isomorphic. 
        \begin{proof} Let $(G, \tau_G, m)$ be the topological group of order $2$. Consider the topological group $(Z_2, \tau_{Z_2}, +)$ and the map $\phi: G \rightarrow Z_2$, given by $\phi(e_G) = 0$, and $\phi(g) = 1$. The fact that $\phi$ is a group isomorphism is well-known. Further, $\phi$ is 1-1 and onto, and $\phi^{-1}$ is continuous, as $\tau_G = \mathcal{P}(G)$ and $\tau_{Z_2} = \mathcal{P}(Z_2)$. This concludes the proof. 
        \end{proof}
\end{enumerate}


\subsection{Homotopy Type}
\begin{enumerate}[(1)]

    \item We prove that if $A$ is a subspace of a topological space $X$ and  $G: X \times I \rightarrow X$ is deformation retract relative to $A$, then $X$ and $A$ have the same homotopy type\footnote{Armstrong does not do a good job of defining a deformation retract. We use the following definition. "Strong" deformation retract: $A$ is a 'strong' deformation retract of $X$ iff there exists a map $D:X \times I \rightarrow X$ such that $D(a,t)=a$ for every $a \in A$, $D(x,0)=x$ and $D(x,1) \in A$ for all $x \in X$. See \href{http://math.stackexchange.com/questions/488527/deformation-retract-and-homotopy-equivalence}{this post on SE.}}: 
        \begin{proof} Indeed: Let $f:A \rightarrow X$ be the inclusion map and $g: X \rightarrow A$ be defined by $g(x) = G(x,1)$. We have previously shown that $f$ and $g$ are continuous. It is only left to show that $f \circ g \simeq 1_X$ and $g \circ f \simeq 1_A$:
            \par Direct computation shows that 
            \begin{align*}
                (g \circ f)(a) & = g(f(a)) = g(a) = G(a,1) = a,
            \end{align*}
            while $f \circ g \simeq_G 1_X$. This completes the proof.
        \end{proof}

    \item We show that any non-empty convex subset, $X$, of a euclidean space is homotopy equivalent to a point: 
        \begin{proof} 
            Without loss of generality, we assume that $X \subset \E^n$. We first show the existence of a deformation retract, $G: X \times I \hookrightarrow \{x\}$, where $x \in X$ is fixed:
            \par Let $G$ be defined as $G(y,t) = (1-t)y + tx$, for all $y \in X$. Then, clearly $G(y,0) = y$, and $G(y,1) = x \in \{x\}$ for all $y \in X$. So, $G$ is deformation retract. By the above, $X$ and $A$ are of the same homotopy type: null-homotopic. 
        \end{proof}

    \item We show  that $(X, \tau)$ is contractible\footnote{A space $(X, \tau)$ is contractible (to a point $x_0 \in X$) provided that there exists a map $F: X \times I \rightarrow X$ such that $F(x, 0) = x$ and $F(x,1) = x_0$.} if, and only if, every map $f:X \rightarrow Y$, $(Y, \tau_Y)$ a topological space, is null-homotopic: 
        \begin{proof}
            $ $\newline
            \begin{itemize}
                \item[] ($\implies$): Suppose that $(X, \tau)$ is contractible to $x_0 \in X$. Let $F$ be the contraction, and $f: X \rightarrow Y$ any map. The claim is that $f \circ F: X\times I \rightarrow Y$ is a homotopy between $f$, and $f(x_0)$. 
                    \par Indeed; We note that the composition of two continuous functions is continuous and 
                    $$(f \circ F)(x,0) = f(F(x,0)) = f(x)$$ As well as,  
                    $$(f \circ F)(x,1) = f(F(x,1)) = f(x_0)$$

                \item[] ($\impliedby$): Suppose that every map $f:X \rightarrow Y$ is null-homotopic. Then, in particular, $i:X \rightarrow X$ is null-homotopic. Thus, there exists a map $F:X \times I \rightarrow X$, such that $F(x,0) = x =i(x)$, and $F(x,1) = x_0 = i(x_0)$.
            \end{itemize}
        \end{proof}

    \item Let $(X, \tau)$ be a topological space. We show that the cone on $X$, $CX$ is contractible. We consider the "cone tip" as $X \times \{0\}$:     \begin{proof} Consider the function $\overline{H}: (X \times I) \times I \rightarrow X \times I$, given by $\overline{H}((x,v),s) = (x,v(1-s))$. By the     product topology, it follows that $\overline{H}$ is continuous. \footnote{For every $t \in I$, we can associate $\pi(\overline{H}((x,v),t))$ with           $(c,v(1-t),t)$, showing that the product topology of $CX \times I$ agrees with that of $(X \times I) \times I$.} Let $\pi:X \times I            \rightarrow CX = \nicefrac{X \times I}{X \times \{0\}}$ be the canonical map. The claim is that $\pi \circ \overline{H}: (X \times I) \times I     \rightarrow CX$ is the desired homotopy.
            \par Direct computation shows that, for all $x \in X$, $v \in V$,
            $$\pi(\overline{H}( (x,v),0)) = \pi((x,v)) = (x,v),$$
            and 
            $$\pi(\overline{H}( (x,v),1)) = \pi((x,0)) = (x,0)$$
            Thus, $H \equiv \pi \circ \overline{H}$ is the desired homotopy. 
    \end{proof}

\end{enumerate}

\clearpage
\chapter{Introduction} \label{chapter:intro}
\section{Euler's Theorem}
There are no exercises listed for this section.
\section{Topological Equivalence}
There are no exercises listed for this section.
\section{Surfaces}
There are no exercises listed for this section.
\section{Abstract Spaces}
There are no exercises listed for this section.
\section{A Classification Theorem}
There are no exercises listed for this section.

\section{Topological Invariants}
\begin{enumerate}

    \item We prove that $v(T) - e(T) = 1$ for any tree $T$: 
        \begin{proof} We proceed by induction on the number of edges in $T$: 
            \begin{itemize}
                \item Basis: If $e(T) =1$, then $v(T) = 2$, and so,  $v(T) - e(T) = 1$.
                \item Hypothesis: Suppose that  $v(T) - e(T) = 1$ for any tree $T$, such that $e(T) = n$, for some $n \in \N$.
                \item Step: Suppose that $T$ is a tree, such that $e(T) = n+1$. Now, removing any edge will disconnect the tree, since by definition, $T$ contains no loops. Say that we remove and edge $e_0 \in T$, breaking up $T$ into $T_1$, $T_2$. Now, clearly, $v(T) = v(T_1) + v(T_1)$, and $e(T) = n+1 = e(T_1) +e(T_2) +1$. Thus, by the inductive hypothesis,
                    \begin{align*}
                        v(T) - e(T) & = v(T_1) + v(T_2) - e(T_1) - e(T_2) - 1 \\
                        & = 1 + 1 - 1 \\
                        & = 1
                    \end{align*}
            \end{itemize}
        \end{proof}

    \item We show that inside any graph we can always find a tree which contains all the vertexes: 
        \begin{proof} Let $G$ be a finite graph. If $G$ is already a tree, we are done. So, suppose that $G$ is not a tree. Then, by the comments on pg. $3$, $G$ contains a (without loss of generality, minimal) loop, $L$. Now, we remove some edge $e_L \in L$. Now, as $L$ is a loop, $e_L$ does not disconnect $L$, and so $L - e_L$ is a tree. We continue in this way, as long as the new graph formed by removing an edge is not a tree.
            \par To conclude, we note that this process must stop at some loop, because $e(G) < \infty$; further, we have not removed any vertexes. 
        \end{proof}

    \item We prove that $v(\Gamma) - e(\Gamma) \leq 1$ for any graph $\Gamma$, with equality, precisely when $\Gamma$ is a tree: 
        \begin{proof} We have previously shown the equality condition. Using the above proof, we select a subtree, with the same vertexes as $\Gamma$. Call this tree $\Gamma'$. Then, $v(\Gamma') - e(\Gamma') = 1$. As $\Gamma'$ was created by removing edges, and not vertexes, we have $v(\Gamma) = v(\Gamma')$, and $e(\Gamma) \leq e(\Gamma')$. So, 
            $$e(\Gamma) - e(\Gamma) \leq v(\Gamma') - e(\Gamma') = 1$$
        \end{proof}

    \item We find a tree in the polyhedron of Fig. 1.3 which contains all the vertexes and construct the dual graph $\Gamma$ and show that $\Gamma$ contains loops:
        \begin{proof} Please refer to figures \hyperref[fig:sub1.1]{\ref{fig:sub1.1}} and \hyperref[fig:sub1.2]{\ref{fig:sub1.2}} 
            \begin{figure}
                \centering
                \begin{subfigure}{.5\textwidth}
                    \centering
                    \includegraphics[width=.4\linewidth]{armstrong1.jpg}
                    \caption{The tree that contains every vertex.}
                    \label{fig:sub1.1}
                \end{subfigure}%
                \begin{subfigure}{.5\textwidth}
                    \centering
                    \includegraphics[width=.4\linewidth]{armstrong2.jpg}
                    \caption{Blue: Dual graph. Green: loops.}
                    \label{fig:sub1.2}
                \end{subfigure}
            \end{figure}
        \end{proof}

    \item Having done  Problem 4, thicken both $T$ and $\Gamma$ in the polyhedron.  $T$ is a tree, so thickening it gives a disc.  We investigate what happens when you thicken $\Gamma$?
        \begin{proof}  $\Gamma$ is basically two loops connected at a point, with some other edges connected that do not make any more loops.  So thickening should produce something homeomorphic to what is shown in Problem 11 (b).
        \end{proof}

    \item Let $P$ be a regular polyhedron in which each face has $p$ edges and for which $q$ faces meet at each vertex. We use Euler's formula to prove that,
        $$ \frac{1}{p} + \frac{1}{q} = \frac{1}{2} + \frac{1}{e} $$
        \begin{proof} We count the number of faces of $P$ as follows: Since each face of $P$ has $p$ edges, $pf = 2e$. We count the number of vertexes as follows: Since each vertex has $q$ faces which meet on it, $qv = 2e$. In all, $f = \nicefrac{2e}{p}$, and $v = \nicefrac{2e}{q}$. Assuming that Euler's formula holds for $P$, we have 
            $$v - e + f = \frac{2e}{q} - e + \frac{2e}{p} = 2$$
            And the result follows. 
        \end{proof}

    \item We deduce that there are only $5$ regular (convex) polyhedra. 
        \begin{proof} If $P$ is a polyhedra, it must satisfy the above criterion.Further, we assume that $p \geq 3$, since if not, we cannot construct a polygon. We use Shl{\"u}t notation. By checking routinely, for values up to $p,q \leq 5$ we have the following set of polyhedra: 
            $$\{ \{3,3\}, \{3,5\}, \{5,3\}, \{3,4\} , \{4,3\}\}$$
            \par Now, if $p \geq 6$, then we must have
            $$ \frac{1}{6} + \frac{1}{q} = \frac{1}{2} + \frac{1}{e}$$
            Further, as $0 \leq \nicefrac{1}{q} \leq \nicefrac{1}{5}$, we have 
            \begin{align*}
                \frac{1}{6} + \frac{1}{q} & = \frac{q+6}{6q} \\
                & < \frac{12}{6q} = \frac{2}{q} \\
                & < \frac{1}{2} + \frac{1}{e}
            \end{align*}
            A contradiction. 
            \par Please see figures \hyperref[fig:sub1.3]{\ref{fig:sub1.3}}, \hyperref[fig:sub1.4]{\ref{fig:sub1.4}}, \hyperref[fig:sub1.5]{\ref{fig:sub1.5}}, \hyperref[fig:sub1.6]{\ref{fig:sub1.6}}, and \hyperref[fig:sub1.7]{\ref{fig:sub1.7}}.
            \begin{figure}
                \centering
                \begin{subfigure}{.5\textwidth}
                    \centering
                    \includegraphics[width=.4\linewidth]{armstrong8.jpg}
                    \caption{Tetrahedron}
                    \label{fig:sub1.3}
                \end{subfigure}%
                \begin{subfigure}{.5\textwidth}
                    \centering
                    \includegraphics[width=.4\linewidth]{armstrong7.jpg}
                    \caption{Cube}
                    \label{fig:sub1.4}
                \end{subfigure}
                \begin{subfigure}{.5\textwidth}
                    \centering
                    \includegraphics[width=.4\linewidth]{armstrong9.jpg}
                    \caption{Octahedron}
                    \label{fig:sub1.5}
                \end{subfigure}
                \begin{subfigure}{.5\textwidth}
                    \centering
                    \includegraphics[width=.4\linewidth]{armstrong10.jpg}
                    \caption{Dodecahedron}
                    \label{fig:sub1.6}
                \end{subfigure}
                \begin{subfigure}{.5\textwidth}
                    \centering
                    \includegraphics[width=.4\linewidth]{armstrong6.jpg}
                    \caption{Icosahedron}
                    \label{fig:sub1.7}
                \end{subfigure}
                \label{fig:sub1.8}
                \caption{All five platonic solids.}
            \end{figure}                 
        \end{proof}

    \item We check that $v-e+f=0$ for the polyhedron shown in Fig. 1.3 and find a polyhedron which can be deformed into a pretzel and calculate its Euler number: 
        \begin{proof} For the polyhedron in Fig. 1.3, $v=20$, $e=40$, $f=20$.  Therefore $v-e+f=0$.  Figure \hyperref[fig:sub1.9]{\ref{fig:sub1.9}} is basically a donut with two holes, i.e. a "pretzel". 
            \begin{figure}
                \centering
                \includegraphics[width=.4\linewidth]{armstrong3.jpg}
                \caption{"pretzel"}
                \label{fig:sub1.9}
            \end{figure}%
            \par Figure \hyperref[fig:sub1.9]{\ref{fig:sub1.9}} has $38$ faces ($10$ on the top, $10$ on the bottom, $10$ inside the hole, and $8$ around the outside sides).  It has $76$ edges ($29$ on the top, $29$ on the bottom, $10$ vertical ones inside the holes and $8$ vertical ones around the outside sides), and it has $36$ vertexes ($18$ on the top, $18$ on the bottom); Therefore $v-e+f=-2$.
        \end{proof}

    \item This is left to the reader as an exercise. 

    \item We find a homeomorphism from the real line to the open interval $(0,1)$, and show that any two open intervals are homeomorphic:
        \begin{proof} We show that the real line, $\R$ is homeomorphic to $(\nicefrac{-\pi}{2}, \nicefrac{\pi}{2})$.This amount to showing that $arctan: \R \rightarrow (\nicefrac{-\pi}{2}, \nicefrac{\pi}{2})$ is continuous, 1-1, onto and that $tan \equiv arctan^{-1}$ is continuous. Thus, $\R \simeq (\nicefrac{-\pi}{2}, \nicefrac{\pi}{2})$.  
            \par To conclude the proof, we show that any open subset, $(a,b) \subset \R$ is homeomorphic to $(\nicefrac{-\pi}{2}, \nicefrac{\pi}{2})$, $a \neq b$:
            \par Indeed, under the mapping $f(x) = (b-a)(\frac{x}{\pi} +\frac{1}{2}) +a$, we have 
            $$(\nicefrac{-\pi}{2}, \nicefrac{\pi}{2}) \rightarrow (a,b)$$
            As $f$ is a degree $1$ polynomial in $\R$, it follows that it is 1-1, onto and continuous with a continuous inverse. 
        \end{proof}

    \item This is left to the reader as an exercise.

    \item We find a homeomorphism from $S^2 - N = S^2 - \{(0,0,1)\}$ to $\E^2$: 
        \begin{proof} As pointed out in Armstrong, we find a formula for stereographic projection. From the fact that $S^2 = \{x \in \R^3: ||x||_2 =1\}$, and rules of calculus, any line passing through $v = (x,y,z) \in S^2 - N$ and $N$ is of the following form: 
            \begin{equation}
                p(t) = (0,0,1) + t(x,y,z) \equiv <tx,ty,1-t(1-x)>    
            \end{equation}
            For $t \in [0, \infty)$. As the $z$ component of $p$ is zero exactly when $t_0 = \nicefrac{1}{1-z}$, we substitute $t_0$ back in $(1)$ to obtain our formula for stereographic projection: 
            \[\Pi (x,y,z) = \big(\frac{x}{1-z},\frac{y}{1-z} \big) \]
            It is left to show that $\Pi$ is 1-1, onto, continuous, with a continuous inverse. 
            \begin{itemize}
                \item[] 1-1: By component wise comparison, the result follows. 
                \item[] Onto: Given $(x_0,y_0) \in \R^2$, we have, after some simple algebra, 
                    \[ \Pi \Big(\frac{x_0}{r}, \frac{y_0}{r}, \frac{x_0^2 + y_0^2}{r}\Big) = (x_0,y_0),\]
                    where $r = 1+x_0^2 +y_0^2$.  
                \item[] Continuity: As $\Pi$ is a linear transformation in $\R^2$, we know that from rules of calculus/real analysis that it is continuous.
                \item[] Cont. Inverse: $\Pi^{-1}$ is as outline above in the derivation of onto. For the same reasons the function is continuous. 
            \end{itemize}
        \end{proof}

    \item Let $x,y \in S^2$. We find a homeomorphism from $S^2$ to $S^2$ which takes $x$ to $y$. We do the same for the plane, and torus: 
        \begin{proof} We assume that results about matrix groups are applicable. 
            \begin{itemize}
                \item $S^2$: Consider $S^2$ and $SO(3)$. Let $x \in S^2$. Then, by the Gram-Schmidt Orthogonalization process, there exists $X \in SO(3)$ for which $x$ is the first column of $X$. Likewise for $y$, its corresponding matrix $Y$. Then, $YX^{-1} \in SO(3)$ and maps $x$ to $y$. As $SO(3)$ is a group, $(YX^{-1})^{-1} \in SO(3)$. Thus, $YX^{-1}$ is a homeomorphism which takes $x$ to $y$.

                \item $\E^2$: Fix $(x,y),(z,w) \in \R^2$. Then, the function, $f$ that sends $(a,b) \in \R^2$ to \[((x+z)-a, (y+w)-b)\]
                    sends  $(x,y)$ to $(z,w)$. As $f$ is linear, it is continuous. Likewise, its inverse is continuous. 

                \item Torus: Consider the torus as $S^1 \times S^1$. Then component wise examination shows that the example above proves the result. 
            \end{itemize}
        \end{proof}

    \item This is left to the reader as an exercise. 

    \item This is left to the reader as an exercise.

    \item This is left to the reader as an exercise.

    \item Define $f:[0,1) \rightarrow C$, by $f(x) = e^{2\pi i x}$. We prove that $f$ is a continuous bijection. In addition, we find a point $x \in [0,1)$ and a neighbourhood $N$, of $x$ in $[0,1)$, such that $f[N]$ is not a neighbourhood of $f(x)$ in $C$; consequently, this means $f$ is not a homeomorphism. 
        \begin{proof} We first show that $f$ is a continuous bijection: 
            \par Via Euler's formula,
            \[ e^{2\pi i x} = \cos(2\pi x) + i \sin(2\pi x), \quad \forall x \in \R \]
            Using this fact, we show that $f$ is 1-1 and onto. 
            \begin{itemize}
                \item 1-1: Suppose that $\cos(2\pi x) + i \sin(2\pi x) = \cos(2\pi y) + i \sin(2\pi y)$ for some $x, y \in [0,1)$. Then, by component-wise comparison of complex numbers we have $\cos(2\pi y) = \cos(2\pi x)$, iff, $2 \pi y = 2\pi x$, iff $x = y$, as $x,y \in [0,1)$; likewise for $\sin$. Thus, $f$ is 1-1. 

                \item Onto: Let $y \in C$. Then, $y = \cos(2\pi y_0) + i \sin(2\pi y_1)$ for some $(y_0, y_1) \in \R^2$. But as $f: [0,1) \rightarrow C$, we have $0 \leq 2 \pi y_0, 2\pi y_1 \leq 2\pi$. Consequently, $f$ is onto.

                \item Continuity: By the definition of continuity in $\mathbb{C}$, and the fact that $\cos$, $\sin$ are continuous, we have that $f$ is continuous. 
            \end{itemize}

            Next, consider $[0, 1/2) \subset [0,1)$. This is a open neighbourhood in $[0,1)$, by definition. But, $f[[0,1/2)]$ is $C$ intersected with the upper half of the plane minus $z = -1$. Which is not a neighbourhood of $z =1 \in C$, because any open ball centered around $z =1$ must contain the lower plane. Therefore, $f$ is not a homeomorphism. 
        \end{proof}

    \item This is left to the reader as an exercise.

    \item This is left to the reader as an exercise.

    \item We prove that the radial projection of the tetrahedron to the sphere is a homeomorphism: 
        \begin{proof} A solid proof of this relies on the fact that a circumscribed polygon with center of mass $0$ is homeomorphic to $S^n$ via radial projection. The result of which are shown in chapter $5$. 
        \end{proof}

    \item  Let $C$ denote the unit circle in the complex plane and $D$ the disc which it bounds.  Given two points $x,y\in  D-C$, we find a homeomorphism from $D$ to $D$ which interchanges $x$ and $y$ and leaves all points of $C$ fixed:
        \begin{proof} This is more or less intuitively obvious.  But writing down an explicit function is not so easy.  First note that for any $a \in \mathbb C$, the function $f(z)=\frac{z-a}{1-\overline{A}z}$ takes $S^1$ to itself.  To see this suppose $|z|=1$.  then
            \begin{align*}
                \left|\frac{z-a}{1-\overline{a}z}\right|&=\left|\frac{z-a}{\overline{z}z-\overline{a}z}\right|\\
                &=\left|\frac{z-a}{(\overline{z}-\overline{a})z}\right|\\
                &=\left|\frac{z-a}{\overline{z-a}}\right|\frac{1}{|z|}\\
                &=1
            \end{align*}
            If $|a|<1$ then the denominator never vanishes so this is a continuous function on $D$.  Also,
            $f(0)=-a$, so if $|a|<1$ then since $f$ takes $C$ to itself and $0$ maps to $-a\in(D-C)$, $f$ must take all of $D$ to itself.  The inverse $f^{-1}$ is therefore also a continuous function from $D$ to $D$ that takes $C$ to $C$.
            Now suppose $x,y\in\mathbb C$ with $|x|<1$ and $|y|<1$, let
            $$f_1(z)=\frac{z-x}{1-\bar xz}$$
            and
            $$f_2(z)=\frac{z-ty_1}{1-t\bar{y_1}z}$$
            where $y_1=f_1(y)$ and $t=\frac{1-\sqrt{1-|y_1|^2}}{|y_1|^2}$. 
            As shown above, both $f_1$ and $f_2$ take $C$ to itself. 
            Finally let
            $$g(z)=ze^{i(1-|z|)\pi/(1-|x_2|)}$$
            where $x_2=f_2(0)$.
            Then $g(x_2)=-x_2$ and $g(-x_2)=x_2$ and $g(z)=z$ $\forall$ $z\in C$.  Since $z\mapsto|z|$ is continuous, $g$ is built up from sums, products and compositions of continuous functions and therefore $g$ is continuous.
            \par The function $f_1^{-1}f_2^{-1}gf_2f_1$ is therefore a continuous function from $D$ to $D$ that switches $x$ and $y$ and fixes $C$.
        \end{proof}

    \item   With $C,D$ as above (in Problem 21), define $h:D-C\rightarrow D-C$ by
        \begin{align*}
            h(0) & = 0 \\
            h(re^{i\theta}) & =r\exp\left[i\left(\theta+\frac{2\pi r}{1-r}\right)\right]
        \end{align*}
        We show that $h$ is a homeomorphism, but that $h$ cannot be extended to a homeomorphism from $D$ to $D$ and draw a picture which shows the effect of $h$ on a diameter of $D$: 
        \begin{proof} The function $h$ restricted to a circle of radius $r$ acts by rotation of $2\pi r/(1-r)$ radians.  Now $2\pi r/(1-r)\rightarrow\infty$ as $r\rightarrow 1$, so as the circle radius grows towards one, it gets rotated to greater and greater angles approaching infinity.  Thus intuitively it's pretty obvious this could not be extended to the boundary. See figure \hyperref[fig:1.9]{\ref{fig:1.9}}.
            \begin{figure}
                \centering
                \includegraphics[width=.4\linewidth]{armstrong4.jpg}
                \label{fig:1.9}
            \end{figure}
            \par Now, we can think of $(r,\theta)$ as polar coordinates in $\mathbb R^2$.  And the topology on $\mathbb C$ is the same as that on $\mathbb R^2$.    Thus as a function of two variables $r$ and $\theta$ this is just a combination of continuous functions by sums, products, quotients and composition.  Since the only denominator involved does not vanish for $|r|<1$, this is a continuous function of $r$ and $\theta$ on $D$ which is clearly onto.  Since it is a simple rotation on each circle of radius $r$, it is also clearly one-to-one.  The inverse is evidently
            \begin{align*}
                h^{-1}(0) & =0 \\
                h^{-1}(re^{i\theta}) & = r\exp\left[i\left(\theta-\frac{2\pi r}{1-r}\right)\right] \\
            \end{align*}
            \par Now, let $r_n=\frac{n}{n+2}$ for $n$ odd and $r_n=\frac{n-1}{n}$ for $n$ even.  Then for $n$ odd, $\frac{r}{1-r}=\frac n2$, and for $n$ even $\frac{r}{1-r}=n-1$.  Therefore $\exp\left[i\left(\frac{2\pi r_n}{1-r_n}\right)\right]$ equals 1 if $r$ is even and $-1$ if $r$ is odd.  Now, $r_n\rightarrow 1$.  So if $h$ could be extended to all of $D$ we must have $h(1)=\lim h(r_n)$.  But $h(r_n)$ does not converge, it alternates between 1 and $-1$.  Therefore, there is no way to extend $h$ to $C$ to be continuous.
        \end{proof}

    \item  Using the intuitive notion of connectedness, we argue that a circle and a circle with a spike attached cannot be homeomorphic (Fig. 1.26):
        \begin{proof}  In the circle, if we remove any one point what remains is still connected.  However in the circle with a spike attached there is one point we can remove that renders the space not-connected.  Since this property of being able to remove a point and retain connectedness must be a topological property preserved by homeomorphism, the two spaces cannot be homeomorphic.
        \end{proof}

    \item Let $X,Y$ be the subspace of the plane shown in Fig. 1.27.  Under the assumption that any homeomorphism from the annulus to itself must send the points of the two boundary circles among themselves, we argue that $X$ and $Y$ cannot be homeomorphic:
        \begin{proof} The two points that connect the two spikes to the two boundary circles in $X$ must go to the two points that connect the two spikes to the boundary circles in $Y$, because those are the only two points on the boundary circles that can be removed to result in a disconnected space, and because by assumption the circles go to the circles.  Since the two points lie on the same circle in $Y$ but on different circles in $X$, some part of the outer circle in $Y$ must go to the outer circle in $X$ and the rest must go to the inner circle in $X$.  But then some part of the outer circle in $Y$ must go to the interior of $X$. I'm not sure exactly how Armstrong expects us to prove this but it basically follows from the intermediate value theorem, applied to the two coordinates thinking of these shapes as embedded in $\mathbb R^2$. 
        \end{proof}

    \item With $X$ and $Y$ as above, consider the following two subspaces of $\mathbb E^3$:
        $$X\times[0,1] = \{(x,y,z)\mid(x,y)\in X, 0\leq z\leq 1\},$$
        $$Y\times[0,1] = \{(x,y,z)\mid(x,y)\in Y, 0\leq z\leq 1\}.$$
        Convince yourself that if these spaces are made of rubber then they can be deformed into one another, and hence that they are homeomorphic: 
        \begin{proof}With the extra dimension, the squareness can be continuously deformed so that it is a solid torus, with two flat rectangular shapes sticking off.  One has both rectangles pointing out and one has one pointing out and the other pointing in.  Since the torus is round, the first space made from $X$ can be rotated at the location where the inner rectangle is a full half turn to point the rectangle out, and as parallel slices (discs) of the torus move away from where the rectangle is attached, the rotation gradually gets less and less until it becomes zero before reaching the other rectangle.  In this way the inner rectangle can be rotated to point out without affecting the other rectangle and with a gradual change in rotation angle between them guaranteeing the operation is continuous.
        \end{proof}

    \item This is left to the reader as an exercise.

    \item This is left to the reader as an exercise.

\end{enumerate}

\clearpage
\chapter{Continuity}
\section{Open and Closed Sets}
\begin{enumerate}[(1)]

    \item We verify each of the following for arbitrary subsets $A,B$ of a space $X$:
        \begin{itemize}
            \item $\overline{A\cup B}=\overline{A}\cup \overline{B}$:
                \begin{proof} $\overline{A}$ and $\overline{B}$ are closed by theorem ($2.3$).  Thus $\overline{A} \cup \overline{B}$ is closed.  Now $A\subseteq \overline{A}$ and $B\subseteq \overline{B}$.  Therefore $\overline{A} \cup \overline{B}$ is a closed set containing $A\cup B$.  By Theorem 2.3 $\overline{A\cup B}$ is the smallest closed set containing $A\cup B$, thus it must be that $\overline{A\cup B}\subseteq \overline{A}\cup \overline{B}$.  Conversely, $\overline{A\cup B}$ is a closed set that contains $A$, so $\overline{A\cup B}\supseteq\overline{A}$.  Similarly $\overline{A\cup B}\supseteq\overline{B}$.  Thus $\overline{A\cup B}\supseteq\overline{A}\cup\overline{B}$.  Thus $\overline{A\cup B}=\overline{A}\cup\overline{B}$.
                \end{proof}

            \item $\overline{A\cap B} \subseteq \overline{A}\cap \overline{B}$:
                \begin{proof} $\overline{A}$ is a closed set that contains $A\cap B$, so $\overline{A\cap B}\subseteq\overline{A}$.   Likewise $\overline{A\cap B}\subseteq\overline{B}$.  Thus $\overline{A\cap B}\subseteq\overline{A}\cap\overline{B}$.
                    \par To see that equality does not hold, let $A=\mathbb Q$ and let $B=\mathbb R - \mathbb Q$.  Then $A\cap B=\emptyset$, so $\overline{A\cap B}=\emptyset$.  But $\overline{A}=\mathbb R$ and $\overline{B}=\mathbb R$, so $\overline{A}\cap \overline{B}=\mathbb R$.
                \end{proof}

            \item $\bar{\overline{A}}=\overline{A}$:
                \begin{proof} $\bar{\overline{A}}$ is the smallest closed set containing $\overline{A}$ by corollary ($2.4$), and $\overline{A}$ is closed by theorem ($2.3$), that contains $\overline{A}$.  Thus $\overline{A}=\bar{\overline{A}}$.
                \end{proof}

            \item $(A\cup B)^{\circ}\supseteq\overset{\circ}{A}\cup\overset{\circ}{B}$:
                \begin{proof} Let $x\in\overset{\circ}{A}\cup\overset{\circ}{B}$.  Assume, without loss of generality, that $x\in\overset{\circ}{A}$. Then there is an open set $U\subseteq A$ such that $x\in A$.  But then $x\in U\subseteq A\cup B$.  So $x\in\overset{\circ}{A}\cup\overset{\circ}{B}$.  
                    \par Thus 
                    \[\overset{\circ}{A}\cup\overset{\circ}{B}\subseteq(A\cup B)^{\circ}\]
                    \par To see that equality does not hold, let $A=\mathbb Q$ and let $B=\mathbb R - \mathbb Q$.  Then $\overset{\circ}{A}=\emptyset$ and $\overset{\circ}{B}=\emptyset$.  And $A\cup B=\mathbb R$, so $(A\cup B)^{\circ}=\mathbb R$.  Therefore $(A\cup B)^\circ=\mathbb R$ but $\overset{\circ}{A}\cap \overset{\circ}{B}=\emptyset$.
                \end{proof}

            \item $(A\cap B)^{\circ}=\overset{\circ}{A}\cap\overset{\circ}{B}$:
                \begin{proof} Let $U$ be an open set in $A\cap B$.  Then $U\subseteq A$ and $U\subseteq B$.  Thus $(A\cap B)^\circ \subseteq \overset{\circ}{A}\cap\overset{\circ}{B}$.  Conversely suppose $x\in\overset{\circ}{A}\cap\overset{\circ}{B}$.  Then $\exists$ open sets $U$ and $V$ s.t.\ $x\in U\subseteq A$ and $x\in V\subseteq B$.  Then $U\cap V$ is open and $x\in U\cap V\subseteq A\cap B$.  Thus $(A\cap B)^\circ \supseteq \overset{\circ}{A}\cap\overset{\circ}{B}$.  
                    \par Thus 
                    \[(A\cap B)^\circ = \overset{\circ}{A}\cap\overset{\circ}{B}\]
                \end{proof}

            \item $(\overset{\circ}{A})^\circ=\overset{\circ}{A}$:
                \begin{proof} Clearly 
                    \[(\overset{\circ}{A})^\circ\subseteq\overset{\circ}{A}\] 
                    Let $x\in\overset{\circ}{A}$.  Then there exists an open set $U$, such that $x\in U\subseteq A$.  Now $\overset{\circ}{A}$ is a union of open sets so is open.  Let $V=\overset{\circ}{A}\cap U$.  Then $x\in V\subseteq \overset{\circ}{A}$.  Therefore $x\in(\overset{\circ}{A})^\circ$ and so, \[\overset{\circ}{A}\subseteq(\overset{\circ}{A})^\circ\]
                    \par Thus, 
                    \[\overset{\circ}{A}=(\overset{\circ}{A})^\circ\]
                \end{proof}

        \end{itemize}

    \item This is left to the reader as an exercise.

    \item We specify the interior, closure and frontier of the following subsets. 
        \setcounter{equation}{0}
        \begin{align} \
            & \{(x,y): 1 < x^2+y^2 \leq 2\} \\
            & {\E}^2 - \{(0,t), (t,0): t\in \R \} \\
            & {\E}^2 - \{(x, sin(1/x): x>0\}
        \end{align} 

        \begin{itemize}
            \item $\{(x,y): 1 < x^2+y^2 \leq 2\} = A$: 
                \begin{align*}
                    \Ft A & = \overline{A} \cap  \overline{(\E - A)} \\
                    & = \overline{A} \cap  \big\{ \{(x,y): 0 \leq x^2+y^2 \leq 1\} \cup \{(x,y): 2 \leq x^2+y^2\} \big\}\\
                    & = \{(x,y): x^2+y^2 = 1 \lor x^2+y^2 = 2 \}
                \end{align*}
                The rest are left to the reader. 

            \item ${\E}^2 - \{(0,t), (t,0): t\in \R \} = B$: 
                \par From the fact that $\overline{{\E}^2 - B} = \E^2$, we have 
                \[\Ft B = B\]
                The rest are left to the reader. 

            \item ${\E}^2 - \{(x, sin(1/x): x>0\} = C$: 
                \par From the fact that $\overline{{\E}^2 - C} = \E^2$, we have 
                \[\Ft C = C\]
                The rest are left to the reader as an exercise. 
        \end{itemize}

    \item This is left to the reader as an exercise.

    \item We show that if $A$ is a dense subset of a space $(X, \tau)$, and if ${\Ocal} \in \tau$, that ${\Ocal} \subset \overline{A \cap {\Ocal}}$:
        \begin{proof} Suppose, to the contrary, that ${\Ocal} \not\subset \overline{A \cap {\Ocal}}$. Then, there exist some $x \in {\Ocal}$, such that $x \notin \overline{A \cap {\Ocal}}$.
            \par As $\overline{A \cap {\Ocal}}$ is closed, $x \in (\overline{A \cap {\Ocal}})^c$ and so,  there exists some ${\Ocal}_x \in \tau$ such that $x \in {\Ocal}_x$, and 
            \[\overline{A \cap {\Ocal}} \cap (\underset{x}{\Ocal} - \{x\}) = \emptyset \]
            But, as $x \notin \overline{A \cap {\Ocal}}$, we have 
            \[ \overline{A \cap {\Ocal}} \cap {\Ocal}_x = \emptyset \]
            and consequently, $A \cap {\Ocal} \cap {\Ocal}_x = \emptyset$. But then, setting $B = {\Ocal} \cap {\Ocal}_x$, we have $x \in B$, $B \in \tau$, but $A \cap B = \emptyset$, contrary to extra lemma; $\overline{A} \neq X$, a contradiction. 
        \end{proof}


    \item We prove that if $Y$ is subspace of $X$, and $Z$ is a subspace of $Y$, that $Z$ is a subspace of $X$: 
        \begin{proof} The open sets in $Y$ are exactly the sets $\Ocal \cap Y$ where $O$ is open in $X$.  The open sets in $A$ as a subspace of $Y$ are therefore sets of the form $A\cap(Y\cap \Ocal)$ where $O$ is open in $X$.  But $A\subseteq Y$, so $A\cap(Y\cap \Ocal)=A\cap \Ocal$.  Therefore the open sets in $A$ as a subspace of $X$ are exactly the same as the open sets of $A$ as a subspace of $Y$.
        \end{proof}

    \item Suppose that $Y$ is a subspace of $(X, \tau)$. We show that a subset $A$ of $Y$ is closed in $Y$ if it is the intersection of $Y$ with a closed set in $X$. Further, we show that we get the same result if we take the closer in $Y$ or $X$: 
        \begin{proof} If $A \subset Y$ is closed in $Y$, then $Y - A$ is open in $Y$. But then, by the definition of subspace topology, $Y - A = Y \cap {\Ocal}_y$, for some open ${\Ocal}_y \in X$. Consequently, we have 
            \begin{align*}
                A & = Y - (Y \cap {\Ocal}_y) \\
                & = Y \cap (Y \cap {\Ocal}_y)^c \\
                & = Y \cap (X - {\Ocal}_y)
            \end{align*}
            And, as $X - {\Ocal}_y$ is closed in $X$, this proves the result. We note that a similar case holds in the case where $A$ is open.
            \par Letting $\overline{A_y}$ and $\overline{A_x}$ denote the closure of $A \subset Y$ in $Y$ and $X$ respectively, we show that  $\overline{A_y} = \overline{A_x}$: 
            \newline \par  From the previous part of this proof, we have that $\overline{A_y} = Y \cap C$, where $C$ is closed in $X$. Now, since $C$ is a closed set in $X$ containing $A$, we have $\overline{A_x} \subset C$, and so $Y \cap \overline{A_x} \subset \overline{A_y} = Y \cap C$. 
            \par Again by the first problem, we have $Y \cap \overline{A_x}$ is closed in $Y$ and contains $A$. So, 
            \[ \overline{A_y} \subset Y \cap \overline{A_x} \subset \overline{A_x}\]
            Thus, $\overline{A_y} = \overline{A_x}$. 
        \end{proof}

    \item Let $Y$ be a subspace of $(X, \tau)$. Given $A \subset Y$, we show that $A_X^\circ \subset A_Y^\circ$, and give an example when the two may not be equal: 
        \begin{proof} Let $x \in A_x^\circ$. Then, there exists and open ${\Ocal}_x \subset X$, with $x \in {\Ocal}_x$ and ${\Ocal}_x \subset A$. Now, since ${\Ocal}_x \subset A \subset Y$, $x \in {\Ocal}_x \cap Y \subset A$ and, ${\Ocal}_x \cap Y$ is open in $Y$, by definition. Thus, $x \in A_y^\circ$. 
            \par An example when they might not be equal is in the following case: Let $X = \R$, $Y = \mathbb{Z}$, and $A = \{0\}$. Then, $A_x^\circ = \emptyset$. But, every point of $\mathbb{Z}$ is open in the subspace topology, and so $A_y^\circ = \{ 0\}$. 
        \end{proof}

    \item Let $Y$ be a subspace of $X$. We show that if $A$ is open (closed) in $Y$, and if $Y$ is open (closed) in $X$, that $A$ is open (closed) in $X$: 
        \begin{proof} Suppose $A\subset Y\subset X$ and $A$ open in $Y$.  Then $A=Y\cap U$ where $U$ is open in $X$.  The intersection of two open sets is open, so if $Y$ is open in $X$ then $A$ is open in $X$.  Similarly, if $A\subset Y\subset X$ and $A$ closed in $Y$, then $A=Y\cap C$ where $C$ is closed in $X$ (by exercise $7$).  The intersection of two closed sets is closed, so if $Y$ is closed in $X$ then $A$ is closed in $X$
        \end{proof}

    \item We show that the frontier of a set always contains the frontier of its interior, and describe the relationship between $\Ft (A \cup B)$ and $\Ft A$, $\Ft B$: 
        \begin{proof} Let $(X, \tau)$ be a topological space, and let $A \subset X$. We want to show that $\Ft A^\circ \subset \Ft A$. 
            \par Let $x \in \Ft A^\circ$. Then, 
            \[ x \in \overline{A^\circ} \cap \overline{(X - A^\circ)} = \overline{A^\circ} \cup \overline{(X-A) \cup (A - A^\circ)}\]
            Now, if $x \in \overline{A^\circ}$ and $x \in \overline{X -A}$, we are done. So suppose that $x \in \overline{A^\circ}$ and $x \in \overline{(A - A^\circ)}$. But then, $x \in \overline{A^\circ} \cup \overline{(A - A^\circ)} = \overline{A}$. Thus, the result follows.
            \newline \par 
            The inclusion $\Ft (A \cup B) \subset \Ft A \cup \Ft B$ always holds, and is left to the reader. To show that the reverse inclusion does not always hold, consider $X = \R$, and $A = \mathbb{Q}$. Then,
            \begin{align*}
                \Ft (A \cup A^c) & = \Ft \R = \emptyset \\
                & \neq \Ft A \cup \Ft A^c  = \R 
            \end{align*}
        \end{proof}

    \item This main part of this exercise is left to the reader. However, we do show that this topology does not have a countable base: 
        \begin{proof} Let $B = \tau$ be the topology specified. Suppose, to the contrary, that $\{B_n\}_{n = 1}^\infty$ is a countable base for the topology $\tau$. Define the function $f: \R \rightarrow \mathbb{N}$ as follows: for each $x \in \R$, let $f(x) = n$, such that $B_n \subset [x,1+x)$. Now, we show that $f$ is 1-1 to arrive at a contradiction: 
            \par Indeed; Suppose to the contrary, without loss of generality, that $x < y$. Then, if $f(x) = f(y)$, $f(x) = [x, x+1) \subset B_{f(y)} = [y, y+1)$, which is impossible. Thus, $x = y$. 
        \end{proof}

    \item We show that if a topological space $(X, \tau)$ has a countable base for its topology, then $X$ contains a countable dense subset. I.e. A second countable space is separable: 
        \begin{proof} Let $\{B_n\}_n$ be a countable base for $\tau$. By the Axiom of Choice, let $A$ be the collection of elements $\{a_i\}_i$ such that $a_i \in B_i$. The claim is that $\overline{A} = X$. 
            \par Indeed; let ${\Ocal} \in \tau$. Then, ${\Ocal} = \bigcup_j B_j$, where $B_j \in B$. Now, as $A = \bigcup_i x_i$, where $x_i \in B_i$, we have $A \cap {\Ocal} \neq \emptyset$. Thus, by extra lemma, $\overline{A} = X$. 
        \end{proof}

\end{enumerate}


\section{Continuous Functions}
\begin{enumerate}[(1)]
    \item We show that if $f: \R \rightarrow \R$ is a map, then the set of points left fixed by $f$ is closed. Further, the kernel of $f$ is closed.
        \begin{proof} Define $f_0(x) = f(x) - x$. The rest is left to the reader. 
        \end{proof}

    \item We show that the function $h(x) = \frac{e^x}{1+e^x}$ is a homeomorphism from the real line to the open interval $(0,1)$: 
        \begin{proof} Let 
            \[f(x)=\frac{e^x}{1+e^x}, \quad y\in(0,1), \quad x=\ln\left(\frac{y}{1-y}\right)\]
            \par We show that $f$ is onto and 1-1: 
            \begin{itemize}
                \item As $f(x)=y$, $f$ is onto.

                \item Suppose $f(x)=f(y)$. Then, 
                    \[\frac{e^x}{1+e^x}=\frac{e^y}{1+e^y} \text{ implies } e^x+e^{x+y}=e^y+e^{x+y} \]
                    Therefore $e^x=e^y$ and $f$ is 1-1. 
            \end{itemize}
            For $a,b\in\mathbb R$, 
            \[f^{-1}((a,b))=\left(\ln\frac{a}{1-a},\ln\frac{b}{1-b}\right), \]
            which is open.  Since the intervals $(a,b)$ are a basis, it follows from Theorem ($2.9 (b)$) that $f$ is continuous.  And $f((a,b)) = (f(a),f(b))$, so $f$ takes open sets in the base to open sets.  Therefore $f^{-1}$ is continuous, and so, $f$ is a homeomorphism.
        \end{proof}

    \item Let $f: \E \rightarrow \R$ be a map and define $\Gamma_f: \E \rightarrow \E^2$ by $\Gamma_f (x) = (x, f(x))$. Whe show that $\Gamma _f$ is continuous and that its image, with the induced topology, is homeomorphic to $\E$: 
        \begin{proof} We show that $\Gamma_f$ is continuous. To do this, we use the sequential criterion for continuity in $\E^n$. Let $\{x_n\}_n$ be a sequence in $\E$, such that $x_n \rightarrow x \in \E$. Then $\Gamma_f$ is clearly continuous by the fact that $f$ is, and component wise comparison.  That is, 
            \[  \Gamma_f (x_n) = (x_n, f(x_n)) \rightarrow (x, f(x)) = \Gamma_f (x) \]
            Next, we show that $\Image \Gamma_f$ is homeomorphic to $\E$:
            \par We claim that the function $p_1: \Image \Gamma_f \rightarrow \E$, defined by $p_1((x, f(x)) = x$ is the desired homeomorphism. 
            \begin{itemize}
                \item[] Continuity: We have previously shown that the projection map is continuous.
                \item[] 1-1: This is true, by construction. 
                \item[] Onto: This is clear, as $f$ a map from $\E$ to $\E$. 
                \item[] Cont. Inverse: This is clear, as $p_1^{-1} \equiv \Gamma_f$, and we have shown $\Gamma_f$ is continuous. 
            \end{itemize}
            This proves the result.\footnote{A similar argument applies to $\E^n$.} 
        \end{proof}

    \item We determine what topology on $X$ implies that every real-valued function definied on $X$ is continuous: 
        \begin{proof} $X$ must have the discrete topology, where every subset is open. To show this, it suffices to show points in $X$ are open. 
            \par Indeed; Fix $x\in X$ and define $f: X\rightarrow \mathbb R$ by
            \[f(x) =
            \begin{cases}
                f(x) = 0 & \\
                f(y) = 1 & y \neq x 
            \end{cases}
            \]
            Then,
            \[f^{-1}((-1/2,1/2))=\{x\}\] and thus $f$ is continuous, since $(-1/2,1/2)$ is open and $\{x\}$ is open.
        \end{proof}

    \item Consider $X = \R$ with the co-finite topology, $(\R, CO)$. We show that $f: \E \rightarrow  X$ defined by $f(x) = x$ is a map, but not a homeomorphism: 
        \begin{proof} To see that $f$ is continuous, let ${\Ocal} \in CO$. Then, $X - {\Ocal} = \R - {\Ocal}$ is finite. But then, $f^{-1}[{\Ocal}] = {\Ocal}$, and ${\Ocal}^c$ is finite in $\E$. This implies that ${\Ocal} = \overline{{\Ocal}}$. Thus, ${\Ocal}^c = {\Ocal}$ is open in $\E$. The fact that $f$ is 1-1 is clear. 
            \par Consider the inverse function of $f$. Note that $(a,b)$, $a < b$ is open in $\E$. But, $f^{-1}[(a,b)] = (a,b) = f[(a,b)]$, which is not open in $X$. Thus, $f^{-1}$ is not continuous.
        \end{proof}

    \item Suppose $X=A_1\cup A_2 \cup\dots$, where $A_n\subseteq\overset{\circ}{A}_{n+1}$ for each $n$.  We show that if $f:X\rightarrow Y$ is a function such that, for each $n$, $f|A_n:A_n\rightarrow Y$ is continuous with respect to the induced topology on $A_n$, then $f$ is itself continuous:
        \begin{proof} Let $x\in X$.  Then $x\in A_n$ for some $n$ and $A_n\subseteq \overset{\circ}{A}_{n+1}$.  Thus $x\in\overset{\circ}{A}_{n+1}$, and $X=\cup \overset{\circ}{A}_n$.
            \par Now, let $U$ be open in $Y$. Since $f$ is continuous on $A_n$, by theorem ($2.8$), $f$ is continuous on $\overset{\circ}{A}_n$. Consequently, 
            \[f^{-1}(U)\cap\overset{\circ}{A}_n=f|_{\overset{\circ}{A}_n}^{-1}(U)\] 
            is open in $\overset{\circ}{A}_n$. So, there exists open sets $V_n$ such that  \[V_n\cap\overset{\circ}{A}_n=f^{-1}(U)\cap\overset{\circ}{A}_n\]
            And so, 
            \[f^{-1}(U) =f^{-1}(U)\cap \bigcup_n(\overset{\circ}{A}_n)=\bigcup_n(f^{-1}(U)\cap \overset{\circ}{A}_n)=\bigcup_n(V_n\cap \overset{\circ}{A}_n),\]
            which is a union of open sets and so is open.
        \end{proof}

    \item Let $(X, \tau)$ be a topological space, $A \subset X$, and $\chi_A$ its characteristic function. We describe the frontier of $A$ in terms of $\chi$: 
        \begin{proof} We note that if $\overline{(X -A)} \cap \overline{A} \neq \emptyset$, then there exists $a \in \overline{(X -A)}$, $a \in \overline{A}$. With this in mind, we claim that $\chi_A$ is continuous at $a \in X$ if, and only if, $a \notin \Ft A$. 
            \par Indeed; 
            \begin{itemize}
                \item[] ($\implies$): Suppose, to the contrary, that $\chi_A$ is continuous at $a \in X$, but that $a \in \Ft A$. Then, $a \in \overline{X-A}$, and $a \in \overline{A}$. Now, let ${\Ocal} = (.99, 1.99) \subset \R$, without loss of generality. Then, $\chi_A(a) = 1 \in {\Ocal}$. Now, since $\chi_A$ is continuous, $\chi_A^{-1} [{\Ocal}]$ is open, and such that $a \in \chi_A^{-1} [{\Ocal}]$. But then, $\overline{X-A} \cap \chi_A^{-1} [{\Ocal}] \neq \emptyset$. Further, this implies that $X-A \cap \chi_A^{-1} [{\Ocal}] \neq \emptyset$. But then, $\chi_A(a) = 0 \in {\Ocal}$, a contradiction. 
                \item[] ($\impliedby$): 
            \end{itemize}
        \end{proof}

    \item We determine which of the following maps are open or closed: 
        \begin{itemize}
            \item $x \mapsto e^x$, on $\R$: 
                \begin{proof} Open: Let $C=f(\R)$.  Since $f(A\cup B)=f(A)\cup f(B)$, it suffices to check that $f$ is open on a base of open sets.  Let $B$ be an open interval in $\R$.  If $B$ has length greater than $2\pi$ then $f(B)$ is all of $C$.  So $f(B)$ is open in $C$.  Otherwise $f(B)$ is an open arc of the circle.  Also open.  Thus in all cases $f$ maps $B$ to an open set.  Since open balls are a base $f$ must be an open map.
                    \par Not Closed: To show $f$ is not closed, for each $n\in \mathbb N$ let $E_n=[2n\pi+1/n,(2n+1)\pi-1/n]$.  Let $E=\cup_nE_n$.  Suppose $x$ is a limit point of $E$.  Then $(x-1/2,x+1/2)$ intersects at most one $E_n$.  Thus $x$ is a limit point of $E_n$.  Thus $x\in E_n$ since $E_n$ is closed.  Thus $E$ is closed.  But $f(E)$ is equal to $C$ intersected with the upper half plane $\text{im}(z)>0$.   This is an open set in $C$, and not closed since $z=1$ is a limit point of $f(E)$ not in $f(E)$. Thus $f(E)$ is not closed.
                \end{proof}

            \item $f: \E^2 \rightarrow \E^2$, $f(x,y) = (x, |y|)$:  
                \begin{proof} Not Open: To see $f$ is not open, let $D$ be the open unit disc.  Let $H^+=\{(x,y)\mid y\geq 0\}$ and $H^-=\{(x,y)\mid y\leq 0\}$.  Then $f(D)=D\cap H^+$.  Therefore $z=0\in f(D)$ and every open ball containing $z$ intersects ${H^+}^c\subset f(D)^c$.  Thus $f(D)$ is not open.
                    \par Closed: Now, suppose $E$ is closed in $\mathbb E^2$.  Let $E'=E\cap H^-$ and let $E''$ be $E'$ reflected about the $x$-axis. Then $f^{-1}(E)= (E\cap H^+)\cup(E'')$.  Now $E$ and $H^+$ are closed so $E\cap H^+$ is closed.  And $E''$ is closed because $H^-$ is closed, so $E'$ is closed, and reflection is a homeomorphism.  Thus $f^{-1}(E)$ is closed.
                \end{proof}

            \item $z \mapsto z^3$ on $\C$: 
                \begin{proof} Open: We first show it is open.  Let $A_{\theta_1,\theta_2,r_1,r_2}=\{re^{i\theta}\in \mathbb C\mid \theta_1,\theta_2,r_1,r_2\in[0,\infty)\text{ and } \theta_1<\theta<\theta_2, r_1<r<r_2\}$.  Then the set $\beta=\{A_{\theta_1,\theta_2,r_1,r_2}\mid\theta_2-\theta_1<2\pi$ and $0\leq r_1<r_2\}$ form a base for the usual topology on $\mathbb C$.  The sets are clearly open and the intersection of any two of them is another one.  Also for any $z\in A_{\theta_1,\theta_2,r_1,r_2}$ there is an open disc $D$ s.t.\ $z\in D\in A_{\theta_1,\theta_2,r_1,r_2}$ and for any open disc $D$ with $z\in D$ we can find ${\theta_1,\theta_2,r_1,r_2}$ s.t.\ $z\in A_{\theta_1,\theta_2,r_1,r_2}\in D$.   Now consider what happens to $A_{\theta_1,\theta_2,r_1,r_2}$ under the function $f$.  If $\theta_2-\theta_1>2\pi/3$ then $f(A_{\theta_1,\theta_2,r_1,r_2})$ is a full open annulus.  Otherwise $f(A_{\theta_1,\theta_2,r_1,r_2})=A_{3\theta_1,3\theta_2,r_1,r_2}$ which is open.  Thus $f$ takes basic open sets in $\beta$ to open sets.  Since $f(A\cup B)=f(A)\cup f(B)$ for any sets $A$ and $B$, it suffices to check open-ness on the base $\beta$. Thus $f$ is an open map.
                    \par Closed: We now show $f$ is a closed map.  Let $E$ be any closed set in $\mathbb C$.  Let $Q_1$ be the first quadrant $\{x+iy\mid x,y\geq 0\}$.  Let $Q_2,Q_3,Q_4$ be the other (closed) quadrants.  Then $Q_1$ is closed, so $E\cap Q_1$ is closed.  The map $f$ restricted to $Q_1$ is a homeomorphism from $Q_1$ to its image $f|_{Q_1}:Q_1\rightarrow f(Q_1)=Q_1\cup Q_2\cup Q_3$ because its inverse $re^{i\theta}\mapsto re^{i\theta/3}$ is continuous.  Thus $f(E\cap Q_1)$ is closed in $Q_1$.  Likewise $f(E\cap Q_i)$ is closed for all $i=1,2,3,4$.  Functions respect unions, thus since $E=\cup_i(E\cap Q_i)$ it follows that $f(E)=\cup_i f(E\cap Q_i)$.  But each $f(E\cap Q_i)$ is closed.  Thus $f(E)$ is closed.
                \end{proof}
        \end{itemize}

    \item We show that the {\it unit ball}\/ in $\mathbb E^n$ (the set of points whose coordinates satisfy $x_1^2+\cdots+x_n^2\leq1$) and the {\it unit cube} (points whose coordinates satisfy $|x_i|\leq1$, $1\leq i\leq n$) are homeomorphic if they are both given the subspace topology from $\mathbb E^n$: 
        \begin{proof} Note that nowhere in the proof of lemma ($2.10$) is it used that we are in two dimensions.  The proof goes through basically without change to any finite dimension where we replace "disc" with "ball".
            Now, let 
            \[f:\mathbb E^n-{0}\rightarrow\mathbb E^n-{0}\]
            be given by 
            \[f({\bf v})=\frac{1}{||{\bf v}||}{\bf v}\]
            Then $f$ is continuous and the image of $f$ is the unit sphere.  Let $g$ be $f$ restricted to {\it the surface} of the unit cube.  Then $g$ is one-to-one continuous.  The intersection of an open ball with the surface of the cube maps to the intersection of an open ball with the sphere.  Thus $g$ is an open map.  Therefore the inverse of $g$ must be continuous. Therefore $g$ is a homeomorphism.  By the generalization of lemma ($2.10$), $g$ may be extended from the boundaries to a homeomorphism from the whole cube to the whole ball.
        \end{proof}

\end{enumerate}



\section{Space-Filling Curves}
\begin{enumerate}[(1)]
    \item We find a Peano curve which fills out the unit square in $\E^2$: 
        \begin{proof} We apply lemma $2.10$. By a previous exercise, we have that $\partial [0,1]^2$ is homeomorphic to $\partial S^1$. Further, the boundary of the unit triangle is homeomorphic to $\partial S^1$. Therefore, by lemma $2.10$, there is a homeomorphism, $f$, from the unit triangle to the unit disc. Let $h: [0,1] \rightarrow T$ be the space filling curve of the triangle as mentioned in Armstrong. Then, $f \circ h$ is a continuous mapping from $[0,1] \rightarrow [0,1]^2$. Further, it is space filling. 
        \end{proof}

    \item We find an onto, continuous function from $[0,1]$ to $S^2$: 
        \begin{proof} From a previous exercise, we have that $\E^2 \cong_f (0,1) \times (0,1)$. Further, we have shown that $\E^2 \cong_\pi S^2 - (0,0,1)$. Thus, we extend $f$ to $g: [0,1]^2 \rightarrow S^2$ by 
            \[g(x) = 
            \begin{cases}
                f(x) & x \notin \partial [0,1]^2 \\
                (0,0,1) & x \in \partial [0,1]^2
            \end{cases}
            \]
            Then, if ${\Ocal} \subset S^2$ is open, and contains $(0,0,1)$, then $g^{-1}[{\Ocal}]$ is the entire square via homeomorphism. Now, if ${\Ocal} \subset S^2$ does not contain $(0,0,1)$ and is open, then $g^{-1}[{\Ocal}] = f^{-1}[{\Ocal}]$ which is clearly open. 
            \par
            Thus, $g \circ h$, where $h$ is the triangle space-filling curve, is the desired function.
        \end{proof}

    \item We determine whether or no a space-filling curve can fill out the plane: 
        \begin{proof} Note, $[0,1]$ is compact, while $\E^2$ is not. 
        \end{proof}

    \item We determine whether or not a space filling curve ca fill out all of the unit cube in $\E^3$: 
        \begin{proof} Let $I=[0,1]$.  Let $f:I\rightarrow I\times I$ be a space filling curve.  Let $g:I\times I\rightarrow I\times I\times I\times I$ be the map $(x,y)\mapsto (f(x),f(y))$ with the natural identifications $(I\times I)\times(I\times I)$ with $I\times I\times I\times I$. Let $p$ be projection onto the first three coordinates of $I\times I\times I\times I$.  Then $p\circ g\circ f$ is a continuous function from $I$ onto $I^3$.
        \end{proof}

    \item To proved a rigorous proof of this, at this point, is out of the question. However, via theorems ($3.3$), and ($3.7$), it is not true. 

\end{enumerate}

\section{The Tietze Extension Theorem}
Throughout these exercises, we assume that $(X, d)$ is a metric space, and that if $A \subset X$, it has the subspace metric. Further, the topology on $X$ is the induced topology, if not stated otherwise. 
\begin{enumerate}[(1)]
    \item We show that $d(x,A) = 0$ if, and only if, $x \in \overline{A}$:
        \begin{proof} $ $\newline
            \begin{itemize}
                \item[] Suppose that $d(x,A) = 0 = \inf_{a \in A} d(x,a)$. Then, for every $\epsilon > 0$, there exists $a \in A$, such that 
                    \[|d(x,a) - 0| = d(x,a) < \epsilon \]
                    Now, let ${\Ocal}$ be an open subset of $X$, $x \in {\Ocal}$. Choose $a \in A$ such that, $d(x,a) < \epsilon$. Then, $a \in {\Ocal}$, and ${\Ocal} \cap A \neq \emptyset$. Thus, $x \in \overline{A}$. 

                \item[] Suppose that $x \in \overline{A}$. Then, we do the canonical ball construction and pick a sequence $\{a_n\}_n$ of elements of $A$, such that $d(x, a_n) \rightarrow 0$, $n \rightarrow \infty$. 
            \end{itemize}
        \end{proof}

    \item We show that if $A,B \subset X$ are disjoint and closed, there exists disjoint open sets $U,V$ such that $A \subset U$ and $B \subset V$:
        \begin{proof} By lemma $2.13$, there exists a continuous function $f: X \rightarrow [-1,1]$, such that $f[A] \equiv 1$, and $f[B] \equiv -1$. Let $O_1 = [-1,0]$, and $O_2 = [0,1]$. Then, $O_1, O_2$ are open in $[-1,1]$. Thus, $f^{-1}[O_i]$ is open, as $f$ is continuous. Further, $f^{-1}[O_1] \cap f^{-1}[O_2] = \emptyset$, and $A \subset f^{-1}[O_1]$, $B \subset f^{-1}[O_2]$. 
        \end{proof}

    \item We consider what topology the discrete metric gives a space: 
        \begin{proof} We first show $d$ is a metric.  It is real-valued. And clearly 
            \[ d(x,y)\geq 0 \text{ iff } x=y\]
            Also clearly $d(x,y)=d(y,x)$.  Finally, the only way we could have 
            \[d(x,y)+d(y,z)<d(x,z)\]
            is if the left-hand-side is $0$. But then $x=y=z$ and so the right-hand-side is also zero.  Thus $d$ is a metric.  \par Since 
            \[\{x\}=\{y\mid d(x,y)<1/2\},\] 
            the sets $\{x\}$ are open. Since every set is a union of its points, every set is open. Thus this metric gives the discrete topology.
        \end{proof}

    \item We show that every closed subset of a metric space is the intersection of a countable number of open sets: 
        \begin{proof} Let $A$ be a closed subset of $X$. Define 
            \[A_n = \{x \in X : d(x,A) < \frac{1}{n} \}\]
            Then, $A_n$ is open, for each $n \in \mathbb{N}$. The claim is that $\bigcap_n A_n = A$: 
            \newline
            \par Clearly, $A \subset A_n$, for each $n$ so, $A \subset \bigcap_n A_n$. Next, suppose that $x \in \bigcap_n A_n$, but $x \notin A$. Then, 
            \[1 > \inf_{a \in \bigcap_n A_n} d(x,a) = \epsilon > 0\]
            By the Archimedian Principal, there exists $n_0$ large enough so that $\nicefrac{1}{n_0} < \epsilon $. But then, 
            \[x \notin \bigcap_n A_n\]
            Thus, $\bigcap_n A_n \subset A$. 
            \par This concludes the proof. 
        \end{proof}

    \item  If $A,B$ are subsets of a metric space, their {\it distance apart} $d(A,B)$ is the infinum of the numbers $d(x,y)$ where $x\in A$ and $y\in B$.  We find two disjoint closed subsets of the plane which are zero distance apart and check that both of the closed sets which you have just found have infinite diameter: 
        \begin{proof} Let  $A$ be the $x$-axis and $B$ be the set 
            \[\{(x,1/x)\mid x>0\}\] 
            The functions $x\mapsto 0$ and $x\mapsto\frac{1}{x}$ are continuous on $(0,\infty)$ so $A$ and $B$ are closed by chapter $2$, problem $15$.  
            \par Now, let 
            \[\{a_n\}_{n=1}^\infty := (n,0), \quad \{b_n\}_{n=1}^\infty : = (n,1/n)\]  
            Then $a_n\in A$ and $b_n\in B$ and $d(a_n,b_n)=\frac{1}{n}$ and thusly, 
            \[ d(A,B)<\frac{1}{n}, \quad \forall n\]
            Consequently, $d(A,B)=0$.  Both sets clearly have infinite diameter.
        \end{proof}

    \item We show that if $A$ is a closed subset of $X$, then any map $f: A \rightarrow \E^n$ can be extended over $X$: 
        \begin{proof} The is the Tietze extension theorem applied component-wise. 
        \end{proof}

    \item We find a map from $\mathbb E^1 - \{0\}$ to $\mathbb E^1$ which cannot be extended over $\mathbb E^1$:
        \begin{proof}  Let $f(x)=1/x$.  Then $f$ is continuous on $\mathbb E^1-\{0\}$ by Theorem 2.9 (b) because $f^{-1}$ of an open interval is an open interval, or the union of two open intervals.  Now suppose $g$ extends $f$ to all of $\mathbb E$.  Let $a_n=\frac1n$.  Then $a_n\rightarrow 0$.  Thus $g(a_n)\rightarrow g(0)$.  But $g(a_n)=f(a_n)=n\rightarrow\infty$.  Thus no such $g$ can exist.
        \end{proof}

    \item This is left to the reader as an exercise.

    \item Given a map $f:X\rightarrow \mathbb E^{n+1}-\{0\}$ we find a map $g:X\rightarrow S^n$ which agrees with $f$ on the set $f^{-1}(S^n)$: 
        \begin{proof}  Let $h:\mathbb E^{n+1}-\{0\}\rightarrow S^n$ be given by 
            \[\mathbf v\mapsto\frac{1}{||\mathbf v||}\mathbf v\]  
            Then $h$ is continuous and $h$ is the identity on $S^n$.  Let $g=h\circ f$.  Then $g$ is continuous and agrees with $f$ on $f^{-1}(S^n)$.
        \end{proof}

    \item  If $X$ is a metric space and $A$ closed in $X$, we show that a map $f:A\rightarrow S^n$ can always be extended over a {\it neighborhood of} $A$, in other words over a subset of $X$ which is a neighborhood of each point of $A$.  (Think of $S^n$ as a subspace of $\mathbb E^{n+1}$ and extend $f$ to a map of $X$ into $\mathbb E^{n+1}$.  now use Problem 35.):
        \begin{proof} Following the hint we think of $S^n$ as a subspace of $\mathbb E^{n+1}$.  Then $f=(f_1,\dots,f_n)$ and  \[f_i \equiv p_i\circ f,\]
            where $p_i$ is the $i$-th projection. The solution to problem $32$ shows $p_i$ is continuous, so that each $f_i$ is continuous.  
            \par By theorem ($2.15$), each component $f_i$ can be extended to a function $g_i$ on all of $X$ such that  $g_i$ agrees with $f_i$ on $A$.  Then $g=(g_1,\dots,g_n)$ extends $f$ on $A$ to a map from $X$ to $\mathbb E^{n+1}$.
            \par The same argument as in problem $32$ shows $g$ is continuous.   Note that 
            \[g^{-1}(0)\cap A=\emptyset\]
            because $f$ maps $A$ into $S^n$.  And $g^{-1}(0)$ is a closed set in $X$ (theorem ($2.9 (e)$)).  
            \par Thus by problem $28$ we can find {\it disjoint} open sets $U$ and $V$ in $X$ such that 
            \[A\subseteq U, \quad g^{-1}(0)\subseteq V\]  
            Let $h$ be the map from problem $35$.  Then $h\circ g$ is well-defined as long as $g(x)\neq0$.  Thus $h\circ g$ is well-defined on $U$.  And 
            \[h\circ g|_U\] agrees with $f$ on $A$, since $h$ is the identity on $S^n$.
        \end{proof}
\end{enumerate}


\newpage
\chapter{Compactness and Connectedness}

\section{Closed and Bounded Subsets of $\E^n$}
There are not exercises for this section. 

\section{The Heine-Borel Theorem}
\begin{enumerate}[(1)]
    \item This is left to the reader as an exercise.

    \item Let $S\supseteq S_1\supseteq S_2\supseteq \cdots$ be a nested sequence of squares in the plane whose diameters tend to zero as we proceed along the sequence. We prove that the intersection of all these squares consists of exactly one point: 
        \begin{proof} Each $S_n$ is a square so is of the form $I_n\times J_n$ for closed one-dimensional intervals.  And $S_n\supset S_{n+1}$ means $I_n\supset I_{n+1}$ and $J_n\supset J_{n+1}$.  Thus we can apply the one-dimensional argument given above to the $I_n$'s and $J_n$'s.  We get a sequence of points $x_n$ converging to $p$ and $y_n$ converging to $q$. So $(x_n,y_n)\in S_n$ therefore converges to $(p,q)$ which must be in $S_n$ for all $n$ since $S_n$ is closed, so $(p,q)\in\bigcap_{n=1}^\infty S_n$. Similarly the one-dimensional argument shows there can be only a unique $x$ and $y$ coordinate of anything in  $\bigcap_{n=1}^\infty S_n$.  Thus,  
            \[\bigcap_{n=1}^\infty S_n=\{(p,q)\}\]
        \end{proof}

    \item We use the Heine-Borel theorem to show that an infinite subset of a closed interval must have a limit point: 
        \begin{proof} By 'infinite', we assume that the author means that the cardinality of the set is non-finite and countable. We proceed by contradiction: 
            \par Let $C$ be the closed interval. Suppose that such an infinite subset, $A$, of $C$ does not have any limit points. Then, for each $x \in C$, there is an open set ${\Ocal}_x$ in $C$, such that $x \in {\Ocal}_x$ and ${\Ocal}_x \cap (A - \{x\}) = \emptyset$. In addition, 
            \[C \subset \bigcup_{x \in C} {\Ocal}_x \]
            So, $\bigcup_{x \in C} {\Ocal}_x$ is an open cover of $C$, and consequently, has a finite subcover: 
            \[\mathcal{F} = \bigcup_{\substack{x_i  \\ i \in \{1,2, \dots, N\}}} {\Ocal}_{x_i}\]
            Then, $A \subset \mathcal{F}$ and ${\Ocal}_{x_i} \cap A = x_i$ for each $i$. Thus, 
            \[A = \mathcal{F} \cap A = \{x_1, x_2, \dots, x_N\}\]
            A contradiction. 
        \end{proof}

    \item We rephrase the definition of compactness in terms of closed sets:\footnote{Definition; The finite intersection propert (FIP): Let $\mathcal{F}$ be a collection of sets. Then, $\mathcal{F}$ has the finite intersection property if whenever $F_1, F_2, \dots, F_n \in \mathcal{F}$, $F_1 \cap F_2 \cap \dots \cap F_n \neq \emptyset$.} 
        \begin{proof} We claim that $X$ is compact if, and only if, for every collection $\{C_i\}_{i \in I}$ of closed sets in $X$ with the FIP, $\bigcap_i C_i \neq \emptyset$. 
            Indeed:
            \begin{itemize}
                \item[] ($\implies$) We proceed by contradiction: Let $X$ be compact, and let $\{C_i\}_{i \in I}$ be a collection of closed sets with the FIP, such that, $\bigcap_i C_i = \emptyset$. 
                    \par Then, by De-Morgan's Laws, as $C_i^c$ is open, 
                    \[\bigcup_i C_i^c = \Big( \bigcap_i C_i \Big)^c = \emptyset^c = X\]
                    Thus, there exists a finite subcover, 
                    \[\{C_{i_1}^c, C_{i_2}^c, \dots, C_{i_n}^c\},\]
                    with $X = \bigcup_k C_{i_k}^c$. And, so, 
                    \[X^c = \emptyset = \Big(\bigcup_k C_{i_k}^c\Big)^c = \bigcap_k C_{i_k} \neq \emptyset\]
                    However, this is clearly a contradiction. Thus, $\bigcap_i C_i \neq \emptyset$. 

                \item[] ($\impliedby$) Let $\mathcal{F} = \{ {\Ocal}_i\}_{i \in I}$ be an open cover of $X$. Then, $\{{\Ocal}_i^c\}_{i \in I}$ is a collection of closed sets, such that 
                    \[\bigcap_i {{\Ocal}}_i^c = \Big( \bigcup_i {\Ocal}_i \Big)^c = X^c = \emptyset\]
                    Thus, there exists a finite set $\{{\Ocal}_1^c, {\Ocal}_2^c, \dots, {\Ocal}_n^c\}$, such that 
                    $$\bigcap_{k = 1,2, \dots , n} {\Ocal}_k^c = \emptyset $$ 
                    \par But then, 
                    \[\Big(\bigcap_{k = 1,2, \dots , n} {\Ocal}_k^c\Big)^c = \bigcup_{k = 1,2, \dots , n} {\Ocal}_k = (X^c)^c = X\]
                    So, $\mathcal{F}$ contains a finite subcover. 
            \end{itemize}
        \end{proof}
\end{enumerate}

\section{Properties of Compact Spaces}
\begin{enumerate}[(1)]
    \item We determine which of the following are compact: 
        \begin{itemize}
            \item $\Q \subset \E$: 
                \begin{proof} By theorem ($3.9$), a compact subset of $\E$ is closed and bounded, $\Q$ is neither.
                \end{proof}

            \item $S^n$ with a finite number of points removed: 
                \begin{proof} $S^n-\{p_1,\dots p_n\}$ is not compact; by theorem ($3.9$), a compact subset of $\mathbb R^{n+1}$ is closed and bounded.  Since $S^n$ lives in $\mathbb R^{n+1}$, the theorem applies.  $S^n$ is bounded, but $S^n-\{p_1,\dots, p_n\}$ is not closed, because one can find a sequence in $S^n$ that converges to any of the removed points.
                \end{proof}

            \item the torus with an open disc removed: 
                \begin{proof} Yes, the torus with an open disc removed is compact.  The torus can be embedded in $\mathbb R^3$ as a bounded subset.  And since we are removing an open disc, what remains is a closed subset of the torus and therefore (by chapter $2$, problem $7$, page $31$) is a closed subset of $\mathbb R^3$. Therefore, by theorem ($3.9$), it is compact.
                \end{proof}

            \item the Klein Bottle: 
                \begin{proof} The Klein bottle is compact.  It is the continuous image of a closed finite rectangle.  By theorem ($3.9$) a closed finite rectangle is compact. So by theorem ($3.4$) the Klein bottle is compact.    Alternatively, the Klein bottle can be embedded into $\mathbb R^4$ as a closed and bounded set.  Therefore, it is compact. 
                \end{proof}

            \item the M\"obius strip with its boundary circles removed: 
                \begin{proof} The M\"obius strip with its boundary circle removed is not compact.  Think of the strip as a subset of $\mathbb R^3$.  Then one can find a sequence of points in the strip that converge to a point on the boundary, which has been removed.  Since compact sets must be closed in $\mathbb R^3$, the M\"obius with boundary removed cannot be compact.
                \end{proof}
        \end{itemize}

    \item We show that the Hausdorff condition cannot be relaxed in theorem ($3.7$): 
        \begin{proof} Let $X=\{a,b\}$ a set with two points.  Let $X_1$ be $X$ with the discrete topology (so every subset is open).  Let $X_2$ be $X$ with the indiscreet topology (in other words the only open sets are $X$ and $\emptyset$.  Then $X_2$ is not Hausdorff.  The function $f:X_1\rightarrow X_2$ given by $f(a)=a$, $f(b)=b$ is one-to-one, onto and continuous.  But $f^{-1}$ is not continuous, so $f$ is not a homeomorphism.
        \end{proof}

    \item We show that Lebesgue's Lemma fails for $\E^2$: 
        \begin{proof}Let $B(0,1)$ be the open ball of radius one, centered around the origin. And, for each $p \neq 0$ in $\E^2$, let $B(p, \nicefrac{1}{||p||_2})$ be the open ball centered at $p$ and radius $\nicefrac{1}{||p||_2}$. Now, let $\delta > 0$, and let $n \in \mathbb{N}$ be such that 
            \[\frac{1}{n} < \frac{\delta}{3}\]
            Then, the open ball of radius $\nicefrac{2 \delta}{3}$ around the point $(n,0)$ is not contained in any $B(p, \nicefrac{1}{||p||_2})$. 
        \end{proof}

    \item \textit{Lindel{\"o}f's Theorem}: We show that if $X$ has a countable base for its topology, $\tau$, then any open cover of $X$ contains a countable subcover: 
        \begin{proof} Suppose that $\F$ is a countable base for the topology on $(X, \tau)$. Then, $\F = \{F_1, F_2, \dots, F_n, \dots \}$. Let $\{\Ocal_\alpha \}_\alpha$ be an open cover of $X$. Then, as $\F$ is a base for $\tau$, 
            \[\bigcup_\alpha \Ocal_\alpha = \bigcup_{i \in I} F_i\]
            In addition, for each $x \in X$ there exists $n \in I$, and $\alpha_0$, such that 
            \[x \in F_n \subset \Ocal_{\alpha_0}\]
            Thus, let 
            \[\{F_{n_k}\}_k\]
            be the collection of all such $F_{n_k}$, as described above. Then, clearly 
            \[\bigcup_k F_{n_k} = X\]
            and, for each $F_{n_k} \in \{F_{n_k}\}_k$, there is an $\alpha_{n_k}$, such that 
            \[F_{n_k} \subset \Ocal_{\alpha_{n_k}}\]
            Consequently, 
            \[\{ \Ocal_{\alpha_{n_k}} \}_{n_k} \]
            is a subcover of $\{\Ocal_\alpha \}_\alpha$. 
            Finally, as $I$ is countable, so is $\{F_{n_k}\}$ and $\{ \Ocal_{\alpha_{n_k}} \}$.
        \end{proof}

    \item This is shown by an extra lemma, and the fact that compact subsets of $T_2$ spaces are closed.

    \item Let $A$ be a compact subset of a metric space $(X,d)$. We show, 
        \begin{itemize}
            \item that the diameter of $A$, $\Diam A$, is equal to $d(x,y)$ for some $x,y \in A$: 
                \begin{proof} By lemma ($2.13$), for a fixed $y \in A$, the function defined by $f(x) = d(x, \{y\})$ is continuous. By theorem $3.10$, $f$ is bounded, and obtains its bounds on $A$; for some $x_0 \in A$, $d(x, \{y\}) \leq d(x_0, \{y\})$ for all $x \in A$.  Further, ranging over $y \in A$, we see that 
                    \[\Diam A = d(x_0, y_0), \quad x_0, y_0 \in A.\]
                \end{proof}
            \item that given $x \in X$, $d(x,A) = d(x,y)$, for some $y \in A$: 
                \begin{proof} This is lemma ($2.13$), and the fact that $A$ is compact. 
                \end{proof}
            \item that given a closed subset $B$, disjoint from $A$, that $d(A,B) > 0$: 
                \begin{proof} Suppose, to the contrary, that $d(A,B) = 0$. Then, by definition, we have 
                    \[\inf_{\substack{x \in A \\ y \in B}} d(x, y) = d(A,B) = 0\]
                    An application of the above shows that, actually, $d(A,B) = 0 = d(a,b)$ for some $a \in A$, and $b \in B$. But then, by properties of metrics spaces, we have $a = b$; a contraction, as $A \cap B = \emptyset$. 
                \end{proof}
        \end{itemize}

    \item We find an example of a topological space $(X, \tau)$ and a compact subset whose closure is not compact: 
        \begin{proof} Consider $\R = X$, and $\tau = \{\R, \emptyset, (-a,a) : a > 0\}$. Then, $\{0\}$ is clearly compact in $X$, and it is closed. But, $\overline{\{0\}} = \R$ as each $(-a, a)^c = (-\infty, -a] \cup [a, \infty)$ and $0 \notin (-\infty, -a] \cup [a, \infty)$. 
        \end{proof}

    \item We show that the real numbers with the finite complement topology are compact, and that the reals with the half-open topology is not compact: 
        \begin{proof} FC: The real numbers with finite-complement topology is compact.  Let $\beta=\{U_\alpha\}$ be an open cover.  Then let $U_{\alpha_0}$ be an element of $\beta$.  Then there are only a finite number $n$ of points that are not in $U_{\alpha_0}$.  We only need $n$ more elements of $\beta$ to cover everything.
            \par HO: The real numbers with the half-open interval topology is not compact.  Let $\beta$ be the open cover $\{[n,n+1)\mid n\in\mathbb Z\}$.  Then $\beta$ clearly does not have a finite subcover.
        \end{proof}

    \item Let $f: X \rightarrow Y$ be a closed map such that $f^{-1}(y)$, $y \in Y$ is a compact subset of $X$. We show that $f^{-1}[K]$ is compact whenever $K$ is compact in $Y$: 
        \begin{proof} This follows from the fact that 
            \[f^{-1}[K] = f^{-1}[\bigcup_{k \in K} \{k\}] = \bigcup_{k \in K} f^{-1}[\{k\}]\]
            and the fact that the union of finite sets is finite. 
        \end{proof}

    \item This is left to the reader as an exercise. Hint: Use the fact that a subset of a $T_2$ space is $T_2$, and theorem 3.7. Show that $f: X \rightarrow f(X)$ is onto. 

    \item The proof of the first part is left to the reader. However, we show
        \begin{itemize}
            \item that any closed subset of a locally compact space is locally compact: 
                \begin{proof} Let $(X, \tau)$ be the locally compact space, and $A \subset X$, with $\overline{A} =A$. Let $p \in A$. Then, as $X$ is locally compact, there exists a compact neighbourhood \[K_X \subset X, \quad p \in K_X\] 
                    Then, via the subspace topology, $K_X \cap A$ is a neighbourhood of $p$ in $A$. 
                    \par Next, let $\F$ be an open cover of $K_X \cap A$. Then, we see that $\F \cup \{A^c\}$ is an open cover of $K_X$. Further, there exists a finite subcover $\F_1 \subset \F$, such that $\F_1 - \{A^c\}$ is a finite subcover of 
                    \[K_X \cap A \subset \F_1 - \{A^c\}\] 
                    and so $K_X \cap A$ is a compact neighbourhood of $p$ in $A$. 
                \end{proof}
            \item that $\mathbb{Q}$ is not locally compact (as a subset of $\R$): 
                \begin{proof} Suppose, to the contrary, that $\Q$ is locally compact. Then, let $A \subset \Q$ be a compact neighbourhood. Then, there exists an open interval $I \subset \R$, with 
                    \[I \cap \Q \subset A\]
                    Now, let $x \in I$, where $x$ is irrational. As $\Q$ is dense in $\R$, there exists a sequence of rationals in $\Q \cap A$ with, 
                    \[\{q_n\} \rightarrow x \notin A, \quad n \rightarrow \infty \]
                    But then, $A$ is not closed. Thus, by theorem ($3.5$), it cannot be compact. 
                \end{proof}
            \item that local compactness is preserved under homeomorphism: 
                \begin{proof} Suppose that $f: X \rightarrow Y$ is a homeomorphism, and that $X$ is locally compact. We want to show that $Y$ is locally compact. 
                    \par Let $p \in Y$. Then, there is a locally compact neighbourhood $K_p$ of $f^{-1}(p)$ in $X$. Now, let $\F$ be an open cover of $f(\{K_p\})$. Then, 
                    \[\F_1 = \{f^{-1}(\{F\}) : F \in \F \}\]
                    is an open cover of $K_p$ in $X$. It follows that $\F_1$ has a finite subcover, $\F_2$, and that 
                    \[\{f(\{F\}) : F \in \F_2\}\]
                    is a finite subcover for $f(\{K_p\}) \subset Y$. 
                \end{proof}
        \end{itemize}

    \item Suppose that $X$ is locally compact and Hausdorff. Given $x \in X$, and a neighbourhood $\Ocal$ of $x$, we find a compact neighbourhood $K$ of $x$ which is contained in $\Ocal$:  
        \begin{proof} Let $\Ocal$ be as given. Then, since $X$ is locally compact, there exists a compact neighbourhood $V$, such that $x \in V$. By the subspace topology, it follows that $x \in V \cap \Ocal$ and $\Ocal \cap V$ is open in $V$. Further, as $V$ is compact, it is closed, by theorem 3.6. 
        \end{proof}

    \item Let $X$ be a locally compact Hausdorff space which is not compact.  Form a new space by adding one extra point, usually denoted by $\infty$, to $X$ and taking the open sets of $X\cup\{\infty\}$ to be those of $X$ together with sets of the form $\{X-K\}\cup\{\infty\}$, where $K$ is a compact subset of $x$.  We check the axioms for a topology, and show that $X\cup\{\infty\}$ is a compact Hausdorff space which contains $X$ as a dense subset: 
        \begin{proof}  Let $\beta$ be the topology on $X$.  Define 
            \[\beta'=\beta\cup\{U\cup\{\infty\}\mid U\subseteq X\text{ and }U^c\text{ is compact }\}\]
            Clearly, $\emptyset\in\beta'$ and since $\emptyset$ is compact, 
            \[ X \cup \{\infty\}\in\beta'\]
            It remains to show $\beta'$ is closed with respect to arbitrary unions and finite intersection: 
            \begin{itemize}
                \item Arbitrary Unions: Let 
                    \[ \{U_\alpha\}_{\alpha\in A}\subseteq\beta' \subset \beta'\] 
                    We must show $\cup_{\alpha\in A}U_\alpha\in \beta'$;
                    Write 
                    \begin{align*}
                        A & = A_1\cup A_2 \\
                        \alpha\in A_1 & \Rightarrow \infty\not\in U_\alpha \\ 
                        \alpha\in A_2 & \Rightarrow \infty\in U_\alpha
                    \end{align*}  
                    Then, 
                    \[\{U_\alpha\}_{\alpha\in A}=\{U_\alpha\}_{\alpha\in A_1}\cup\{U_\alpha\}_{\alpha\in A_2}\]
                    If $A_2=\emptyset$, then 
                    \[\bigcup_{\alpha\in A} U_\alpha=\bigcup_{\alpha\in A_1} U_\alpha\in\beta\subseteq\beta'\]  
                    If $A_2\not=\emptyset$, then we must show that 
                    \[\left(\bigcup_{\alpha\in A} U_\alpha\right)^c\] is compact.  
                    \par For $\alpha\in A_2$, $U_{\alpha}^c$ is compact: Since $X$ is Hausdorff $U_{\alpha}^c$ is closed (theorem ($3.6$). Thus 
                    \[\bigcap_{\alpha\in A_2}U_\alpha^c\] is closed.  Now, fix any $\alpha'\in A_2$.  Then $\cap_{\alpha\in A_2}U_\alpha^c$ is a closed subset of $U_{\alpha'}^c$ and so is compact (theorem $3.5$).  Thus \[\left(\bigcup_{\alpha\in A_2}U_\alpha\right)^c\] is compact.  Thus,
                    \[\left(\bigcup_{\alpha\in A}U_\alpha\right)^c = \left(\bigcap_{\alpha\in A_1} =  U_\alpha^c\right)\bigcap\left(\bigcap_{\alpha\in A_2} U_\alpha^c\right)\], which is a closed subset of the compact set $\cap_{\alpha\in A_2} U_\alpha^c$ and therefore is compact.

                \item Finite intersections: Now, let 
                    \[\{U_\alpha\}_{\alpha\in A}\subseteq\beta'\] be a finite subset of $\beta'$. As above, decompose $A$ into $A_1\cup A_2$.  If $A_1 \neq \emptyset$, then 
                    \[\infty\in\left(\bigcap_{\alpha\in A}U_\alpha\right)\] And, 
                    \[\left(\bigcap_{\alpha\in A} U_\alpha\right)^c=\left(\bigcap_{\alpha\in A_2}U_\alpha\right)^c=\bigcup_{\alpha\in A_2}U_\alpha^c\] is a finite union of compact sets and therefore compact.  
                    \par If $A_1 \neq \emptyset$, then 
                    \[ \infty\not\in\left(\bigcap_{\alpha\in A}U_\alpha\right)\]  and 
                    \[\left(\bigcap_{\alpha\in A}U_\alpha\right)^c = \left(\bigcup_{\alpha\in A_1} U_\alpha^c\right) \cup \left(\bigcup_{\alpha\in A_2}U_\alpha^c\right)\]
                    \par These are all finite unions, thus this last expression is the union of a closed and a compact and is therefore closed.  Thus $\left(\bigcap_{\alpha\in A}U_\alpha\right)^c$ is closed.  Thus $\left(\bigcap_{\alpha\in A}U_\alpha\right)$ is open.
            \end{itemize}

            \par $X \cup \{\infty\}$ is compact: Let $\{U_\alpha\}$ be an open cover of $X$.  Then $\exists$ $\alpha_0$ such that $\infty\in U_{\alpha_0}$.  Since $U_{\alpha_0}^c$ is compact, and $\{U_\alpha\}$ is an open cover of $U_{\alpha_0}^c$, there are $U_{\alpha_1},\dots,U_{\alpha_n}$ such that 
            \[U_{\alpha_0}^c\subseteq U_{\alpha_1}\cup\dots\cup U_{\alpha_n}\]
            But, then 
            \[U_{\alpha_0},U_{\alpha_1},\dots,U_{\alpha_n}\] 
            is an open cover of $X\cup\{\infty\}$ and so we have found a finite subcover.

            \par  We now show $X\cup\{\infty\}$ is Hausdorff.  Let $x,y\in X\cup\{\infty\}$.  If $x,y\in X$ then since $X$ itself is Hausdorff we know there are open sets that separate them.  So suppose without loss of generality that $y=\infty$.  Since $X$ is locally compact there is an open set $U\subseteq X$ and a compact set $K\subseteq X$ with $x\in U\subseteq K$.  Then $\infty\in K^c$ is an open set in $X\cup\{\infty\}$ and $K^c\cap U=\emptyset$.

            \par  It remains to show $X$ is dense in $X\cup\{\infty\}$.  The only way this is not true is if $\{\infty\}$ is an open set.  But $\{\infty\}^c=X$ which was assumed to be {\it not} compact.
        \end{proof}

    \item We prove that $\E^n \cup \{\infty \}$ is homeomorphic to $S^n$:
        \begin{proof} As before \footnote{reference this}, there exists a homeomorphism 
            \[h: {\E}^n \rightarrow S^n- \{p\}, \quad p \in S^n\]
            Clearly, the one point compactification of $\E^n$ is $\E^n \cup \{\infty \}$.  And further, the one point compactification of $S^n- \{p\}$ is $S^n$. 
            \par As both these spaces are $T_2$, it follows by an extra lemma that 
            \[ {\E}^n \cup \{\infty \} \simeq S^n \]
        \end{proof}

    \item Let $X$, and $Y$ be locally compact Hausdorff spaces, and let $f: X \rightarrow Y$ be an onto map. We show that $f$ extends to a map from $X \cup \{\infty\}$ onto $Y \cup \{\infty\}$, if and only if, $f^{-1}[K]$ is compact for each compact subset $K$ of $Y$. Further, we deduce that if $X$ and $Y$ are homeomorphic spaces, then so are their one-point compactifications and find two spaces which are not homeomorphic, but have homeomorphic one-point compactifications:
        \begin{proof} This problem was first shown as an extra lemma. Thus, we find two spaces which are note homeomorphic but have homeomorphic one point compactifications: 
            \par Consider 
            \[X_1 = [0,1) \cup (1/2, 1] \quad X_2 = [0,1) \]
            and the result follows. 
        \end{proof}



\end{enumerate}


\section{Product Spaces} \label{sec:productspaces}
\begin{enumerate}[(1)]
    \item This is left to the reader as an exercise.

    \item Suppose that $A,B$ are compact in $X,Y$, respectively. We show that if $W$ is a neighbourhood of $A \times B$ in $X \times Y$, that there exists a neighbourhood $U$ of $A$ in $X$, and a neighbourhood $V$ of $B$ in $Y$, such that 
        \[U \times V \subset W\]
        \begin{proof}  Since $A, B$ are compact, by theorem ($3.15$), $A \times B$ is compact. As this is the case, there exists a finite set of $(a,b)$, $a \in A$, $b \in B$, with neighbourhoods 
            \[U_{a_1} \times V_{b_1}, U_{a_2} \times V_{b_2} , \dots , U_{a_n} \times V_{b_n} \]
            such that 
            \[A \times B \subset \bigcup_i U_{a_i} \times V_{b_i}\]
            Now, for each $a \in A$, let 
            \[E_a = \bigcap_{\substack{U_{a_i} \\ a \in U_{a_i} }} U_{a_i},\]
            and 
            \[F_b = \bigcap_{\substack{V_{b_i} \\ b \in V_{b_i} }} V_{b_i}\]
            for each $b \in B$. Then, as $\{U_{a_i} \times V_{b_i}\}_i$ is finite, 
            \[ \{ E_x \times F_y : x \in A, y \in B \}\]
            is finite. Further, we see that 
            \[A \times B \subset \bigcup_{a,b} E_a \times F_b = U \times V\]
            Consequently, $U \times V \subset W$, by construction. 
        \end{proof}

    \item We prove that 
        \begin{itemize}
            \item the product of two second-countable spaces is second-countable: 
                \begin{proof} Let $X,Y$ be second countable spaces. Then, as the finite union of countable sets is countable, it follows that the topology on $X \times Y$ has a countable base. 
                \end{proof}
            \item the product of two separable spaces is separable: 
                \begin{proof} Suppose that $X,Y$ are separable. Then, they contain a countable dense subset, $A \subset X$, $B \subset Y$. From set theory, it follows that $A \times B$ is countable. Thus, we show that $\overline{A \times B} = X \times Y$: 
                    \par Indeed; from exercise twenty on page fifty-five, we have 
                    \[\overline{A \times B} = \overline{A} \times \overline{B} = X \times Y\]
                \end{proof}
        \end{itemize}

    \item We prove that $[0,1) \times [0,1) \simeq [0,1] \times [0,1)$:
        \begin{proof} As previously show, there exists some $h: [0,1) \times [0,1) \rightarrow D_1^2$, 
            \[D_1^2 = D^2 - \{ x = (x_1, x_2) \in S^1 : x_1 > 0 \land x_2 > 0 \},\]
            a homeomorphism. 
            Similarly, under the same map, $h: [0,1] \times [0,1) \rightarrow D_2^2$, 
            \[D_2^2 = D^2 - \{ x = (x_1, x_2) \in S^1 :  x_2 > \sqrt{2}/2 \},\]
            is a homeomorphism. 
            But, $ \{ x = (x_1, x_2) \in S^1 :  x_2 > \sqrt{2}/2 \} = B$,  $\{ x = (x_1, x_2) \in S^1 : x_1 > 0 \land x_2 > 0 \} = A$ are open in $S^1$, and so homeomorphic via, say, $h'$. 
            \par Consequently, 
            \[ S^1 - A \simeq_{h'} S^1 - B\]
            But then, 
            \[D_1^2 = \Int D^2 \cup (S^1 - A) \simeq_{h' \circ i} \Int D^2 \cup (S^1 - B) =  D_2^2 \]
            It follows that 
            \[ [0,1) \times [0,1) \simeq_{h} D_1^2 \simeq D_2^2 \simeq_{h} [0,1] \times [0,1) \]
        \end{proof}

    \item Let $x_0\in X$ and $y_0\in Y$.  We prove that the functions $f:X\rightarrow X\times Y$, $g:Y\rightarrow X\times Y$ defined by $f(x)=(x,y_0)$, $g(y)=(x_0,y)$ are embeddings (as defined in problem $14$): 
        \begin{proof} By theorem ($3.13$), $f$ is an continuous iff $p_1\circ f$ and $p_2\circ f$ are continuous.  Thus, since $p_1 \circ f(x)=x$ (the identity function) and $p_2\circ f(x)=y_0$ (a constant function), it follows that $f$ is continuous.  Now $f$ is clearly one-to-one and onto its image 
            \[ f(X)\subseteq X\times Y\]
            Note that $p_1|_{f(X)}$ is the inverse of $f$ on $f(X)$.  Since $p_1$ is continuous, it follows that $p_1|_{f(X)}$ is continuous on $f(X)$.  Thus $f$ is a homeomorphism.  The proof that $g$ is a homeomorphism is basically identical.
        \end{proof}

    \item The first part is trivial. So, we show that $X$ is Hausdorff if, and only if, $\Delta(\{X\})$ is closed in $X \times X$: 
        \begin{proof} 
            \begin{itemize}
                \item[] ($\implies$) Suppose that $X$ is $T_2$. We show that $\Delta(\{X\}) ^ c$ is open in $X \times X$. Now, for $(x,y) \notin \Delta(\{X\})$, we have that there exists open sets, $U_x, U_y$, such that 
                    \[ x \in U_x, \quad y \in U_y, \quad U_x \cap U_y = \emptyset\] 
                    \par Now, let $U_x \times U_y = W$, which is open in $X \times X$. If $p \in W \cap \Delta(\{X\})$, then $p \in W$, and so $U_x \cap U_y \neq \emptyset$. Thus, 
                    \[W \subset \Delta(\{X\}) ^ c\]
                    and so $\Delta(\{X\})$ is closed. 

                \item[] ($\impliedby$) Suppose that $\Delta(\{X\})$ is closed in $X \times X$. Now, if $(x,y) \notin \Delta(\{X\})$, then $(x,y) \in \Delta(\{X\}) ^ c$, which is open. And so, by definition of the topology on $X \times X$, there exist open sets $U, V$, such that 
                    \[(x,y) \in U \times V \subset \Delta(\{X\}) ^c\]
                    But, then 
                    \[ (U \times V)  \cap \Delta(\{X\}) = \emptyset,\]
                    implies $U \cap V = \emptyset$, by set theory. 
            \end{itemize}
        \end{proof}

    \item This is left to the reader as an exercise. Hint: Consider $f: \R^+ \rightarrow \R$, given by $f(x) = \frac{1}{x}$ and use the sequential criterion for closure. 

    \item Given a countable number of spaces $X_1, X_2,\dots$, a typical point of the product $\Pi X_i$ will be written $x=(x_1,x_2,\dots)$.  The {\it product topology} on $\Pi X_i$ is the smallest topology for which all of the projections $p_i:\Pi X_i\rightarrow X_i$, $p_i(x)=x_i$, are continuous. We construct a base for this topology from the open sets of the spaces $X_1, X_2,\dots$: 
        \begin{proof} Define the base $\beta$ for a topology on $\Pi X_i$ by 
            \[ \beta=\{\emptyset\}\cup\{U_1\times U_2\times \cdots \mid U_i\in X_i\} \] is open, and $U_i=X_i$ for all but finitely many $i\}$.  Clearly, 
            \[\emptyset\in\beta \quad \Pi X_i\in\beta\]
            It is also obvious that $\beta$ is closed under finite intersection since 
            \[ \bigcap_\alpha(A_\alpha\times B)=(\bigcap_\alpha A_\alpha)\times B\]
            \par Now let $U\subseteq X_i$. Then, 
            \[ p_i^{-1}(U)=X_1\times\cdots\times X_{i-1}\times U\times X_{i+1}\times X_{i+2}\times\cdots\in\beta\] 
            Thus, $p_i$ is continuous.
            \par Now, suppose $B$ is the smallest topology for which $p_i$ is continuous.  Then for $U\subseteq X_i$, \[p_i^{-1}(U)=X_1\times\cdots\times X_{i-1}\times U\times X_{i+1}\times X_{i+2}\times\cdots\] must be in $B$. Since any element of $\beta$ is a finite intersection of such sets, it follows that $\beta\subseteq B$.  Since $B$ is the smallest topology for which $p_i$ is continuous for all $i$, and the topology generated by $\beta$ is contained in $B$, it must be that the topology generated by $\beta$ equals $B$.
        \end{proof}

    \item This is left to the reader as an exercise. Hint: Consider the function $f: \R^+ \rightarrow \R$ given by $f(x) = \frac{x}{1+x}$.

    \item We show that
        \begin{itemize}
            \item the box topology contains the product topology: 
                \begin{proof} As $X_i$ is open in $(X_i, \tau_i)$, the result follows. 
                \end{proof}
            \item the box topology and the product topology are equal if, and only if, $X_i$ is an indiscreet space for all but finitely man values of $i$: 
                \begin{proof} If $U_i = \emptyset$, for any $i$, then
                    \[\prod X_i = \emptyset\]
                    And, so the only sets of the form 
                    \[U_1 \times U_2 \times \dots \]
                    which are non-empty, are those which have $U_i = X_i$ for all but finitely many $i$. 
                \end{proof}
        \end{itemize}

\end{enumerate}

\section{Connectedness}
\begin{enumerate}[(1)]
    \item Let $X = \{ (a,b) \in \E^2 : a \in \Q \lor b \in \Q\}$. We show that $X$ with the induced topology is connected: 
        \begin{proof} We seek to apply theorem ($3.25$), on page fifty-eight. Indeed, for each $q \in Q$, let 
            \[X_q = \{ (a,b) \in \E^2 : a = q \lor b = q\}\]
            Then clearly 
            \[X_q = A_q \cup B_q,\]
            where $A_q = \{ (a,b) \in \E^2 : a = q, b \in \E \}$, $B_q = \{ (a,b) \in \E^2 : a \in \E \lor b = q\}$. Further, for each $q \in \Q$, 
            \[A_q \cong \E \cong B_q\]
            So, $A_q$, $B_q$ are connected, and as $A_q \cap B_q = \{(q,q)\}$, $X_q$ is connect by theorem ($3.25$).
            \par For $t,q \in \Q$, we have 
            \[ X_t \cap X_q = \{(q,t), (t,q)\} \neq \emptyset\]
            and so, another application of $3.25$ shows that 
            \[X = \bigcup_{q \in \Q} X_q\]
            is connected. 
        \end{proof}

    \item Give the real numbers the finite-complement topology. We determine the connected components of the resulting space, and answer the same question for the half-open interval topology: 
        \begin{proof} $ $\newline
            \begin{itemize}
                \item FC: The entire space is connected, since no set can be both open and closed.

                \item HO: The space is totally disconnected.  Let $x<y$.  Then by chapter $2$, problem $11$ every set $[a,b)$ is both open and closed. Thus, $[x,y)$ contains $x$ and not $y$, and $[x,y)$ and $[x,y)^c$ are both open.  Thus, $x$ and $y$ are not in the same connected component. Since $x$ and $y$ are arbitrary, connected components cannot contain more than one point.
            \end{itemize}
        \end{proof}

    \item If $X$ has only a finite number of components, we show that each component is both open and closed and find a space none of whose components are open sets: 
        \begin{proof} $X = C_1\cup\cdots\cup C_n$ a disjoint union of connected components.  By thereom ($3.27$), each $C_i$ is closed.  Since a finite union of closed sets is closed, $X - C_i=\cup_{j\not=i}C_j$ is closed.  Thus, 
            \[C_i= \left(\bigcup_{j\not=i}C_j\right)^c\] is open.  
            \par The space $\mathbb Q$ with the natural metric topology is totally disconnected, but points in $\mathbb Q$ are not open sets.
        \end{proof}

    \item (Intermediate Value Theorem) We show that if $f: [a,b] \rightarrow \E$ is a map such that $f(a) < 0$ and $f(b) > 0$, then there exists some $c \in [a,b]$ for which $f(c) = 0$: 
        \begin{proof} By theorem ($3.21$), page fifty-eight, $f([a,b]) \subset \E$ is connected. By theorem ($3.19$), page fifty-seven, $f([a,b])$ is an interval. In particular, $f([a,b]) = [f(a'), f(b')]$, where $f(a')$, $f(b')$, are the minimum and maximum of $f$ on $[a,b]$. Thus, $[f(a), f(b)] \subset [f(a'), f(b')] $, and $0 \in [f(a), f(b)]$. Thus, there exists $c \in [a,b]$, $f(c) = 0$. 
        \end{proof}

    \item We show that
        \begin{itemize}
            \item $\E^n$ is locally connected: 
                \begin{proof} As has been shown, 
                    \[ \{I_1 \times I_2 \times \dots \times I_n : I_i \in \tau_{\E} \}\]
                    is a base for the topology in $\E^n$. Thus, if $x \in \Ocal_x$ for some open neighbourhood $\Ocal_x$ in $\E^n$, there exists 
                    \[I = I_1 \times I_2 \times \dots \times I_n \subset \Ocal_x\]
                    And, as each $I_i$ is connected, it follows by exercise 3.20, that $I$ is connected. 
                \end{proof}

            \item $X = \{0\} \cup \{1/n : n \in \N\}$ is \textit{not} locally connected: 
                \begin{proof} Let $\Ocal$ be an open neighbourhood of $0 \in X$, which is connected. But then, by theorem ($3.19$), $\Ocal$ is an interval. However, $X$ clearly contains no intervals. 
                \end{proof}
        \end{itemize}

    \item We show that local connectedness is preserved by a homeomorphism, but need not be preserved by a continuous function: 
        \begin{proof} The first part of this proof is direct. For the second part, consider $Y = \{0\} \cup \{1/n: n \in \N\}$ as above, and $X = \mathbb{Z}^+$ as subspaces of $\E$. Let $f: X  \bij Y$ be the canonical map. Then, $f$ is clearly continuous, but we have previously show that $Y$ is not locally compact. 
        \end{proof}

    \item We show that $(X, \tau)$ is locally connected, if, and only if every component of each open subset of $X$ is an open set: 
        \begin{proof} 
            \begin{itemize}
                \item[] ($\implies$): Suppose that $X$ is connected, and let $\Ocal$ be an open set in $X$, and $C$ a (maximal) connected component of $\Ocal$. Then, for some $x \in C$, there exists a connected open set $V_x$, such that 
                    \[x \in V_x \subset \Ocal\]
                    But, since $C$ is maximally connected, $V_x \subset C$. Then, 
                    \[C = \bigcup_{x \in C} V_x \]
                    and so $C$ is open. 

                \item[]  ($\impliedby$): Now, let $x \in X$, and $\Ocal$ an open set containing $x$. Then, the components of $\Ocal$ are open. Thus, let $V$ be a component such that $x \in V \subset \Ocal$. But then, $V$ is open, closed, and so connected. Thus, $X$ is locally connected. 
            \end{itemize}
        \end{proof}

\end{enumerate}

\section{Joining Points by Paths}
\begin{enumerate}[(1)]
    \item We show that the continuous image of a path-connected space is path-connected: 
        \begin{proof} Without loss of generality, suppose that $f: X \rightarrow Y = f(X)$ is onto, and $X$ is path-connected. Then, for 
            \[y_1 = f^{-1}(x_1), \quad  y_2 = f^{-1}(x_2)\]
            in $Y$, there exists a path $h$ from $x_1$ to $x_2$. Thus, $f \circ h$ is the desired path.   
        \end{proof}

    \item This is left to the reader as an exercise. Hint: Consider the case $x \neq -y$, and use the straight-line homotopy composed with the canonical map. In the case where $x = -y$, pick a third point which is not equal and apply the above technique.  

    \item We prove that the product of two path-connected spaces is path-connected: 
        \begin{proof} This follows directly from the definition of the product topology and set theory. 
        \end{proof}

    \item If $A$ and $B$ are path-connected subsets of a space $(X, \tau)$, and $A \cap B$ is non-empty, we show that $A \cup B$ is path-connected: 
        \begin{proof} Suppose, without loss of generality, that 
            \[x,y \in A \cup B, \quad x \in A, \quad x \in B\]
            Pick $c \in A \cap B$, and by assumption, there exists paths $\gamma_1$ in $A$, $\gamma_2$ in $B$, which connect $x$ to $c$, and $c$ to $y$, respectively. 
            \par It follows that $\gamma_1 \circ \gamma_2$ is the desired path. 
        \end{proof}

    \item We find a path-connected subset of a space whose closure is not path-connected:
        \begin{proof} As the comments on page sixty-two point out, letting 
            \[Z = \{(x, \sin(\pi/x): x \in \R^+\}\]
            we see 
            \[\overline{Z} = Y \cup Z = X\]
        \end{proof}

    \item We show that any indiscreet space is path-connected: 
        \begin{proof} As any function on an indiscreet space is continuous, the result follows. 
        \end{proof}

    \item We determine whether or not the space shown in fig. 3.4, page sixty-three, is locally path-connected, and covert $X = \{0\} \cup \{ 1/n : n = 1,2, \dots \}$ into a subspace off $\E^2$ which is path-connected but not locally path-connected:
        \begin{proof} It is not the case that fig. 3.4 is locally path-connect. To see this, we note that any path at the origin must contain points from $Y$ and $Z$. As the comments on page sixty-two point out, the result follows.  
            \par For the second part of the proof, let 
            \[X_0 = \{(0,y) : y \in [0,1] \}, \quad X_n = \{(1/n, y) : y \in [0,1]\}\]
            Further, let $Y = \{(x,0) : x \in [0,1] \}$ and
            \[X = Y \cup \Big( \bigcup_n X_n \Big)\] 
            Then, $X$ is path connected as $Y \cap X_n \neq \emptyset$, for all $n$. \par To see that $X$ is not path connected, let $p \in (0,1)$, and consider $\Ocal \subset B(p,1)$ be open. Then, $\Ocal \cap Y = \emptyset$, so $\Ocal \cap X$ is a collection of line segments separate from each other. So, $V \cap X$ is not path connected. 
        \end{proof}

    \item We prove that a space which is connected, and locally path-connected is path-connected: 
        \begin{proof}  Let $x \in X$. And let $\Ocal_x$ be the set of all $y \in X$ such that $y$ is path-connected to $x$; then, $\Ocal_x \neq \emptyset$. We claim that $\Ocal_x = X$. 
            \par Indeed; as $\Ocal_x$ is a maximally connected component it is open, by extra lemma\footnote{cite this}. Similarly, $\Ocal_x^c$ is open. Thus, $\Ocal_x$ is both open and closed. Therefore, as $X$ is connected, $\Ocal_x = X$. This completes the proof. 
        \end{proof}

\end{enumerate}

\newpage
\chapter{Identification Spaces}
\section{Constructing the M\"obius Strip}
There are no exercises listed for this section. 

\section{The Identification Topology}
\begin{enumerate}[(1)]
    \item  We check that the three descriptions of ${\RP}^n$ all lead to the same space: 
        \begin{proof} We first note that the canonical construction of $\RP^n$ is formed via the relation 
            \[ x \sim_\mathcal{R} \lambda x \quad \forall \lambda \neq 0, \quad \lambda \in \R\]
            As $\lambda$ can always be chosen so that $||x \cdot \lambda || = 1$, we consider such $\lambda$ as equivalence class representatives. 
            \par To see that (a), (b) are equivalent, let $h: \E^{n+1} \rightarrow \E / \sim_{\mathcal{R}}$ be the aforementioned quotient map and $i: S^n \rightarrow \E^{n+1}$ be the natural embedding. Then, as $S^n$ is compact, $\E / \sim_{\mathcal{R}}$ is Hausdorff, it follows by an extra lemma, and corollary ($4.4$), that $h \circ i$ is an identification map. So, (a), (b) are equivalent. 
            \par To see that (b) and (c) are equivalent, let $i: S^n \rightarrow B^n$ be the natural embedding, and $g$ is quotient map given in (c). Then, again, by corollary ($4.4$), $g \circ i$ is an identification map. Thus, (b) and (c) are equivalent. 
        \end{proof}

    \item We determine which space do we obtain if we take a M\"obius strip and identify its boundary circle to a point: 
        \begin{proof} We obtain the projective plane $\RP^2$.  To see this, let $R$ be a rectangle as in Fig.\ 4.1 (page 66).  Instead of first identifying the right and left edges to get a M\"obius and then collapsing the boundary circle, we first collapse the top and bottom borders to two points $a$ and $b$, then identify the (image of) the right and left edges as they would have been identified to form a M\"obius from $R$.  This operation identifies the two points $a$ and $b$.  So after these two identifications we end up with the same space.  So by this argument we can do the identifications in the opposite order.   Proceeding in this way, we first collapse the top horizontal border to one point and the bottom border to another point.  From this identification we get a space homeomorphic to a disc $D$.  The identification map $f:R\rightarrow D$ can be seen as taking the left border of $R$ to the left semi-circle of the boundary of $D$ and the right border to the right semi-circle of the boundary of $D$.  Then we follow $f$ with the identification map from $D$ to $\RP^2$ given in Problem 1 above.  The result is an identification map from $D$ to $\RP^2$ that (since it identifies antipodal points on the boundary) identifies the same two points on the two boundary semicircles that came from two points in $R$ that would be identified if we formed the M\"obius first.  Thus the identification space is homeomorphic to the image $\RP^2$.
        \end{proof}

    \item Let $f:X\rightarrow Y$ be an identification map, let $A$ be a subspace of $X$ and give $f(A)$ the induced topology from $Y$. We show that the restriction $f|_A:A\rightarrow f(A)$ need not be an identification map: 
        \begin{proof} Let $X=[0,1]$ and $Y=S^1\subset \mathbb C$ and let $f:X\rightarrow Y$ be given by 
            \[ x\mapsto e^{2\pi i x}\]
            Then, by corollary ($4.4$), $f$ is an identification map because $X$ is compact and $Y$ is Hausdorff.  
            \par Let $A=[0,1)$. Then, $f|_A(A)=Y$, but $f|_A$ does not identify any points.  Thus, if $f|_A$ were an identification map it would (by theorem ($4.2$)) induce a homeomorphism from $A$ to $Y$. But we know $f$ is not a homeomorphism\footnote{See section $1.4$, example $3$, page $14$}.
        \end{proof}

    \item With the terminology from the previous problem, we show that if $A$ is open in $X$, and if $f$ is an open map then $f|_A : A \rightarrow f(A)$ is an identification map: 
        \begin{proof} This is theorem ($4.3$), page sixty-seven. 
        \end{proof}

    \item Let $X$ denote the union of all circles of the form 
        \[\Big(x - \frac{1}{n}\Big)^2 + y^2 = \Big(\frac{1}{n}\Big)^2, \quad n \in \N\]
        with the induced topology. Let $Y$ denote the identification space obtained from the real line b identifying all the integers to a single point. We show that $X \not\cong Y$: 
        \begin{proof} We claim that $X$ is compact, but $Y$ is not. 
            \par Indeed; To show that $X$ is compact, first let $\{ \Ocal_\alpha \}_\alpha$ be an open cover of $X$. Then, there exists some $\alpha_0$, such that 
            \[\{(0,0)\} \subset \Ocal_{\alpha_0} \]
            Thus, as $\Ocal_{\alpha_0}$ is an open set containing $\{(0,0)\}$, there exists some $n_0 \in \N$ for which 
            \[\Big(x - \frac{1}{n}\Big)^2 + y^2 \subset B(\Ocal_{\alpha_0}), \quad n \geq n_0 \]
            where $B(\Ocal_{\alpha_0})$ denotes the solid ball with boundary $\Ocal_{\alpha_0}$. Consequently, only finitely many circles are outside $B(\Ocal_{\alpha_0})$. Thus, compactness follows.
            \par To show that $Y$ is not compact, consider the open cover of $\R$ given by 
            \[\{(n -1/2, n+1+1/2) \}_{n \in \Z} \]
            There is clearly no finite subcover of $\R$. This completes the proof. 
        \end{proof}

    \item We given an example of an identification map which is neither open nor closed: 
        \begin{proof} Let $\pi_1: \R \times \R \rightarrow \R$ be projection onto the first coordinate and consider
            \[A = \{(x,y) : x \geq 0 \lor y = 0 \} \subset \R \times \R\]
            We show that $\pi_1|_A$ is a quotient map which is neither open nor closed: 
            \par Indeed; We know that $\pi_1$ is a quotient map, as it send open sets to open sets. In addition, $\pi_1|_A$ is obviously saturated and so a quotient map. To see that it is neither open nor closed, consider 
            \[C = \{ (x, 1/x) : x \in \R - \{0\} \}\]
            which we have previously shown is closed. Then, clearly $C \cap A$ is clearly closed in $A$, but 
            \[\pi_1 |_A [C \cap A] = (0,\infty)\]
            which is open. And, 
            \[V = A \cap (\R \times (-1, \infty))\]
            is open in $A$, but 
            \[\pi_1|_A [V] = [0, \infty) \]
            is closed in $\E$. This proves the result. 
        \end{proof}

    \item This is left to the reader as an exercise.

    \item Suppose that $X$ is a compact $T_2$ space. We show that 
        \begin{itemize}
            \item the cone on $X$ is homeomorphism to the one-point compactification of $X \times [0,1)$: 
                \begin{proof} Note that, as $X$ is compact, $X \times [0,1]$ is compact. Further, 
                    \[X \times [0,1] = X \times [0,1) \cup X \times \{1\}\] 
                    As the one point compactification of $X \times [0,1)$ is $X \times [0,1]$ and $X \times \{1\}$ is compact in $X \times [0,1]$, the result follows. 
                \end{proof}

            \item if $A$ is closed in $X$, then $X/A$ is homeomorphic to the one-point compactification of $X - A$: 
                \begin{proof} Let $Y = X - A \cup \{\infty\}$ be the one point compactification of $X - A$. Further, let $P$ be the image of $A$ in $X/A$, under the identification map 
                    \[i : X \rightarrow X/A \]
                    We show that the function $h: X/A \rightarrow Y$ given by 
                    \[h(x) = 
                    \begin{cases}
                        x & x \in X - A \\
                        \infty & x \in P 
                    \end{cases}
                    \]
                    is a homeomorphism. We apply theorem ($3.7$) and claim that $h \circ i$ is continuous. 
                    \par Indeed; If $\Ocal$ is open in $Y$ and $\infty \notin \Ocal$, then $\Ocal \subset X - A$, and so, 
                    \[(h \circ i)^{-1}(\Ocal) = \Ocal \subset X\]
                    is open. Now, if $\infty \in \Ocal$, then $\Ocal^c$ is compact in $X - A$ by definition and is closed. Thus, $(h \circ i)^{-1}(\Ocal^c)$ is closed in $X$. Consequently, 
                    \[(h \circ i)^{-1}(\Ocal^c)^c = (h \circ i)^{-1}(\Ocal) \]
                    is open in $X$. So, $h$ is continuous.  
                    \par As $X - A \cup \{\infty\}$ is $T_2$ and $X - A$ compact (it is the continuous image under $i$ of $X$), $h$ is a homeomorphism by theorem ($3.7$), page forty-eight. This completes the proof. 
                \end{proof}
        \end{itemize}

    \item Let $f: X \rightarrow X'$ be a continuous function and suppose that we have partitions $\mathscr{P}$, $\mathscr{P}'$ of $X$ and $X'$ respectively, such that if two points of $X$ lie in the same member of $\mathcal{P}$, their images under $f$ lie in the same member of $\mathscr{P}'$. We show that if $Y$, $Y'$ are the identification spaces given by these partitions, that $f$ induces a map $f': Y \rightarrow Y'$, and that if $f$ is an identification map then so is $f'$: 
        \begin{proof} Let $\pi : X \rightarrow Y$, $\pi': X' \rightarrow Y'$ be the given identification maps. We show that $f$ induces a map $f': Y \rightarrow Y'$: 
            \par Define $f': Y \rightarrow Y'$ by $f'(\pi(\mathcal{P})) = \pi' (f(\mathcal{P}))$, for all $\mathcal{P} \in \mathscr{P}$. To show that $f'$ is a map, we first show that it is well-defined. For all $\mathcal{P} \in \mathscr{P}$, there exists $\mathcal{P}' \in \mathscr{P}'$, such that $f(\mathcal{P}) \subset \mathcal{P}'$, by assumption. The fact that $f'$ is well-defined follows and everywhere defined is clear. 
            \par To show that $f'$ is a map, let $\Ocal$ be open in $Y'$. Then, we have 
            \[f'^{-1}(\Ocal) = (\pi' \circ f)^{-1}(\Ocal) = f^{-1}(\pi'^{-1}(\Ocal)),\]
            which is open as $f$ is continuous. Thus, $f'$ is a map. 
            \par To conclude the proof, we show that if $f$ is an identification map, then so is $f'$: Indeed; Let $\Ocal$ be open in $Y$. We claim that $f'(\Ocal)$ is open in $Y'$. By definition, if $\Ocal$ is open in $Y$, 
            then as $\pi$ is an identification map, $\pi^{-1}(\Ocal)$ is open in $X$. Further, 
            \[(\pi' \circ f \circ \pi^{-1})(\Ocal) = \pi'(f(\pi^{-1}(\Ocal))) = f'(\pi(\pi^{-1}(\Ocal))) = f'(\Ocal) \]
            and, 
            \[f^{-1}(\pi'^{-1}(f'(\Ocal))) = f^{-1}(\pi'^{-1}(\pi'(f(\pi^{-1}(\Ocal))))) = \pi^{-1}(\Ocal) \]
            Thus, by definition of identification map, $f'(\Ocal)$ is open in $Y'$. This concludes the proof.
        \end{proof}

    \item Let $S^2$ be the unit sphere in $\mathbb E^3$ and define $f:S^2\rightarrow \mathbb E^4$ by $f(x,y,z)=(x^2-y^2,xy,xz,yz)$. We show that $f$ induces an embedding of the projective plane in $\mathbb E^4$: 
        \begin{proof}  Let $X=\text{im}(f)\subseteq\mathbb E^4$.  Since $\mathbb E^4$ is Hausdorff and a subspace of a Hausdorff space is Hausdorff, $X$ is Hausdorff.  And $S^2$ is compact.  Thus by Corollary 4.4 $f:S^2\rightarrow X$ is an identification map.  By the example ``Projective spaces (a)" on page 71, $\mathbb P^2$ is $S^2$ with antipodal points identified.  Thus by Theorem 4.2 (a) we will be done if we show $f$ identifies antipodal points (and no others).  It clearly does identify antipodal points, $f(x,y,z)=f(-x,-y,-z)$.  So suppose $f(x_1,y_1,z_1)=f(x_2,y_2,z_2)$.  We must show $(x_1,y_1,z_1)=(x_2,y_2,z_2)$ or $(x_1,y_1,z_1)=(-x_2,-y_2,-z_2)$.  We have
            \begin{equation}
                x_1^2-y_1^2=x_2^2-y_2^2
            \end{equation}
            \begin{equation}
                x_1y_1=x_2y_2
            \end{equation}
            \begin{equation}
                x_1z_1=x_2z_2
            \end{equation}
            \begin{equation}
                y_1z_1=y_2z_2
            \end{equation}
            From (3) and (4) it follows that
            \begin{equation}
                (x_1^2-y_1^2)z_1^2=(x_2^2-y_2^2)z_2^2
            \end{equation}

            \par\underline{case 1}: $x_1^2-y_1^2 \neq 0$.

            \par Combining (1) and (5) we get $z_1^2=z_2^2$.  Thus $z_1=\pm z_2$.  If $z_1=z_2$, then

            {\it case a:} $z_1\not=0$.  Then (3) and (4) imply $x_1=x_2$ and $y_1=y_2$.  If $z_1=-z_2$ then (3) and (4) imply $x_1=-x_2$ and $y_1=-y_2$.  Thus either $(x_1,y_1,z_1)=(x_2,y_2,z_2)$ or $(x_1,y_1,z_1)=(-x_1,-y_1,-z_1)$.\\

            {\it case b:} $z_1=0$.  Then $x_1^2+y_1^2=1$.  So one of $x_1$ or $y_1$ must be different from zero.  Assume wlog that $x_1\not=0$.   
            Now $x_1^2+y_1^2=1=x_2^2+y_2^2$ together with (1) implies $2x_1^2=2x_2^2$.  Thus $x_1=\pm x_2$.  If $x_1=x_2$ then (2) implies $y_1=y_2$, thus $(x_1,y_1,z_1)=(x_2,y_2,z_2)$.  If $x_1=-x_2$ then (2) implies $y_1=-y_2$, thus $(x_1,y_1,z_1)=(-x_2,-y_2,-z_2)$.  
            \\

            \par\underline{case 2}: $x_1^2-y_1^2=0$.\\

            \par Suppose $x_1=0$.  Then $y_1=0$, and combining (1) and (2) it follows that $x_2=0$ and $y_2=0$.  Now if $x_1=y_1=0$ then necessarily $z_1=\pm 1$.  Likewise $z_2=\pm1$.  Thus in this case either $(x_1,y_1,z_1)=(x_2,y_2,z_2)$ or $(x_1,y_1,z_1)=(-x_1,-y_1,-z_1)$.  By symmetry, the same thing happens if $y_1=0$, $x_2=0$ or $y_2=0$.\\

            \par Therefore we have reduced to the case that none of $x_1,x_2,y_1,y_2$ are zero.  By assumption $x_1=\pm y_1$ and from (1) it follows that $x_2=\pm y_2$. It then follows from (2) that $x_1=\pm x_2$.\\

            {\it case a:}  $x_1=x_2$.  Then (2) implies $y_1=y_2$ and (4) implies $z_1=z_2$.  Thus $(x_1,y_1,z_1)=(x_2,y_2,z_2)$.\\

            {\it case b:}  $x_1=-x_2$.   Then (2) implies $y_1=-y_2$ and (4) implies $z_1=-z_2$.  Thus $(x_1,y_1,z_1)=(-x_2,-y_2,-z_2)$.
        \end{proof}

    \item We show that the function $f:[0,2\pi]\times[0,\pi]\rightarrow\mathbb E^5$ defined by $f(x,y)=(\cos x, \cos 2y, \sin 2y, \sin x \cos y, \sin x \sin y)$ induces an embedding of the Klein bottle in $\mathbb E^5$: 
        \begin{proof} Let $Y=[0,2\pi]\times[0,\pi]$.
            Let $X=\text{im}(f)\subseteq\mathbb E^5$.  Since $\mathbb E^5$ is Hausdorff and a subspace of a Hausdorff space is Hausdorff, $X$ is Hausdorff.  And $[0,2\pi]\times[0,\pi]$ is compact.  Thus by Corollary 4.4 $f:Y\rightarrow X$ is an identification map. We know the Klein bottle can be obtained from $Y$ by identifying two opposite edges in the same orientation and the other two in the opposite orientation (see Figure 1.12, page 10).

            \par First note that $f$ identifies all four corners of $Y$ together.  So in what follows we will examine what happens to all of the other points.

            \par Suppose $f(x_1,y_1)=f(x_2,y_2)$.  First suppose $y_1\not=y_2$ and assume $y_1,y_2\in(0,\pi)$. Inspection of the graph of $\sin(x)$ tells us that if $\sin 2y_1 = \sin 2y_2$ then one of two  cases hold: (i) $0<2y_1,2y_2<\pi$ and $2y_1=\pi-2y_2$ or (ii) $\pi<2y_1,2y_2<2\pi$ and $2y_1=3\pi-2y_2$.  Now $\cos(2y_1)=\cos(2y_2)$ implies $2y_1=2\pi-2y_2$, or equivalently $y_1=\pi-y_2$.  But this is incompatible with both (i) and (ii).  Thus if we continue to assume $y_1\not=y_2$, it must be that at least one of $y_1$ or $y_2$ equals zero or $\pi$.  Since $y_1=\pi-y_2$, if $y_1=0$ then $y_2=\pi$ and conversely.  Thus we have shown that if  $f(x_1,y_1)=f(x_2,y_2)$ and $y_1\not=y_2$ then one of $(x_1,y_1)$ or $(x_2,y_2)$ must be on the top side and the other on the bottom side.
            In other words $\{y_1,y_2\}$ equals $\{0,\pi\}$.  We need to examine what happens to the $x$-coordinate of such points.

            \par So suppose $f(a,0)=f(b,\pi)$.  And suppose $0<a,b<2\pi$.  We have $\cos a=\cos b$ and $\sin a = -\sin b$ (from the first and fourth coordinates of $f$). From $\cos a=\cos b$ we know that either $a=b$ or $a=2\pi-b$.  From $\sin a=-\sin b$ we know that the only one of these that is possible is $a=2\pi-b$ (keep in mind we are assuming here that $a$ and $b$ are strictly between 0 and $2\pi$).
            \par Thus the points where $f(x_1,y_1)=f(x_2,y_2)$ with $y_1\not=y_2$ and $x_1,x_2\in(0,2\pi)$ are exactly the points of the form $((x_1,y_1),(x_2,y_2))=((x,0),(2\pi-x,\pi))$ for some $x\in(0,2\pi)$.  The only other points on the top or bottom sides are the corner points which we have already handled.

            \par Suppose now that $y_1=y_2$, and assume $x_1\not=x_2$.  Assume $y_1,y_2\in(0,\pi)$.  We know from the first coordinate of $f(x_1,y_1)=f(x_2,y_2)$ that $\cos x_1=\cos x_2$.  Since $x_1\not=x_2$, this holds only if $x_1=2\pi-x_2$.  
            Now from the fifth coordinate of $f(x_1,y_1)=f(x_2,y_2)$ we also know $\sin x_1=\sin x_2$ (substitute $y_2=y_1$ and then we can cancel $\sin y_1$ from both sides since we are assuming $y_1\in(0,\pi)$), which together with $x_1=2\pi-x_2$ implies $-\sin x_2=\sin x_2$, which implies $\sin x_2=0$.  Thus $x_2=0$ or $x_2=\pi$.
            We do not need to further evaluate the $y$-coordinates of such points since in this case we have assumed $y_1=y_2$.  Thus $f$ identifies points on the opposite vertical sides at the same vertical height to each other. The only other points on the left or right sides are the corner points which we have already handled.

            \par In summary we have shown that any points that $f$ identifies to each other must lie on the boundary of the square; and that $f$ identifies the four corners of the square to one point; and $f$ identifies points on the top and bottom sides in pairs, where the point with $x$-coordinate $x$ on the top is identified with the point with $x$-coordinate $2\pi-x$ on the bottom; and finally $f$ identifies points on the left and right sides in pairs, where points with the same $y$-coordinate are identified to each other.

            \par It follows that the identification space is the Klein bottle.
        \end{proof}

    \item With the notation of problem $11$, show that if $(2+\cos x)\cos 2y=(2+\cos x')\cos 2y'$ and $(2+\cos x)\sin 2y=(2+\cos x')\sin 2y'$, then $\cos x=\cos x'$, $\cos 2y=\cos 2y'$, and $\sin 2y=\sin 2y'$.  Deduce that the function $g:[0,2\pi]\times[0,\pi]\rightarrow \mathbb E^4$ given by $g(x,y)=((2+\cos x)\cos 2y, (2+\cos x)\sin 2y, \sin x\cos y, \sin x\sin y)$ induces an embedding of the Klein bottle in $\mathbb E^4$: 
        \begin{proof} Suppose that\setcounter{equation}{0}
            \begin{equation}
                (2+\cos x)\cos 2y=(2+\cos x')\cos 2y'
            \end{equation}
            \begin{equation}
                (2+\cos x)\sin 2y=(2+\cos x')\sin 2y'
            \end{equation}
            We want to show
            \begin{equation}
                \cos x=\cos x'
            \end{equation}
            \begin{equation}
                \cos 2y=\cos 2y'
            \end{equation}
            \begin{equation}
                \sin 2y=\sin 2y'
            \end{equation}
            {\it case 1:} $\{y,y'\}\cap\{\pi/4,3\pi/4\}=\emptyset$.  Then $\cos 2y\not=0$ and $\cos 2y'\not=0$.  Thus we can divide (2) by (1) to get $\tan 2y=\tan 2y'$.  Since $y,y'\in[0,\pi]$, the only way this is possible is if $y=y'$ or (wlog) $2y=2y'+\pi$.  If $y=y'+\pi/2$ then $\cos(2y)=-\cos(2y')$.  But then (1) canont hold (remember $\{y,y'\}\cap\{\pi/4,3\pi/4\}=\emptyset$). Thus $y=y'$, from which (4) and (5) obviously hold and it follows immediately from (1) or (2) that (3) holds.

            \par {\it case 2:} $\{y,y'\}\cap\{\pi/4,3\pi/4\}\not=\emptyset$.  If $y=\pi/4$, then (1) implies $y'=\pi/4$ or $y'=3\pi/4$.  But if $y'=3\pi/4$ then (2) does not hold.  So it must be $y=y'=\pi/4$.  Then (4) and (5) hold and (2) implies (3).  If on the other hand $y=3\pi/4$, then as before (1) implies $y'=\pi/4$ or $y'=3\pi/4$.  But if $y'=\pi/4$ then (2) does not hold.  So it must be that $y=y'=3\pi/4$.  Then (4) and (5) hold and (2) implies (3).

            \par  Now let $f$ be the function defined in Problem 11.  Then $g$ is an identification map for the same reasons $f$ is.  And it follows from what we just proved above that $g(x_1,y_1)=g(x_2,y_2)$ $\Leftrightarrow$ $f(x_1,y_1)=f(x_2,y_2)$.  Thus the identification space for $g$ is the same as the identification space for $f$.  Thus the image of $g$ is homeomorphic to the Klein bottle.
        \end{proof}
\end{enumerate}

\section{Topological Groups}
Throughout this section, and the rest of the book, the author assumes that such groups are $T_2$. Unless otherwise stated, this will be the case throughout the solutions.  
\begin{enumerate}[(1)]

    \item This is left to the reader as an exercise.

    \item Suppose that $(G, m, \tau)$ is a topological group. We show that 
        \begin{itemize}
            \item if $H$ is a subgroup of $G$, then its closure $\overline{H}$ is also a subgroup: 
                \begin{proof} Let $a,b \in \overline{H}$. We claim that $a + b \in \overline{H}$. Now, as $H \subset \overline{H}$, $m^{-1}(H) \subset m^{-1}(\overline{H})$. It follows that as 
                    \[m^{-1}(H) = \{(a,b) : a,b \in H\} = H \times H\]
                    we have $H \times H \subset \overline{H} \times \overline{H} 
                    \subset m^{-1}(\overline{H})$. But, as $\overline{H}$ is closed, and $m$ is continuous, we have 
                    \[m(\overline{H} \times \overline{H}) \subset \overline{H}\]
                    \par Similarly we can show this for the inverse function.
                \end{proof}

            \item if $H$ is normal, then so is $\overline{H}$: 
                \begin{proof} We use the characterization for normality give by left and right co-set equality. As previously show, $\overline{H} \leq G$. As $gH = Hg$, we have that 
                    \begin{itemize}
                        \item $gH = Hg \subset g\overline{H}$, and so $\overline{H}g \subset g \overline{H}$.
                        \item Similarly, $g\overline{H} \subset \overline{H}g$.
                    \end{itemize}
                \end{proof}
        \end{itemize}

    \item Let $G$ be a compact Hausdorff space which has the structure of a group. We show that $G$ is a topological group if the multiplication $m: G \times G \rightarrow G$ is continuous:  
        \begin{proof}Let $\pi$ be the canonical homeomorphism between $G$ and $\{e\} \times G$, and $p_2$ projection onto the second coordinate. From the fact that compositions of homeomorphisms are homeomorphisms, and the comments on page seventy-five, it follows that $L_{g^{-2}}$ is a homeomorphism and 
            \[p_2 \circ (\pi \circ L_{g^{-2}}) \equiv i: G \rightarrow G^{-1}\] is continuous. 
        \end{proof}

    \item We prove that $O(n)$ is homeomorphic to $SO(n)$ and that they are isomorphic as topological groups: 
        \begin{proof} Consider the determinate function restricted to $O(n)$. It follows that the function $g: O(n) \rightarrow SO(n) \times \Z_2$ given by 
            \[g(X) = (\det(X) X , \det X)\]
            is 1-1 and onto. We show that $g$ is continuous. Let $\Ocal$ be open in $SO(n) \times \Z_2$. Then, $\Ocal = U \times \{1\}$, or $\Ocal = U \times \{0\}$ for some open $U$ in $SO(n)$. But, as $SO(n)$ is a subgroup of $O(n)$, $U$ is open in $O(n)$. Thus, as $U, -U$ are open by extra lemma, it follows that 
            \[g^{-1}(U \times \{1\}) = -U, \quad g^{-1}(U \times \{0\}) = U\]
            are open. Thus, $g$ is continuous. As $O(n)$ is compat, and $SO(n) \times \Z_2$ is Hausdorff, it follows from theorem ($3.7$), page forty-eight, that $g$ is a homeomorphism. 
            \par To conclude the proof, we note that $\det XY = \det X \det Y$.  
        \end{proof}

    \item Let $A,B$ be compact subsets of a topological group. We show that the product set $AB$ is compact: 
        \red { \begin{proof} 
            Given an arbitrary open cover $\mathcal{O}$ of $AB$ we will construct a finite sub-cover.
            From $\mathcal{O}$ select finite open subcovers $\mathcal{B}_a, a\in A$, each $\mathcal{B}_a$ covering $aB$. 
            Notice each $\mathcal{B}_a$ covers $c_aB$ for $c_a$ some open neighborhood of $a$: 
            indeed for each $s\in \mathcal{B}_a$ with an element of the form $ab$ has $sb^{-1}$ a neighborhood of $a$. 
            The intersection of these $sb^{-1}$ is the $c_a$ above, open since the intersection is finite.

            $\{c_a\}_{a\in A}$ is an open cover of $A$ so it contains some finite subcover with indices $\iota \subset A$. 
            Thus $\cup_{a\in \iota} \mathcal{B}_a$ is a finite subcover of $AB$: by construction it is finite and by construction any $ab$ has some $i\in \iota$ for which $a\in c_i,$ and $s\in \mathcal{B}_i$ where $c_ib\subset s$ whence $ab\in s \in \mathcal{O}$.
        \end{proof} }

    \item We show that if $U$ is a neighbourhood of $e$ in a topological group, there is a neighbourhood $V$ of $e$ fow which $VV^{-1} \subset U$:  
        \red{ \begin{proof} 
            Given $U$ we will construct such a $V$.
            Consider $W = m^{-1}(U) \subset G\times G$, open since $m$ is continuous.
            $U$ contains $e$ so $W$ is a neighborhood of $m^{-1}(e)$ and in particular $(e,e)$. 
            So by construction of the product topology $W$ must contain some neighborhood $(e,e)$ of the form $W_1\times W_2$. 
            Take $V=(W_1\cap W_2)\cap(W_1\cap W_2)^{-1}$ and notice it is a neighborhood of $e$: 
                finite $\cap$ preserves open-ness, and $(\cdot)^{-1}$ being an open map makes $(W_1\cap W_2)^{-1}$ open. 
            Also notice from $V$'s definition that $V=V^{-1}$. 
            Finally $VV^{-1} = m(V,V) \subset m(W_1,W_2) \subset U$.
        \end{proof} }

    \item Let $H$ be a discrete subgroup of a topological group $(G, m, \tau)$. We find a neighbourhood $N$ of $e$ in $G$ such that the translates $hN = L_h(N)$, $h \in H$ are all disjoint: 
        \begin{proof} As $H$ is disrete, for some open cover $\{\Ocal_\alpha\}_\alpha$, each $\Ocal_\alpha$ contains one element of $H$. In particular, there exits some $\Ocal_{\alpha_0}$, such that 
            \[\{e\} \subset \Ocal_{\alpha_0}\]
            Thus, $\Ocal_{\alpha_0} \cap H \neq \{e\}$. By an exercise above, there exists some open $N$ in $G$, for which $NN^{-1} \subset \Ocal_{\alpha_0}$. Now, suppose to the contrary that for some $h,g \in H$, $h \neq g$, $hN \cap gN = \emptyset$. 
            \par Then, there exits some $n_1, n_2 \in N$, for which $hn_1 = g n_2$, implying that 
            \[g^{-1}h = n_2n_1^{-1} \in NN^{-1} \subset \Ocal_{\alpha_0} \]
            But then, $g^{-1}h = n_2n_1^{-1} \in \Ocal_{\alpha_0}$. This implies that 
            \[g^{-1}h \in \Ocal_{\alpha_0} \cap H = \{e\};\]
            a contradiction.
        \end{proof}

    \item We show that if $C$ is a compact subset of a topological group $(G, m, \tau)$, and if $H$ is a discrete subgroup of $G$ that $H \cap C$ is finite: 
        \begin{proof} By extra lemma, $\overline{H} = H$, and as $C$ is compact in a $T_2$ space, $C$ is closed. This implies that $H \cap C \subset C$ is closed, and compact by theorem ($3.5$), page forty-seven. 
            \par Suppose, to the contrary, that $H \cap C$ was not finite. As 
            \[\bigcup_{x \in C} \{x\}\]
            is an open cover of $H \cap C$, no finite set of $\{x : x \in C\}$ could cover $H \cap C$, contradicting the compactness of $H \cap C$. 
        \end{proof}

    \item We prove that every nontrivial discrete subgroup of $\R$ is infinite cyclic: 
        \begin{proof} Let $G$ be a non-trivial discrete subgroup of $\R$. Let $g \in G$, $g \neq 0$. Then, as $(G, m, \tau)$ is discrete, $\{0,g\}$ is a neighbourhood of $0$, and so, $g$ generates $G$ (theorem ($4.11$), page seventy-five). The proof that $G = \langle g \rangle$ is clear from the division algorithm. To conclude, we show that $|G| = \infty$: 
            \par Suppose, to the contrary, that $|G| = p$, for some $p \in \N$. Then, as $\overline{G} = G$, $\inf G \in G$. But then, by the Archimedian Principle, there exists some $n \in \N$, for which $ng > \inf G$, $hg \in \langle g \rangle$. Thus, $|G| = \infty$. 
        \end{proof}

    \item We prove that every non-trivial discrete subgroup of the circle is finite and cyclic: 
        \begin{proof} Let $G$ be a discrete subgroup of $S^1$. Define $f: [0,1] \rightarrow S^1$ by $f(t) = e^{2\pi i t}$. Then, $G$ is finite, since it is discrete, and compact. It is easy to see that $f^{-1}(G)$ is a discrete subgroup of $\R$, so is cyclic. Thus, $G = f(f^{-1}(G))$ is cyclic. 
        \end{proof}

    \item Suppose that $A,B \in O(2)$, such that $\det A  = 1$, $\det B = -1$. We show that $B^2 = I$, and $BAB^{-1} = A^{-1}$, and deduce that every discrete subgroup of $O(2)$ is either cyclic or dihedral: 
        \begin{proof} By exhaustion, 
            \[ B = 
            \begin{bmatrix}
                \pm1 & 0 \\
                0 & \mp1
            \end{bmatrix} 
            \]
            Thus, either way, $B^2 = I$. The fact that $BAB^{-1} = A^{-1}$ is routine. To complete the proof, we show that every discrete subgroup of $O(2)$ is either cyclic or dihedral:  
            \par From the comments on page seventy-seven, and exercise sixteen, we have that 
            \[O(2) \cong SO(2) \times \Z_2 \cong S^1 \times \Z_2\]
            Now, let $G$ be a discrete subgroup of $S^1 \times \Z_2$. Then, for every $X \in O(2)$, $\det X \neq \pm 1$, we have 
            \[f(X) \cong SO(2) \times \{1\} \cong S^1 \times \{1\} \cong S^1\]
            as in the previous exercise. By the exercise a above, we have that $f(X)$ must be finite cyclic. In other-words, $f^{-1}(f(X)) = X$ is finite cyclic. 
            \par To finish the proof, suppose that $X \in G$ was such that $\det X = -1$, without loss of generality. Now, let $K = G \cap SO(2)$. Then, $K$ is cyclic and $K$ is the set of $Z \in G$, such that $\det Z = 1$. Let $M$ be the generator of $K$. Likewise, let $N$ be the generator of $G-K$. Then, as $N^2 = I$, $NMN^{-1} = N^{-1}$, $\langle M, N \rangle$ is dihedral. 
            \par We claim that $G = \langle M, N \rangle$; Let $L \in G - \langle M, N \rangle$. Then, $L \notin K$, and so $\det L = -1$. Consequently, $\det LMN = 1$, and so, $LMN \in K$. But then, $L \in G$; a contradiction. This completes the proof.
        \end{proof}

    \item We show that  
        \begin{itemize}
            \item if $T$ is an automorphism of the topological group $\R$, that $T(r) = rT(1)$, for any rational $r$:
                \begin{proof} If $n \in \N$, then we must have $T(n + x) = T(n) + T(x)$. Further, as $T$ sends generators to generators, we must have 
                    \[T(n) = n, \quad T(n + x) = n + T(x)\]
                    It follows from the equality 
                    \[T(1) = \sum_{i = 1}^n T(1/n) = n + T(1/n)\]
                    that 
                    \[T(m/n) = m T(1/n) = \frac{m}{n} T(1)\]
                    Note the change in notation. 
                    Thus, as $\Q$ is dense in $\R$, it follows from the sequential criterion for continuity, that $xT(1) = T(x)$ for all $x \in \R$.
                \end{proof}
            \item the automorphism group of $\R$ is isomorphic to $\R \times \Z_2$:
                \begin{proof} We first show that $\Aut \R \cong \R \times \Z_2$; The map $T_x: \R \rightarrow \R$, given by $T_x(y) = yx$, $x \neq 0$ is an automorphism of $\R$, such that $T_x(1) = x$. Thus, consider $f: \R - \{0\} \rightarrow \Aut \R$ given by $f(x) = x = T_x(1)$, for each $T_x \in \Aut \R$, as described above. Then, clearly, $f$ is 1-1 and onto. Further, 
                    \[f(xy) = T_x(1) T_y(1) = xy = f(x)f(y)\]
                    \par To conclude, we show that $\R - \{0\} \cong \R \times \Z_2$; Consider the map 
                    \[g: \R \times \Z_2 \rightarrow \R, \quad 
                    g(x,y) = 
                    \begin{cases}
                        e^x & y = 0 \\
                        -e^x & y =1
                    \end{cases}
                    \]
                    We claim that $g$ is a homeomorphic isomorphism. As $e^x: \R \rightarrow \R^+$ is a homeomorphism from $\R$ to $\R^+$, it follows that $g$ is a homeomorphism. To show that it is a isomorhism, we show the homomorphism property: 
                    Without loss of generality, suppose that $(x,y), (x',y') \in \R \times \Z_2$ is such that $y = 0$, $y' = 1$; then, 
                    \[g(x,y)g(x',y') = e^x(-e^{x'}) = -e^{x + x'} = g(x + x', y + y')\]
                    This completes the proof. 
                \end{proof}
        \end{itemize}

    \item We show that the automorphism group of the circle group is isomorphic to $\Z_2$: 
        \begin{proof}   We first note that the identity map $i: S^1 \rightarrow S^1$ is the trivial automorphism. In addition, we have show that the conjugate map given by $\overline{i}(x) = \overline{x}$ is an automorphism. We show that $|\Aut S^1| = 2$: \par We show that if $f$ is any other automorphism, the $f \equiv i$, or $f \equiv \overline{i}$. Indeed; first note that, as 
            \[S^1 \cong U(1) = \bigcup_n U_n, \quad U_n = \{c \in \C: c^n = 1\},\]
            if $f \in \Aut S^1$, then $f$ must send generators to generators. So, consider the generators of $U(1)$, given by $U_2$, and $U_3$. Then, 
            \[U_2 \cong \Z_2, \quad U_3 \cong \Z_3\]
            Further, for $f \in \Aut S^1$, we must have $f(-1) = -1$, and thusly, 
            \[f( \frac{-1 + i \sqrt{3}}{2}) = \frac{-1 \pm i \sqrt{3}}{2}\]
            In which case, $f \cong i$ and $f \cong \overline{i}$, respectively. Thus, by extra lemma, we have that $|\Aut S^1| = 2$ implies $\Aut U(1) \cong \Z_2$.
        \end{proof}

\end{enumerate}

\section{Orbit Spaces}
\begin{enumerate}[(1)]
    \item We give an action of $\Z$ on $\E \times [0,1]$ which has the M\"obius Strip as an orbit space: 
        \begin{proof} Define $\pi : \Z \times (\E \times [0,1]) \rightarrow \E^2$, by
            \[ \pi (z, (x,y)) = 
            \begin{cases}
                (x+z,y) & z \in 2\Z \\
                (x+z, 1-y) & z \in 2\Z +1
            \end{cases}
            \]
            We show that $\pi$ is a topological group action:  
            \begin{itemize}
                \item Clear by construction.
                \item As, $\pi(0,(x,y)) = (x + 0, y) = (x,y)$, the result follows. 
                \item As the components are continuous, $\pi$ is continuous. 
            \end{itemize}
        \end{proof}

    \item  We find an action of $\Z_2$ on the torus with oribit space the cylinder: 
        \begin{proof} Consider the cylinder on page eighty. Define $\pi : \Z_2 \times T \rightarrow T$, by 
            \[\pi(g, (x,y,z)) = 
            \begin{cases}
                (x, -y , z) & g = 1 \\
                (x, y, z) & g = 0 
            \end{cases}
            \]
            We check that $\pi$ is a $T$ group action: 
            \begin{itemize}
                \item By definition. 
                \item As $ 0 \equiv e \in \Z_2$, clearly $\pi (0 ,(x,y,z))  = (x, y, z)$. 
                \item As the components are continuous, $\pi$ is continuous.
            \end{itemize}
            To show that $\pi(\Z_2 \times T) \cong C$, we note that $\{(x,y,z) : y \in \R\}$ is the canonical cylinder.
        \end{proof}

    \item We describe the orbits  of the natural action of $SO(n)$ on $\E^n$ as a group of linear transformation and identify the orbit space: 
        \begin{proof} Let $r\in \R$, $r\geq0$, let 
            \[S_r=\{p\in \E^n \mid \|p\|=r\}\]
            Then since $\text{SO}(n)$ preserves distances, $SO(n)$ must take $S_r$ to itself.  Furthermore, the action on $S_r$ is transitive, because it is transitive on $S^{n-1}\subset \E^n$ and $S_r = r \cdot S^{n-1}$.  To see the action on $S^{n-1}$ is transitive, for any vector $\mathbf{v} \in S^{n-1}$, it can be put into an orthonormal basis $\mathbf{B}$.  Then there is a change of coordinates matrix $M$ from the standard basis $\mathbf{e_1}, \dots, \mathbf{e_n}$ to $\mathbf{B}$ that takes $\mathbf{e_1}$ to $v$.  Since both bases are orthonormal, $M \in O(n)$.  Clearly $M$ can be chosen to be in $SO(n)$ such that $M(\mathbf{e_1})=v$ (if it's not already in $SO(n)$, just multiply one of the other $\mathbf{e_i}$'s by $-1$). Since any element of $S^{n-1}$ can be taken to $\mathbf{e_1}$, the action must be transitive.  Now if $\mathbf{v} \in \E^n$ is arbitrary ($v \neq 0$) then just scale $v$ to be in $S^{n-1}$, transform within $S^{n-1}$ and scale back.  The two scaling operations assure that the resulting transformation has determinant equal to one, and therefore is in $SO(n)$.

            \par Now let $r,r'\in \R$, $r,r'>0$, $r \neq r'$.  Since things in $SO(n)$ are length preserving, an element of $SO(n)$ cannot take an element of $S_r$ to an element of $S_{r'}$.  Thus each $S_r$ is exactly one orbit (true also if $r=0$ since $S_0$ consists of one point).

            \par Let $f:\E^n \rightarrow [0,\infty)$ be given by $f(\mathbf{v})=\|\mathbf{v}\|$.  Then $f$ is a continuous function that identifies each orbit of the action to a single point. Let $B$ be an open ball in $\E^n$.  Then clearly $f(B)$ is an interval, open in $[0,\infty)$.  Since functions respect unions, it follows that $f$ is an open map.  By Corollary 4.4 $f$ is an identification map.  Thus the identification space is homeomorphic to the image of $f$, which is $[0,\infty)$.  Thus the orbit space of $SO(n)$ on $\E^n$ is homeomorphic to $[0,\infty)$.
        \end{proof}

    \item Suppose that $\pi : X \rightarrow X/G$ is the natural identification map. We show that 
        \begin{itemize}
            \item if $\Ocal$ is open in $X$, then $\pi^{-1}(\pi(\Ocal))$ is the union of the sets $g(\Ocal)$, $g \in G$:
                \begin{proof} Let $a\in \Ocal$ and $g\in G$.  Since $ga$ is in the same orbit as $a$, $\pi(ga)=\pi(a)$.  Thus $\pi( g(\Ocal))=\pi(\Ocal)$, for all $g\in G$,  \[g(\Ocal)\subset \pi^{-1}(\pi(\Ocal)) \quad \forall g \in G\] 
                    and, 
                    \[\bigcup_{g\in G}g(\Ocal) \subset \pi^{-1}(\pi(\Ocal))\]
                    Now suppose $x\in\pi^{-1}(\pi(\Ocal))$; Then $\pi(x)\in\pi(\Ocal)$.   Thus $x$ is in the same orbit as some element of $\Ocal$, i.e. $x=ga$ for some $g\in G$ and $a\in \Ocal$. Consequently, $x\in g(\Ocal)$ implies  \[\pi^{-1}(\pi(\Ocal))\subset\bigcup_{g\in G}g(\Ocal)\]   
                    Since we have containment in both directions, we can conclude that 
                    \[\pi^{-1}(\pi(\Ocal))=\bigcup_{g\in G} g(\Ocal)\]
                \end{proof}

            \item $\pi$ is an open map: 
                \begin{proof} Now suppose $\Ocal$ is open in $X$. Recall a set $U \subset X/G$ is open in $X/G$ if, and only if, $\pi^{-1}(U)$ is open in $X$. Now, $\pi^{-1}(\pi(\Ocal))$ is a union of sets of the form $g(\Ocal)$ and (since each $g$ induces a homeomorphism of $X$) $g(\Ocal)$ is open in $X$, for all $g\in G$.  Thus $\pi^{-1}(\pi(\Ocal))$ is open. Consequently, $\pi(O)$ is open in $X/G$. 
                \end{proof}

            \item $\pi$ is not necessarily a closed map: 
                \begin{proof} We will show by counter-example that $\pi$ is not a closed map.  Let $\Z$ act on $\R$ by translation: $x \mapsto x+z$.  This orbit space is $S^1$.  For each $n= 0,1,2,\dots$, let \[A_n=[n+\frac{1}{n+3},n+\frac{1}{n+2}]\]
                    So, 
                    \begin{align*}
                        A_0 & = [1/3,1/2] \\
                        A_1 & =[1+\frac{1}{4},1+\frac{1}{3}] \\
                        A_2 & =[2+\frac{1}{5},2+\frac{1}{4}] \\
                        & \vdots 
                    \end{align*}
                    Let $A = \bigcup_n A_n$.  Then $A$ is closed in $\E$, but $\pi(A)=\pi((0,1/2])$ which is not a closed subset of $S^1$.
                \end{proof}
        \end{itemize}

    \item We show that
        \begin{itemize}
            \item if $X$ is Hausdorff, it is not necessarily the case that $X/G$ be Hausdorff:
                \begin{proof} Consider $X = S^1 \times S^1$. We claim that 
                    \[X/\R\]
                    given in example seven, page eighty-three is not $T_2$. Clearly, $S^1 \times S^1 = T$ is $T_2$. Suppose to the contrary, that for some $x,y \in X/\R$, there exists disjoint neighbourhoods $\Ocal_x$, $\Ocal_y$. But then, as the comments on page eighty-three point out, $\Orb x$ is dense in $T$. But then, 
                    \[\Ocal_x \cap \Ocal_y \neq \emptyset\]
                \end{proof}

            \item if $(X, m, \tau)$ is a topological group and $G$ is a closed subgroup acting on $X$ by left translation, that $X/G$ is Hausdorff:
                \begin{proof} Let 
                    \[C=\{(x,y)\in X\times X\mid x^{-1}y\in G\}\]
                    and $h:X\times X\rightarrow X$ be the map $h(x,y)=x^{-1}y$. Then $h^{-1}(G)=C$. 
                    \par Since $G$ is closed and $h$ is continuous, it follows that $C$ is closed.  Let $f:X\rightarrow X/G$ be the identification map.  Let $g:X\times X\rightarrow X/G\times X/G$ be the map $g(x,y)=(f(x),f(y))$.  By Problem 29 $f$ is an open map.  It follows that $g$ is an open map.  Thus by Theorem 4.3 $g$ is an identification map.   Let $\Delta$ be the diagonal in $X/G\times X/G$.  Then $g^{-1}(\Delta)=C$. Since $g$ is an identification map and $C$ is closed in $X\times X$, it follows that $\Delta$ is closed in $X/G\times X/G$.  By Chapter 3, Problem 25 (page 55) it follows that $X/G$ is Hausdorff.\\
                \end{proof}
        \end{itemize}

    \item We show that the stabilizer of any point is closed subgroup of $G$ when $X$ is Hausdorff, and that points in the same orbit have conjugate stabilizers for any $X$: 
        \begin{proof} If $g$ and $g'$ are in the stabilizer of $x$, then $gg'x=gx=x$ so $gg'$ is in the stabilizer of $x$.  And $g^{-1}gx=1\cdot x=x$, but also $g^{-1}gx=g^{-1}x$.  Thus $g^{-1}x=x$ so $g^{-1}$ is in the stabilizer of $x$.  It follows that the stabilizer of $x$ is a subgroup of $G$.
            Now, let $f:G\rightarrow X$ be given by $f(g)=gx$.  Then $f$ is continuous.  Since $X$ is Hausdorff, by Theorem 3.6 points are closed (finite sets are always compact).  Thus $f^{-1}(x)$ is closed in $X$.  But $f^{-1}(x)$ is exactly the stabilizer of $x$.  Thus the stabilizer of $x$ is closed in $X$.
            \par  It remains to show points in the same orbit have conjugate stabilizers.  Let $x,y$ be in the same orbit, so $x=gy$ for some $g\in G$ and, let $a\in \Stab x$.  Then
            \begin{align*}
                g^{-1}agy & =g^{-1}ax=g^{-1}x=y \\    
                & \implies g^{-1}ag\in \Stab y \\
                & \implies g^{-1} (\Stab x)g \subset \Stab y
            \end{align*}
            Now let $a \in \Stab y$.  Then, 
            \[ gag^{-1}x=gay=gy=x\] 
            So,
            \begin{align*}
                gag^{-1}\in \Stab x \implies \Stab y \subset g^{-1} (\Stab x)g
            \end{align*}
            Since we have set containment in both directions it follows that 
            \[g^{-1} (\Stab x)g= \Stab y\]
        \end{proof}

    \item Suppose that $G$ is compact, $X$ is Hausdorff and that $G$ acts transitively on $X$. We show that $X$ is homeomorphic to the orbit space $G/\text{(stabilizer of x)}$ for any $x \in X$: 
        \begin{proof} Let $x\in X$, $f:G\rightarrow X$ be given by $f(g)=gx$. Since $G$ acts transitively, $f$ is onto.  Since $G$ is compact and $X$ is Hausdorff, $f$ is an identification map by corollary ($4.4$).  So $G^*$, the identification space associated to $f$, is homeomorphic to $X$.  
            \par Now, suppose $f(g_1)=f(g_2)$.  Then $g_2^{-1}g_1 \in \Stab x$.  Consequently, $g_1$ is in the same coset as $g_2$ with respect to the subgroup $(\text{stabilizer of }x)$. Thus $G^*$ is exactly $G/(\text{stabilizer of }x)$.
        \end{proof}

    \item We prove that the resulting space is homeomorphic to the Lens space $L(p,q)$: 
        \begin{proof}
        \end{proof}

    \item We show that $L(2,1)$ is homeomorphic to $\RP^3$ and that if $p$ divides $q-q'$, that $L(p,q)$ is homeomorphic to $L(p,q')$: 
        \begin{proof} Using the definition of $L(2,1)$ on page eighty-two, $L(2,1)$ is an identification space of $S^3$, the orbit space under the action of $\mathbb Z_2$.  The action of the generator of $\mathbb Z_2$ is the homeomorphism $(z_0,z_1)\mapsto(e^{\pi i}z_0,e^{\pi i}z_1)=(-z_0,-z_1)$.  And $(-z_0,-z_1)$ is the antipodal point to $(z_0,z_1)$.  Thus, by example $2$ on page eighty, the identification space is exactly $\mathbb P^3$.
            \par Now suppose $p|(q-q')$.  Then $q=q'+np$ for some $n\in\mathbb N$.  Thus,
            \[e^{2\pi qi/p}=e^{2\pi (q'+np)i/p}=e^{2\pi q'i/p+2\pi npi/p}=e^{2\pi q'i/p}e^{2\pi ni}=e^{2\pi q'i/p}\]
            And, so, the action of $\mathbb Z_p$ is identical in both cases; the resulting orbit spaces are identical (and therefore homeomorphic).
        \end{proof}

\end{enumerate}


\newpage 
\chapter{The Fundamental Group}
\section{Homotopic Maps}
Throughout this section $i$ will typically denote the identity map, unless otherwise stated. 
\begin{enumerate}[(1)]
    \item Let $C$ denote the unit circle in the plane. Suppose that $f: C \rightarrow C$ is a map which is not homotopic to the identity. We show that $f(x) = -x$ for some $x \in C$: 
        \begin{proof} Suppose, to the contrary, that for any $x \in C$, $f(x) \neq -x$. Then, we claim that $f \simeq i$. Indeed; as the example on page eighty-nine points out, 
            \[F(x,t) = \frac{(1-t)f(x) + tg(x)}{||(1-t)f(x) + tg(x)||}\]
            is such that 
            \[f \simeq_F i \equiv g;\]
            a contradiction.
        \end{proof}

    \item With $C$ as above, we show that the map which takes each point of $C$ to its antipodal is homotopic to the identity: 
        \begin{proof} Consider the matrix 
            \[ A_\theta = 
            \begin{bmatrix}
                \cos(\theta) & -\sin(\theta) \\
                \sin(\theta) & \cos(\theta) 
            \end{bmatrix}
            \] 
            Then, $A_\pi \equiv f(x) $. Thus, define $F: C \times I \rightarrow C$ by 
            \[ F (x,t) =
            \begin{bmatrix}
                \cos( \pi(1-t)) & -\sin(\pi(1-t)) \\
                \sin(\pi(1-t)) & \cos(\pi(1-t)) 
            \end{bmatrix} x
            \]
            We show that $F$ is homotopy from $f$ to $i$: 
            \par Clearly, 
            \begin{align*}
                F(x,0) &= A_\pi x \equiv f(x) \\
                F(x,1) &= A_0 x = Ix \equiv i
            \end{align*}
            Further, as $F$ is the sum of continuous functions, it is continuous. Thus, 
            \[f \simeq_F i\]
        \end{proof}

    \item Let $D$ be the disc bounded by $C$, and parametrize $D$ using polar coordinates. Let $h: D \rightarrow D$ be the homeomorphism defined by $h(0) = 0$, $h(r, \theta) = (r , \theta + 2\pi r)$. We find a homotopy $F$, from $h$ to the identity, such that the functions
        \[F|_{D \times \{t\}} : D \times \{t\} \rightarrow D, \quad t \in [0,1] \]
        are all homeomorphisms: 
        \begin{proof} Let $i(r,\theta)=(r,\theta)$ the identity map.  Define $F:D\times I\to D$ by $F((r,\theta),t)=(r,\theta+2\pi r(1-t))$.  $F$ is given by polynomials in $r,\theta$ and $t$, so $F$ is continuous.  Further, 
            \begin{align*}
                F((r,\theta),0) & =h(r,\theta) \\
                F((r,\theta),1) & =(r,\theta) = i(r,\theta) 
            \end{align*}
            Thus,
            \[h \simeq_F i\]
            Since $F|_{D\times\{t\}}\to D$ is a one-to-one continuous map from a compact space to a Hausdorff space, theorem ($3.7$) implies that  $F|_{D\times\{t\}}\to D$ is a homeomorphism.
        \end{proof}

    \item With the above terminology, we show that $h$ is homotopic to the identity relative to $C$: 
        \begin{proof} As we are consider $h$ relative to $C$, $r = 1$ for all $x \in C$. Thus, 
            \[h = h' = (1, \theta + 2\pi)\]
            Define $F: \partial D \times I \rightarrow \partial D$ by 
            \[F(x, t)  = (1, \theta + 2\pi(1-t))\]
            Then, 
            \begin{align*}
                F(x,0) & = (1 , \theta + 2\pi) = h'(x) \\
                F(x,1) & = (1, \theta) = i(x), 
            \end{align*}
            where $i$ is the identity on $C$. Likewise, $F$ is continuous as it is polynomial in $\theta$, $t$. Consquently, 
            \[h' \equiv h|_{\partial D} \simeq_F i \]
        \end{proof}

    \item Let $f: X \rightarrow S^n$ be a map that is not onto. We show that $f$ is null-homotopic: 
        \begin{proof} Let $p$ be a point in $S^n$ such that its antipodal point $-p$ is not in the image of $f$.   Now let $g:X\to S^n$ be the constant function $g(x)=p$. Then $g(x)$ and $f(x)$ never give a pair of antipodal points for any $x\in X$.  By example two on page eighty-nine, $f$ and $g$ are homotopic.
        \end{proof}

    \item Let $CY$ denote the cone on $Y$. We show that any two maps $f,g: X \rightarrow CY$ are homotopic: 
        \begin{proof} By extra lemma \footnote{reference  this} $f$, $g$ are null-homotopic. 
        \end{proof}

    \item We show that a map from $X$ to $Y$  is null homotopic if, and only, if it extends to a map from the cone on $X$ to $Y$: 
        \begin{proof}
            \begin{itemize}
                \item[] ($\implies$): Suppose that $f: X \rightarrow Y$ is null-homotopic, via $F$. Thus, 
                    \[F: CX \rightarrow Y, \quad \text{as} \quad CX = \nicefrac{X}{X \times \{1\}}\]
                    Thus, we need to show that $F$ is an extension of $f$. 
                    \par Indeed, $X \times \{0\} \subset X \times I$, and 
                    \[\{f(x) : x \in X\} \cup \{0\} \subset Y\]
                    So, 
                    \[F|_{X \times \{0\}} (x,t) = F(x,0) = f(x), \quad \forall x \in X \times \{0\} \cong X.\]
                    As $f$ is continuous, it follows that $F|_{X \times \{0\}}$ is. 

                \item[] ($\impliedby$): Suppose that $f: X \times Y$ is a map and that $F:CX \rightarrow Y$ is an extension of $f$. Then, consider the map $H(x,t) = F(x,t)$. As 
                    \[F(x,1) \equiv X \times \{1\},\]
                    we have 
                    \[F(x,1) = H(x,1) = c, \quad \text{for some } c \in Y.\]
                    In addition, 
                    \[H(x,0) \equiv F|_{X \times \{0\}} \equiv f \]
                    and we have 
                    \[f \simeq_H i\]
            \end{itemize}


        \end{proof}

    \item Let $A$ denote the annulus $\{(r,\theta) : r \in [1,2], \theta \in [0,2\pi]\}$, and let $h$ be a homeomorphism of $A$ defined by $h(r, \theta) = (r, \theta + 2\pi(r-1))$. We show that $h$ is homotopic to the identity map:
        \begin{proof} Consider the function $F: A \times I \rightarrow A$ given by 
            \[F((r,\theta) , t) = (r, \theta + 2\pi t(r-1))\]
            Then, 
            \begin{align*}
                F((r,\theta), 0 ) & = (r, \theta) = i(r, \theta) \\
                F((r, \theta), 1) & = (r, \theta + 2 \pi (r  - 1)) 
            \end{align*}
            As $F$ is polynomial in $\theta, t, r$, it follows that $F$ is continuous. Thus, 
            \[h \simeq_F i\]
        \end{proof}

\end{enumerate}

\section{Construction of the Fundamental Group}
\begin{enumerate}[(1)]
    \item Let $\alpha, \beta, \gamma$ be loops in a space $X$, all based at $p$. We write out the formulae for $(\alpha.\beta).\gamma$, and $\alpha.(\beta.\gamma)$ and show that they are homotopic loops relative $\{0,1\}$: 
        \begin{proof} By definition, we have 
            \[((\alpha . \beta). \gamma)(t) = 
            \begin{cases} 
                \alpha(4t) & t \in [0, 1/4] \\
                \beta(4t -1) & t \in [1/4,1/2] \\
                \gamma(2t -1) & t \in [1/2,1]
            \end{cases} 
            \]                            
            and,
            \[ (\alpha.(\beta.\gamma))(t) = 
            \begin{cases}
                \alpha(2t) & t \in [0,1/2] \\
                \beta(4t-2) & t \in [1/2,3/4] \\
                \gamma(4t-3) & t \in [3/4,1] 
            \end{cases}
            \]
            Considering the canonical diagram,we have claim that $H: X \times I \rightarrow X$, defined by
            \[ H(t,s) = 
            \begin{cases}
                \alpha(\nicefrac{4t}{s+1}) & t \in [0, \frac{s+1}{4}] \\
                \beta(4t-1-s) & t \in [ \frac{s+1}{4}, \frac{s +2}{4}] \\
                \gamma(\nicefrac{4t-2-s}{2-s}) & t \in [\frac{s+2}{4}, 1] 
            \end{cases}
            \]
            is a homotopy. Indeed; 
            \[H(t,0) = 
            \begin{cases}
                \alpha(4t) & t \in [0, 1/4] \\
                \beta(4t -1) & t \in [1/4,1/2] \\
                \gamma(2t -1) & t \in [1/2,1]                        
            \end{cases}
            \]
            and, similarly, 
            \[ (H(t,1) = \alpha.(\beta.\gamma)) (t) \]
            By two applications of the Gluing Lemma, $H$ is continuous. So, 
            \[(\alpha.\beta). \gamma \simeq_H \alpha.(\beta.\gamma) \text{ rel} \{0,1\}\]
        \end{proof}

    \item Let $\gamma$, $\sigma$ be two paths in the space $X$ which start at $p$ and end at $q$. We show that $\sigma_*$ is the composition of $\gamma_*$ and the inner automorphism of $\pi_1(X, q)$ induced by the element $\langle \sigma^{-1} \gamma \rangle$: 
        \begin{proof} The isomorphism defined by $\sigma_*$ and $\gamma_*$ are 
            \[\sigma_*(\innerone{\alpha}) = \sigma^{-1}. \alpha. \sigma \]
            and, 
            \[\gamma_*( \innerone{\alpha}) = \gamma^{-1}.\alpha.\gamma\]
            where $\alpha$ is a loop based at $p$. The claim is that 
            \[\sigma_*(\innerone{\alpha}) = \gamma_*( \innerone{\gamma.\sigma^{-1}} \innerone{\alpha} \innerone{\sigma.\gamma^{-1}})\]
            Indeed; 
            \begin{align*}
                \gamma_*( \innerone{\gamma.\sigma^{-1}} \innerone{\alpha} \innerone{\sigma.\gamma^{-1}}) & = \gamma_*( \innerone{\gamma.\sigma^{-1}}) \gamma_*( \innerone{\alpha}) \gamma_*( \innerone{\sigma.\gamma^{-1}}) \\
                & = (\gamma^{-1}.\gamma.\sigma^{-1}.\gamma).(\gamma^{-1}.\sigma. \gamma).(\gamma^{-1}.\sigma.\gamma^{-1}.\gamma) \\
                & = \sigma^{-1}.\alpha.\sigma \\ 
                & = \sigma_*(\innerone{\alpha})
            \end{align*}
        \end{proof}

    \item Let $X$ be a path connected space. We describe when it is true that for any two points $p,q \in X$, all paths from $p$, $q$ induce the same isomorphism between $\pi_1(X, p)$ and $\pi_1(X, q)$: 
        \begin{proof}As the above proof shows, if $\gamma$, $\sigma$ are paths from $p$ to $q$, then
            \[\sigma_*(\innerone{\alpha}) = \sigma^{-1}.\alpha.\sigma\]
            Thus, $\sigma_* = \gamma_*$  if, and only, if 
            \[\gamma_*( \innerone{\gamma.\sigma^{-1}} \innerone{\alpha} \innerone{\sigma.\gamma^{-1}}) = \gamma_*(\innerone{\alpha}),\]
            i.e. $\innerone{\alpha} . \innerone{\beta} = \innerone{\beta} . \innerone{\alpha}$. Thus, $\pi_1(X)$ must be abelian. 
        \end{proof}

    \item We show that any indiscreet space has trivial fundamental group: 
        \begin{proof} By a previous problem, if $(X, \tau)$ is indiscreet, then $X$ is path connected as $\tau = \{X, \emptyset\}$. The idea here is that there are no open sets (holes) for a path to get caught on. We claim that $X$ is contractible: 
            \par Consider $\pi_1(X, x_0)$, for some fixed $x_0 \in X$. Define $i \in \pi_1(X, x_0)$ by 
            \[i(x) = x_0, \quad \forall x \in X\]
            Now, for an arbitrary $f \in \pi_1(X, x_0)$, define $H: X \times I \rightarrow X$ by 
            \[H(x,t) =
            \begin{cases}
                i(x) & t \in [0,\nicefrac{1}{2}] \\
                f(x) & t \in [\nicefrac{1}{2}, 1]
            \end{cases}
            \]
            Then, $H$ is the desired homotopy as 
            \begin{align*}
                H(x,0) & = i(x) \\
                H(x,1) & = f(x)
            \end{align*}
            and it is continuous as the only non-trivial open sets in $\tau $ is $X$ and $H^{-1}(X) = X \times I$, which is open in $X \times I$. Therefore, 
            \[f \simeq_H i \]
        \end{proof}

    \item Let $(G, m, \tau)$ be a path-connected topological group. Given two loops $\alpha$, $\beta$ based at $e$ in $G$, define a map $F: [0,1]^2 \rightarrow G$ by 
        \[F(s,t) = \alpha(s).\beta(t)\]
        We show the effect of this map on the square, and prove that the fundamental group of $G$ is abelian: 
        \begin{proof} We want to show that for any two loops $\alpha$, $\beta$ that
            \[\innerone{\alpha}\innerone{\beta} = \innerone{\beta}\innerone{\alpha}\]
            As 
            \[F(s,0) = F(s,1) = \alpha(s), \quad F(0,t) = F(1,t) = \beta(t),\]
            we have that 
            \[\alpha \simeq_F \alpha(s), \quad \beta \simeq_F \beta(t).\]
            We first show that $\alpha.\beta \simeq m(\alpha(t), \beta(t))$: 
            \par Define $P: [0,1]^2 \rightarrow G$ by
            \[P(s,t) = 
            \begin{cases}
                \alpha(\frac{2t}{1-s}) & t \in [0, \frac{1+s}{2}] \\
                e & t \in (\frac{1+s}{2}, 1]
            \end{cases}
            \]
            and, $Q: [0,1]^2 \rightarrow G$ by
            \[ Q(s,t) =
            \begin{cases}
                e & t \in [0, \frac{1-s}{2}] \\
                \beta(\frac{2t-1+2}{1+s}) & t \in (\frac{1-s}{2}, 1]
            \end{cases}
            \]
            Then, let $H(s,t) = m(P(s,t), Q(s,t))$. Note that $H$ is a homotopy between $\alpha.\beta$ and $m(\alpha(t), \beta(t))$. Similarly, we can show that 
            \[m(\alpha(t), \beta(t)) \simeq_{H'} \beta. \alpha, \]
            where $H' = Q P$. Therefore, 
            \[\alpha.\beta \text{ rel} \{0,1\} \simeq_H m(\alpha(t), \beta(t)) \simeq_{H'} \beta.\alpha \text{ rel} \{0,1\} \]
            I.e. $\pi_1(G)$ is abelian. 

        \end{proof}

    \item We show that the space $\E^n - B$, where $B = \{(x,y,z) : y = 0 \land 0 \leq z \leq 1\}$ has trivial fundamental group:
        \begin{proof}  As $B$ is homeomorphic to a point, it follows by extra lemma that $\E^n - B$ has trivial fundamental group. 
        \end{proof}

\end{enumerate}

\section{Calculations}
\begin{enumerate}[(1)]
    \item We use theorem ($5.13$) to show that the M\"obius strip and the cylinder both have fundamental group $\Z$: 
        \begin{proof} Letting $X = \E^1 \times I$, it is clear that as $\E$ and $[0,1]$ are convex, that $X$ is simply connected. Consider the action of $\Z$ on the identification square which yields the M\"obius loop. It follows that 
            \[\pi_1( \E \times [0,1] / \Z) = \pi_1(M) \cong \Z\]
            \par This is similar for the torus. 
        \end{proof}

    \item Consider $S^n \subset \E^{n+1}$. Given a loop $\alpha$ in $S^n$, we find a loop $\beta$ in $\E^{n+1}$ which is based at the same point as $\alpha$, and is made up of a finite number of straight line segments, and satisfy 
        \[|| \alpha(s) - \beta(s)|| < 1, \quad s \in [0,1]\]
        And, we use this to deduce that $S^n$ is simply connected when $n \geq 2$: 
        \begin{proof} Note that for each $n$, $S^n$ is compact. Likewise, as $[0,1]$ is compact, $\alpha([0,1])$ is compact. Now, either $\alpha$ contains antipodal points, or it doesn't. Let 
            \[\{ \Ocal_b\}_b\]
            be an open cover of $\alpha([0,1])$. Then, it has an finite subcover; say, 
            \[\{ \Ocal_{b_i}\}_{i=1}^N\]
            \par If $\alpha$ has no antipodal points, then we are done as the path $\beta$ which connects one point form each $\Ocal_{b_i}$ at the same value is the desired path. 
            \par Now, if $\alpha$ has antipodal points for some $s \in [0,1]$ wew can disjointize neighbourhoods so that $\alpha(s) \in V_i$, and $-\alpha(s) \in V_j$, such that $\alpha(s) \notin V_j$ and $-\alpha(s) \in V_i$, such that 
            \[\bigcup V_i = \bigcup \Ocal_{b_i}\]
            and the construction of $\beta$ follows, as above. 
            \par To conclude the proof, we show that $S^n$ is simply connected: Indeed; by construction, no line adjoining $\alpha(s)$ and $\beta(s)$ goes through the origin. Thus, the projective straight linke homotopy shows that $S^n$ is simply connected, as $\E^{n+1}$ is for $n \geq 2$. 
            \par Note, this breaks down in the case where the  stereographic projective homotopy fails. This completes the proof. 
        \end{proof}

    \item Considering the 'proof' of theorem ($5.13$), we show that for $g_1, g_2 \in G$, $\gamma_1.(g_1 \circ \gamma_2)$ joins $x_0$ to $g_1g_2(x_0)$, where $\gamma_1$, $\gamma_2$ are paths from $x_0$ to $g_1(x_0)$, $g_2(x_0)$, respectively. In addidtion, we use this to deduce that $\phi$ is a homomorphism: 
        \begin{proof} As the statement on page one-hundered-two points out, 
            \[\gamma_1.(g_1 \circ \gamma_2)\]
            is a path from $x_0$ to $\gamma_1\gamma_2(x_0)$.
            Thus, as in the proof of ($5.13$), 
            \begin{align*}
                \phi(g_1g_2) & = \innerone{\pi \circ (\gamma_1. g_1 \circ \gamma_2) } \\
                & = \innerone{\pi \circ \gamma_1. \pi \circ \gamma_2} \\
                & = \innerone{\pi \circ \gamma_1} \innerone{\pi \circ \gamma_2} \\
                & \phi(g_1) \phi(g_2) 
            \end{align*}
            As $\pi \circ \gamma_1. \pi \circ \gamma_2$ is a loop at $\pi(x_0)$ (it follows that they are in the same homotopy class). 
        \end{proof}

    \item Let $\pi: X \rightarrow Y$ be a covering map and $\alpha$ a path in $Y$. We show that $\alpha$ lifts to a (unique) path in $X$ which begins at any preassigned point of $\pi^{-1}(\alpha(0))$: 
        \begin{proof} As $X = \bigcup_{y \in Y} \{y\}$, we can form an open cover of $Y$ by evenly covered, i.e. canonical, neighbourhoods. Let $\Ocal$ be the aforementioned open cover, and set 
            \[\alpha^{-1}(\Ocal)\]
            to be the family of open sets which cover $[0,1]$. As $[0,1] \subset \R$, is compact, and 
            \[\bigcup \alpha^{-1}(\Ocal)\]
            is an open cover of $[0,1]$, $\bigcup \alpha^{-1}(\Ocal)$ has a Lebesgue number, $\delta$. 
            \par Now, choose $n \in \N$, so that $\nicefrac{1}{n} < \delta$. Then, consider the partition of $[0,1]$, given by 
            \[\{0,1/n, 2/n \dots , \nicefrac{n-1}{n}\}\]
            Then, for each $i = 1, 2, \dots, n$, 
            \[\alpha\Big(\Big[\frac{i-1}{n}, \frac{i}{n}\Big]\Big)\]
            lies inside a canonical neighbourhood of $Y$. Thus, by setting 
            \[t_i = \frac{i-1}{n},\]
            the result follows. 
        \end{proof}

    \item Let $\pi: X \rightarrow Y$ be a covering map, $p \in Y$, $q \in \pi^{-1}(p)$, and $F: I \times I \rightarrow Y$ a map such that 
        \[F(0,t) = F(1,t) = p, \quad t \in [0,1]\]
        We use the argument of lemma ($5.11$) to find a map $F': I \times I \rightarrow X$ which satisfies 
        \[\pi \circ F' = F, \quad F'(0,t) = q, \quad t \in [0,1]\]
        and is unique: 
        \begin{proof} As the argument of lemma ($5.11$) is really quite bad, we proceed with a more technical proof via problem twenty. 
        \end{proof}

    \item With the terminology above, we note that for each $t \in [0,1]$, we have a path $F_t(s) = F(s,t)$ in $Y$ which begins at $p$. Let $F'_t$ be its unique lift to a path $X$ which begins at $q$, and set $F'(s,t) = F'_t(s)$. We show that $F'$ is continuous and lifts $F$: 
        \begin{proof} By the homotopy lifting lemma, there exists a unique homotopy lift, $B$ of $F$. By the proof so mentioned, $B$ is continuous and satisfies 
            \[\pi \circ B \equiv F\]
            But, by the path lifting lemma, as $F'(s,t_0)$, $t_0 \in [0,1]$ is a lift of a path $F(s,t)$ with initial point $q$, 
            \[B = F'\]
        \end{proof}

    \item We describe the homomorphism $f_*: \pi_1(S^1,1) \rightarrow \pi_1(S^1, f(1))$ induced by each of the following maps: 
        \begin{itemize}
            \item $f(e^{i\theta}) = e^{i(\theta + \pi)}$, $\theta \in [0, 2\pi]$: 
                \begin{proof} Note, $f_*(\innerone{\alpha}) = \innerone{f(\alpha)} = \innerone{-\alpha}$. And, as $f(1) = -1$, $f_* \equiv id$.
                \end{proof}

            \item $f(e^{i\theta}) = e^{in\theta}$, $\theta \in [0,2\pi]$, $n \in \Z$: 
                \begin{proof} Note, $f(1) = 1$. Thus, for each loop, we have $f_*(\innerone{\alpha}) = \innerone{n\alpha}$. So, $f_*$ sends the loop component to the $n$-th degree of $\alpha$. 
                \end{proof}

            \item $f(e^{i \theta}) = 
                \begin{cases} 
                    e^{i \theta } & \theta \in [0, \pi] \\
                    e^{i(2\pi - \theta)} & \theta \in [\pi, 2\pi]
                \end{cases}$: 
                \begin{proof} Note, $f(1) = 1$. $f_*$ means that any loop part on the upper half of $S^1$ is identified to the lower part of $S^1$. Thus, it returns the straight line for each loop on $S^1$. 
                \end{proof}
        \end{itemize}

    \item For each of the three different action of $\Z_2$ on the torus, in section $4.4$, we describe the homomorphism from the fundamental group of the torus to that of the orbit space induced by the natural identification map: 
        \begin{proof}
        \end{proof}

    \item As in problem eight, page ninety-one, we show that it is impossible to find a homotopy from $h$ to the identify which is relative to the two boundary circles of $A$: 
        \begin{proof} If $h \simeq_F i$ rel $\{c_1, c_2\}$, then 
            \[\alpha^{-1} \beta \simeq_F c \text{ rel} \{0,1\}\]
            However, $\alpha^{-1} \beta$ is homotopic to the loop of constant radius which we know is non-trivial via a deformation retract. 
        \end{proof}

\end{enumerate}

\section{Homotopy Type}
\par Throughout this section, $\simeq$ will typically denote homotopy equivalence.
\begin{enumerate}[(1)]
    \item If $X \simeq Y$ and $X' \simeq Y'$, we show that $X \times X' \simeq Y \times Y'$: 
        \begin{proof} We are given maps $f: X \rightarrow Y$, $g: Y \rightarrow X$, $f':X' \rightarrow Y'$ and $g': Y' \rightarrow X'$, such that $g \circ f \simeq_{\gamma'} 1_X$, $f \circ g \simeq_{\gamma} 1_Y$, $g' \circ f' \simeq_{\delta '} 1_{X'}$ and $f' \circ g' \simeq_{\delta} 1_{Y'}$. 
            \par From the component-wise definition of the product topology, $F: X \times X' \rightarrow Y \times Y'$, defined by $F(x,x') = (f(x), f'(x'))$, and $G: Y \times Y' \rightarrow X \times X'$, defined by $G(y,y') = (g(y), g'(y'))$ are continuous. Further, $F \circ G \simeq 1_{Y \times Y'}$, and $G \circ F \simeq 1_{X \times X'}$, via the canonical maps, component-wise defined: $\mathcal{F}:(X \times X') \times I \rightarrow X \times X'$, $\mathcal{F} \equiv (\gamma', \delta')$, and $\mathcal{G}:(Y \times Y') \times I \rightarrow Y \times Y'$, $\mathcal{G} \equiv (\gamma, \delta)$, respectively. 
            \par Thus, $X \times X' \simeq Y \times Y'$.
        \end{proof}

    \item We show that the cone, $CX$, is contractible for any space $(X, \tau_X)$:\footnote{We proved another proof for this. The first of which is in the extra lemmas section.}
        \begin{proof} By definition, we have $CX = \nicefrac{X \times I}{X\times \{1\}}$. Define ${H}: CX \times I \rightarrow CX$ by $${H}((x,t),s) = (x, t(1-s))$$ Then, we have 
            \begin{align*}
                {H}((x,t), 0) & = (x,t) = 1_{CX} \\
                {H}((x,t),1) & = (x,0) 
            \end{align*}
            Now, if ${\Ocal}_{CX} \in \tau_{CX}$, then $H^{-1}({\Ocal}_{CX}) = ({\Ocal}_{CX}, I) \in  CX \times I$. Thus, $H$ is continuous. This proves the result. 
        \end{proof}

    \item We show that the punctured torus deformation retracts onto the one-point union of two circles. 
        \begin{proof} We consider the torus as the identification space of a square, $X$, bounded by the box whose edge points are $(-1,-1),(-1,1),(1,1),(1,-1)$. Assume, without loss of generality, that the point $(0,0)$ is removed. 
            \par We show that $F:T \times I \rightarrow T$ defined by 
            $$F(x,t) = (1-t)x + t\frac{x}{||x||}$$
            is a deformation retract onto the one point union of two circles: 
            \par As
            \begin{align*}
                F(x,0) & = x \\
                F(x,1) & = \frac{x}{||x||} 
            \end{align*}
            $F$ is a deformation retract onto $\partial X$. The function $f: \partial X \rightarrow S_1^1 \bigvee S_2^1$, given by 
            \[f(x) = 
            \begin{cases}
                g(x) & x \in (-1,\pm1)t + (1,\pm1)(1-t) \\
                s(x) & x \in (\pm1,1)t + (\pm,-1)(1-t)
            \end{cases}
            \]
            for all $t \in I$, where $g,s$ are the guaranteed homeomorphisms from $[a,b]$ to $S^1$, is a homeomorphism by the Gluing Lemma. 
        \end{proof}

    \item For each of the following cases, we choose as base point in $C$ and describe the generators for the fundamental groups of $C$ and $S$. Further, we write down the homomorphism, in terms of these generators, the fundamental groups induced by the inclusion of $C$ in $S$. 
        \par Consider the following examples of a circle $C$ embedded in a surface $S$: 
        \begin{enumerate}
            \item $S = \text{M{\"o}bius Strip}$ and $C = \partial S$: 
            \item $S = S^1 \times S^1 = T^1$ and $C = \{(x.y) \in T^1 : x = y\}$:
            \item $S = S^1 \times I$ and $C = S^1 \times 1$:
        \end{enumerate}

    \item Suppose that $f,g: S^1 \rightarrow X$ are homotopic maps. We prove that the spaces formed from $X$ by attaching a disc, using $f$ and using $g$ are homotopy equivalent; in other words, we prove that $X \cup_f D \simeq X \cup_g D$: 
        \begin{proof} We show that this is true for $S^{n-1}$. To show this, we claim that $X \cup_f D^n$ and $X \cup_g D^n$ are deformation retracts of the same space, $X \cup_F (D^n \times I)$, where 
            \[f \simeq_F g\]
            As $F: S^1 \times I \rightarrow S^1$ is a homotopy between $f$ and $g$, we have that 
            \begin{align*}
                F(x,0) & = f(x) \\ 
                F(x, 1) & = g(x) 
            \end{align*}
            We claim that the function $r: D^n \times I \rightarrow D^n \times \{0\} \cup S^{n-1} \times I$ given by 
            \[ r(x,t) = 
            \begin{cases}
                \big( \frac{2x}{2-t} , 0 \big) & t \in [0, 2(1-||x||)] \\
                \big( \frac{x}{||x||}, 2 - \frac{2-t}{||x||} \big) & t \in [2(1-||x||), 1]
            \end{cases}
            \]
            is a retraction. 
            \begin{Lemma} \textit{$r$, as defined above is a retraction: }
                \par In the first case, suppose that 
                \[(x,t) \in D^n \times \{0\} \cup S^{n-1} \times I = A \]
                is such that $x \in D^n$, and $t = 0$. Then, we have that 
                \[r(x,t) = r(x,0) = (x,0)\]
                In the second case, if $t > 0$, and $x \in S^{n-1}$, then $||x||= 1$ and so $ t \in [0,1]$ and
                \[r(x,t) = \Big( \frac{x}{1} , 2 - \frac{2-t}{1}\Big) = (x,t) \]
                Thus, $r|_A = 1_A$. A similar argument shows that $r(D^n \times I) = A$.
                \par To complete the proof of the lemma, we note that as $r$ is polynomial in $x$ and $t$, it follows that $r$ is continuous. Consequently, $r$ is a retraction. 
            \end{Lemma}

            We further claim that $r$ is a deformation retraction. To show this, consider the function 
            \[d: (D^n \times I) \times I \rightarrow D^n \times I\]
            \[d((x,t),s) = s (r(x,t)) + (1-s)(x,t) \]
            As 
            \begin{align*}
                d((x,t),0 & = (x,t) \\
                d((x,t), 1) & = r(x,t)
            \end{align*}
            it follows that $d$ is a homotopy between $r$ and the identity. Consequently, $r$ is a deformation retraction. 
            \par Using the fact that $r$ is a deformation retraction we have that $X \cup_F (D^n \times I)$ deformation retracts onto 
            \[X \cup_F (D^n \times \{0\} \cup S^{n-1} \times I) = X \cup_f (D^n \times \{0\}) \cong D^n   \]
            (where $X \cup_F (D^n \times \{0\} \cup S^{n-1} \times I) \cong (D^n \times \{0\} \cup_f X)$ via the map sending $D^n \times \{0\}$ and $X$ to itself, and $S^{n-1} \times I$ to $F(S^{n-1} \times I)$.)
            \par A similar argument show a deformation retract of $D^n \times I$ onto 
            \[D^n \times \{1\} \cup S^{n-1} \times I\]
            which gives the identification space with $g$ as the attaching map. And, the result follows. 
        \end{proof}

    \item We use the previous problem, and the third example of homotopy given in section $5.1$, to show that the 'dunce hat' has the homotopy type of a disc, and is therefore contractible: 
        \begin{proof} This is so horrendously explained by Armstrong. By "the third example of homotopy given in section 5.1," we assume that Armstrong really means "(\textit{to take as the canonical definition of 'dunce cap') that the dunce cap is constructed by } gluing $D^2$ to $S^1$ via the map $g: S^1 \rightarrow S^1$ given by 
            \[ g(e^{i\theta}) = 
            \begin{cases}
                e^{4i\theta} & 0 \leq \theta \leq \pi/2 \\
                e^{4i(2\theta - \pi)} & \pi/2 \leq \theta \leq 3\pi/2 \\
                e^{8i(\pi - \theta)} & 3\pi/2 \leq \theta \leq 2 \pi 
            \end{cases}"
            \]
            That is, $S^1 \cup_g D^2$ is the dunce hat. With this assumption, we continue with the proof: 
            \par Consider the adjunct space given by $i: S^1 \rightarrow D^2$; $X \cup_i D^2$. To conclude, we show that $g$ is homotopic to the identity map: 
            \par Consider $F: S^1 \times I \rightarrow S^1$ given by 
            \[ F( e^{i\theta} , t) = tg(e^{i\theta}) + (1 - t)e^{i\theta}\]
            This is the straight line homotopy; 
            \[g \simeq_F i\]
            Thus, by exercise twenty-seven, page one-hundered-nine, $i,g : S^1 \rightarrow D^2$ are homotopic, so that 
            \[S^1 \cup_g D^2 \simeq S^1 \cup_i D^2 \cong D^2\]
            and the result follows. 
        \end{proof}

    \item We show that the 'house with two rooms' is contractible: 
        \begin{proof} A rigorus proof of this theorem is probably beyond the scope of this section. However, a sketch of it is given in Allen Hatcher's \textit{Algebraic Topology}, Chapter 0.  text\footnote{cite this}.
        \end{proof}

    \item We give a detailed proof to show that the cylinder and the M{\"o}bius strip have the homotopy type of the circle: 
        \begin{proof} We have previously shown that both the cylinder and the M\"obius strip are homotopic to $S^1$. For a more general argument: 
            \par A deformation retract $F:X\times I\rightarrow X$ of $X$ onto $A\subset X$ induces a homotopy equivalence by taking $F(-,1):X\rightarrow A$ and the inclusion $\iota:A\rightarrow X$. Thus, the composition of the homotopy equivalences $M\rightarrow S^1\rightarrow S^1\times I$ prove the result. 
        \end{proof}

    \item Let $X$ be the comb space. We prove that the identity map of $X$ is not homotopic $\text{rel} \{p\}$, to the constant map, $p = (0,1/2)$: 
        \begin{proof} This is a problem in Munker's \textit{Topology}. In his text, the point $p$ is $(0,1)$. We proceed with this assumption: 
            \par Let $p = x_0=(0,1)$. Consider a neighborhood $U$ of $x_0$ which is disjoint from $I\times\{0\}$. Suppose $H:X\times I\to X$ is the homotopy starting with the identity on $X$ and ending with the constant map $X\to\{x_0\}$ such that $H(x_0,t)=x_0$ for all times $t$. That means $\{x_0\}\times I$ has a neighborhood $H^{-1}(U)$. By the tube lemma (this is the statement that in the product topology, there is an open set containing a compact product), there is an open set $V$ such that $\{x_0\}\times I\subset V\times I\subset H^{-1}(U)$. That means every point in $V$ stays in $U$ during the entire deformation. However a point $y=(a,1)$ must traverse a path to $x_0$ and no such path exists within $U$.
        \end{proof}

    \item \textit{FTA}: We prove the fundamental theorem of algebra: 
        \begin{proof} This is shown in Munker's \textit{Topology}, page three-hundered-fifty. 
        \end{proof}
\end{enumerate}

\section{Brouwer Fixed-Point Theorem}
\par Throughout, we say a topological space $(X, \tau_X)$ has the fixed-point property if every continuous function from $X$ to itself has a fixed point. 

\begin{enumerate}[(1)]
    \item We determine which of the following have the fixed point property: 
        \begin{itemize}
            \item The $2$-Sphere: $S^2$ does not exhibit the fixed point property. 
                \begin{proof}
                    We have previously shown that the map $f:S^n \rightarrow S^n$, given by $f(x) = -x$ is a homeomorphism.
                \end{proof}

            \item The Torus: $T^1 = S^1 \times S^1$ does not exhibit the fixed point property. 
                \begin{proof}
                    As $T^1 = S^1 \times S^1$, and the antipodal map, $f$, on $S^1$ is a homeomorphism, we have that $F: T^1 \rightarrow T^1$, defined by $F((x,y)) = (f(x),f(y))$, is continuous and doesn't exhibit the fixed point property. 
                \end{proof}

            \item The interior of the unit disc: $\text{Int}D^1$ does not exhibit the fixed point property: 
                \begin{proof} Note that the interior of the unit disc is homeomorphic to the euclidean plane by a homeomorphism $h: \text{Int}D^1 \rightarrow \mathbb{E}^2$. We define a function $g$ from the euclidean plane to the euclidean plane by $g((x,y))=(x+1,y)$. Then, $h^{-1}\circ g\circ h$ is a continuous function from the disc to itself that has no fixed point.
                \end{proof}

            \item The one point union of two circles: $X \vee Y = \nicefrac{ X  \amalg Y }{\{p\}}$ does not exhibit the fixed point property: 
                \begin{proof} Define a function $f$ as follows: If $x\in X$ then we map $e^{ix}$ to $e^{i(x+\pi)}$. Also if $y \in Y$, we map $y$ to $f(p)=e^{\nicefrac{i\pi}{2}}$. This function $f$, as defined on the one point union, leaves no points fixed.
                \end{proof}
        \end{itemize}

    \item Suppose $X$ and $Y$ are of the same homotopy type and $X$ has the fixed-point property. We prove that $Y$ does not necessarily have the fixed point property: 
        \begin{proof} Let $Y$ be the subspace $(0,1) \subset \E^n$, and let $X = \{1/2\}$. Note that $Y$ is homotopic to $X$ by the straight line homotopy, and every map from $X$ to itself has a fixed point. Yet, the function $f: Y \rightarrow Y$, defined by $f(y)=y^2$ has no fixed point. 
        \end{proof}

    \item Suppose that $X$ retracts onto the subspace $A \subset X$, and that $A$ has the fixed point property. We show that $X$ may not exhibit the fixed point property: 
        \begin{proof} Take $X$ and $Y$, as above. The straight-line homotopy proves the assertion. 
        \end{proof}


    \item We show that if $X$ retracts onto the subspace $A$, and $X$ has the fixed-point property, then $A$ also has it: 
        \begin{proof} Let $f:A\rightarrow X$ be a continuous function. Since $X$ retracts onto $A$, there exists a map $g:X\rightarrow A$ such that $g \restriction{A} \equiv 1_A$. Then, $f \circ g$ is a continuous function from $X$ to $X$, and so has a fixed point. Hence, there exists an $a\in A$ such that $f(g(a))=f(a)=a$. This completes the proof. 
        \end{proof}

    \item We deduce that the fixed-point property holds for the 'house with two rooms', $X$: 
        \begin{proof} As was previously show, $X$ is contractible to some $x_0$. Thus, there exists some map $F: X \times I \rightarrow X$ such that $F(x,0) = x$, and $F(x,1) = x_0$. Thus, by extra lemma \footnote{include this}, every map $f:X \rightarrow Y$ is null-homotopic. In particular, this includes that maps $g: X \rightarrow X$. Thus, $X$ has the fixed point property\footnote{If this is not totally clear, see extra lemma.}. 
            \par To use the previous problems hints, we could think about starting with the unit cylinder, and pushing in the areas from the top and bottom. However, this method is not rigorous. 
        \end{proof}

    \item Let $f$ be a fixed-point-free map from a compact metric space $(X,d)$ to itself. We prove there is a positive number $\epsilon$ such that $d(x,f(x)) > \epsilon$, $\forall x\in X$: 
        \begin{proof} We show the contrapositive; Suppose that for all $\epsilon>0$ there exists an element $x \in X$ such that $d(x,f(x)) \leq \epsilon$. Pick $x_1\in X$ such that $d(x_1,f(x)) \leq \frac{1}{2} < 1$.
            \par It follows that there exists a set, $\{x_1,\dots,x_n\}\subset X$ such that $d(x_i,f(x_i)) <d(x_k,f(x_k))$, where $i<k$ and $d(x_k,f(x_k))<\dfrac{1}{k}$, for all $1\leq k\leq n$.
            \par Now, let 
            $$\epsilon_0 = \dfrac{1}{2}\min \Big\{d(x_1,f(x_1)),\dots,d(x_n,f(x_n),\frac{1}{n+1}\Big\}$$
            Then there exists an element $x\in X$ such that $d(x,f(x))\leq\epsilon_0$. Pick $x\in X$ that satisfies this property, and call this $x_{n+1}$. Note here that $d(x_n,f(x_n)) < d(x_{n+1},f(x_{n+1}))$ by construction. So that for all $i<n+1$, $d(x_i,f(x_i))\leq d(x_n,f(x_n))< d(x_{n+1},f(x_{n+1}))$. Thus by induction, we have created an infinite sequence $\{x_n\}_{n = 1}^\infty$ of distinct points such that $d(x_k,f(x_k))<\nicefrac{1}{k}$ for each positive integer $k$. And, $\{d(x_n,f(x_n))\}_n$ is a monotone decreasing sequence.
            \par Since $X$ is compact, every infinite subset has a limit point. Therefore $\{x_n\}_{n=1}^\infty $ has a limit point $x \in X$. 
            \par Let $\epsilon > 0$ be given. Since $f$ is continuous at $x$, then there exists a $\delta >0$ such that $d(x,a)<\delta\implies d(f(x),f(a)<\nicefrac{\epsilon}{3}$ for all $a\in X$. By the Archimedian Property, we can find a positive integer $k$, such that 
            $$\frac{1}{k}<\frac{\epsilon}{3}$$
            Set $r= \min\{\dfrac{\epsilon}{3},\delta\}$. Then, since $x$ is a limit point of $\{x_n\}_{n = 1}^\infty$, there are infinitely many points of the sequence, such that $d(x_N,x)<r$. Thus, there exists a point $x_N$ such that $N>k$ and $d(x_N,x)<r$. Since $d(x_N,x)<\delta$, then $d(f(x_N),f(x))<\nicefrac{\epsilon}{3}$. Since $N>k$,  $$d(x_N,f(x_N)) < d(x_k,f(x_k)) < \nicefrac{1}{k} < \frac{\epsilon}{3}$$
            Therefore $$d(x,f(x))\leq d(x,x_N)+d(x_N,f(x_N))+d(f(x_N),f(x))<\frac{\epsilon}{3}+\frac{\epsilon}{3}+\frac{\epsilon}{3}=\epsilon$$
            \par Thus, for all $\epsilon>0$, we have that $d(x,f(x))<\epsilon$. Thus, $d(x,f(x)) \leq 0$, and hence $d(x,f(x))=0$. Consequently, $f(x)=x$, and so $f$ is not a fixed-point-free map.
        \end{proof}

    \item We show that the unit ball, $B^n$, in $\E^n$, with $(1,0,\dots,0)$ removed does not exhibit the fixed point property: 
        \begin{proof} Let $p=(1,\dots,0)$. Consider the function $f:B^n\setminus \{p\} \rightarrow B^n\setminus \{p\}$, defined by $f(x) = \frac{x+p}{2}$. We show that $f$ is continuous: 
            \par Let ${\Ocal}$ be an open set in $B^n\{p\}$. Let $x \in f^{-1}({\Ocal})$. Then $f(x) \in {\Ocal}$, and since ${\Ocal}$ is open there exists an $r>0$ such that $B_r(f(x))\subset {\Ocal}$. Let $y\in B_{2r}(x)$. Then $|x-y|<2r$. Then, 
            $$|f(x)-f(y)|=|\frac{x+p}{2}-\frac{y+p}{2}|=\frac{1}{2}|x-y|<r$$
            Thus, $y\in B_r(f(x))$. Hence, $y\in f^{-1}({\Ocal})$. Thus, $B_{2r}(x)\subset f^{-1}({\Ocal})$. Therefore, $f^{-1}({\Ocal})$ is open and $f$ is continuous.
            \par Further, $f$ does not have the fixed-point property; If $f(x) = x = (x_1,\dots, x_n)$, then we have $2x = x+p$. Implying that $x_i = 2x_i$, $2 \leq i \leq n$. Thus, $x_i = 0$. However, then the only solution to $x_1 + 1 = 2x_1$ is $x_1 = 1$. This is a contradiction since we must have $x = p \notin B^n\setminus\{p\}$. 
        \end{proof}

    \item We show that the one-point union of $X$ and $Y$, $X \vee Y$, has the fixed-point property if, and only if, both $X$ and $Y$ have it:
        \begin{proof} 
            $ $\newline
            \begin{itemize}
                \item[] ($\implies$): Suppose that the one point union $X\vee Y$ has the fixed-point-property. Let $f:X\rightarrow X$ be a map, and let $p$ be the point glued together in the one point union. Then, define $g:X\vee Y\rightarrow X\vee Y$, as $g(x)=f(x)$, if $x\in X$, and $g(x)=f(p)$, if $x \in Y$. By the gluing lemma, $g$ is a continuous map and so, by hypothesis, has a fixed point $x_0$. Note that $g$ is a map into $X$. Thus the fixed point must be in $X$. Hence, by construction, $x_0=g(x_0)=f(x_0)$, so that $f$ has a fixed point. Similarly, any continuous function from $Y$ to itself has a fixed point.
                \item[] ($\impliedby$): Suppose $X$ and $Y$ have the fixed point property and let $f:X\vee Y\rightarrow X\vee Y$ be a map. Then, suppose $f(p)\in X$, and define a map $g:X\rightarrow X$ such that $g(x)=f(x)$, if $f(x)\in X$, and $g(x)=p$, if $f(x)\notin X$. Then, since $g$ is continuous, it has a fixed point. By construction, the fixed point must be one $x = g(x) = f(x)$. Thus, $f$ has a fixed point. Next, suppose $f(p)\in Y$, and define a map $g:Y\rightarrow Y$ such that $g(y)=f(y)$, if $f(y)\in Y$, and $g(y)=p$, if $f(y)\notin Y$. Then, since $g$ is continuous, it has a fixed point. By construction, the fixed point must be one $y=g(y)=f(y)$. Consequently, $f$ has a fixed point.
            \end{itemize}
        \end{proof}

    \item How does changing 'continuous function' to 'homeomorphism' in the definition of the fixed-point property affect problem $33$, $37$?
        \begin{proof} We first examine problem $33$: 
            \begin{itemize}
                \item[] $S^2$: This topological space would not exhibit the fixed point property. We know that the antipodal map is a homeomorphism which leaves no points fixed. 
                \item[] $S^1 \times S^1$: Likewise, the antipodal map on each component, $S^1$, provides another counter example.
                \item[] $\Int{D}$: Neither does this space have the homeomorphic fixed point property. To see this, we note that $\Int{D} \simeq \E$, and the map $g: \E \rightarrow \E$, given by $g(x) = x+1$ is a homeomorphism which leaves no points fixed.
                \item[] $X \bigvee Y$: First, consider the map $E_1: X \bigvee Y \rightarrow X \bigvee Y$, given by $E_1(x) = x$, if $x \in X -\{p\}$. And, $E_2(y) = e^{yi\nicefrac{\pi}{2}}$. Then, $E_1$ is a homeomorphism, which leaves $X$ fixed, except for $p$. Similarly, define $E_2$ for $Y$. Then, it follows that $E_2 \circ E_1:X \bigvee Y \rightarrow X \bigvee Y$ is a homeomorphism that leaves no points of $X \bigvee Y$ fixed. Thus, $X \bigvee Y$ does not exhibit the fixed point property\footnote{We make use of the previous proof; that is $X \bigvee Y$ has the fixed point propert iff $X$ and $Y$ both have it.}. 
            \end{itemize}
            \par Now, we examine problem $37$:
            \par This shape does not exhibit the homeomorphic fixed point property. Consider $\partial B^n = S^{n-1}$, and let $p = (1,0,\dots,0)$. Then, $B^n - \{p\} = \Int{B^n} \cup S^{n-1} - \{p\}$. Furthermore, without loss of generality, $S^{n-1} -\{p\} \simeq \{-p\}$.\footnote{See previous exercises.} It follows that $\Int{B^n} \cup S^{n-1} - \{p\} \simeq \Int{B^n} \cup \{-p\}$. To conclude, we exhibit a homeomorphism on $\Int{B^n} \cup \{-p\}$  which does not have a fixed point: 
            \par Consider $f: \Int{B^n} \cup \{-p\} \rightarrow \Int{B^n} \cup \{-p\}$ defined by 
            \[ f(x) = 
            \begin{cases}
                -x & x \notin \{-p, 0\} \\
                0 & x = -p \\
                -p & x = 0 
            \end{cases}\]
            The fact that $f$ is 1-1 and onto is clear by construction. Further, $f$ is continuous, by the Gluing Lemma\footnote{Armstrong, pg. $69$.}. It follows that $f$ is a homeomorphism, as $f^{-1} \equiv -f$. However, $f$ clearly leaves no points fixed. 
        \end{proof}

\end{enumerate}

\section{Separation of the Plane}
Most of the exercises in the section are too advanced for an undergraduate course in algebraic topology. As a result, we refer to well-known sources, which prov e the results. 
\begin{enumerate}[(1)]
    \item Let $A$ be a compact subset of $\E^n$. We show that $\E^n - A$ has exactly one unbounded component: 
        \begin{proof} As $A$ is compact, it is closed and bounded. Further, we identify $\E^n$ with $S^n - \{p\}$ under stereographic projection, $\pi$. Then, $A$ is a compact set in $S^n$ such that $\{p\} \not\subset A \cap S^n$. The  remainder of this proof is an extension of lemma ($61.1$), Munkers' \textit{Topology}\footnote{cite this}. 
        \end{proof}

    \item Let $J$ be a polygonal Jordan curve in the plane. Let $p$ be a point in the unbounded component of $\E^2 - J$ which does not lie on any of the lines produced by extending each of the segments of $J$ in both directions. Given a point $x$ of $\E^2 - J$, say that $x$ is inside (outside) $J$ if the straight line joining $p$ to $x$ cuts across $J$ an odd (even) number of times. We show that the complement of $J$ has exactly two components, namely the set of inside points and the set of outside points: 
        \begin{proof} See \textit{The Jordan-Sch\'onflies Theorem and the Classification of Surfaces}, by Carsten Thomassen. 
        \end{proof}

    \item Let $J$ be a polygonal Jordan curve in the plane, and let $X$ denote the closure of the bounded component of $J$. We show that $X$ is homeomorphic to a disc: 
        \begin{proof} See \textit{The Jordan-Sch\'onflies Theorem and the Classification of Surfaces}, by Carsten Thomassen. 
        \end{proof}

    \item We prove Sch\"onflies theorem for polygonal Jordan curves: 
        \begin{proof} See \textit{The Jordan-Sch\'onflies Theorem and the Classification of Surfaces}, by Carsten Thomassen. 
        \end{proof}

    \item If $J$ is a Jordan curve in the plane, we use theorem ($5.12$) to show that the frontier of any component of $\E^2 - J$ is $J$: 
        \begin{proof} See \textit{The Jordan-Sch\'onflies Theorem and the Classification of Surfaces}, by Carsten Thomassen. 
        \end{proof}

    \item We give an example of a subspace of the plane which has the homotopy type of a circle, which separates the plane into two components, but which is not the frontier of both these components: 
        \begin{proof} Consider the standard annulus in the plane, 
            \[A = \{(x,y) : x^2 + y^2 = r \land 1 \leq r \leq 2 \}\]
            Then, as we have previously shown, $A$ deformation retracts onto $S^1 \subset A$, and so $A$ has the homotopy type of a circle. Further, 
            \[\Int A = \{(x,y) : x^2 + y^2 < 1\},\]
            with $\partial \Int A = S^1 \neq A$. In addition, 
            \[\Out A = \{(x,y) : x^2 +y^2 > 2\},\]
            and $\partial \Out A = 2S^1 \neq A$. 
        \end{proof}

    \item We give example of simple closed curves which separate, and fail to separate
        \begin{itemize}
            \item the torus: 
                \begin{proof} Considering the torus as the identification space of $[0,1]^2$, the circle of radius $1/4$ centered at $(1/2,1/2)$, separates $T$, while the line $x = 1/2$ does not. 
                \end{proof}

            \item $\RP^2$: 
                \begin{proof} Note that $\RP^2$ is given by the equivalence relation 
                    $x ~ \lambda x$, $\lambda\neq 0$. Thus, a curve in $\RP^2$ is a set of equivalence classes. Thus, a circle of radius $1/4$ centered at $(1/2,1/2)$, which is equivalent to an arc on the first quadrant of $S^1$ separates $\RP^2$. And, $S^1$ does not separate $\RP^2$. 
                \end{proof}

        \end{itemize}

    \item Let $X$ be the subspace of the plane which is homeomorphic to a disc. We generalize the argument of theorem ($5.21$) to show that $X$ cannot separate the plane: 
        \begin{proof} See \textit{The Jordan-Sch\'onflies Theorem and the Classification of Surfaces}, by Carsten Thomassen. 
        \end{proof}

    \item Suppose that $X$ is both connected and locally path-connected. We show that a map $f: X \rightarrow S^1$ lifts to a map $f': X \rightarrow \R$ if, and only, if the induced homomorphism $f_*: \pi_1(X) \rightarrow \pi_1 (S^1)$ is the zero homomorphism: 
        \begin{proof} See the accompanying solution\footnote{http://stanford.edu/class/math215b/Sol3.pdf}. 
        \end{proof}

\end{enumerate}

\section{The Boundary of a Surface}
\begin{enumerate}[(1)]
    \item We use an argument similar to that of theorem ($5.23$) to prove that $\E^2$ and $\E^3$ are not homeomorphic: 
        \begin{proof} We have previously show that $\pi_1(\E^m)$ is trivial for $m \geq 3$, while $\pi_1(\E^2)$ is not. The result follows. 
        \end{proof}

    \item We use the material of this section to show that the spaces $X$, and $Y$ illustrated in problem twenty-four of chapter one are not homeomorphic: 
        \begin{proof} Suppose, to the contrary, that $X$ and $Y$ were homeomorphic. Then, by corollary ($5.25$), they have homeomorphic boundaries. Further, it follows from the proof of theorem ($5.24$), that the outer boundaries, $\Out X$, of $X$ and $\Out Y$ of $Y$ must be homeomorphic, as well as the inner boundaries. However, we have previously shown that they are not homeomorphic, as one is separable by removing a point, and the other is not. 
        \end{proof} 

\end{enumerate}

\chapter{Triangulations}
Unfortunately, this is the point in Armstrong when the definitions presented are even less clear. For this reason, the reader is referred to \href{https://www.cs.duke.edu/courses/fall06/cps296.1/Lectures/sec-III-1.pdf}{this short paper} which provides an excellent introduction to simplicial complexes and triangulations; the definitions presented are inline with those in Armstrong.  
\section{Triangulating Spaces}
\begin{enumerate}[(1)]
    \item We construct triangulations for
        \begin{itemize}
            \item the cylinder:
                \begin{proof} Please see figure \hyperref[fig:tikz:triCyl]{\ref{fig:tikz:triCyl}}, with the usual identifications. 
                    \begin{figure}[p]
                        \centering
                        \includestandalone[width=\textwidth]{cylinder}
                        \caption{Triangulation of the cylinder.}
                        \label{fig:tikz:triCyl}
                    \end{figure}
                \end{proof}

            \item the Klein Bottle: 
                \begin{proof} Please see figure \hyperref[fig:tikz:triKlein]{\ref{fig:tikz:triKlein}}, with the usual identifications.
                    \begin{figure}[p]
                        \centering
                        \includestandalone[width=\textwidth]{klein}
                        \caption{Triangulation of Klein bottle.}
                        \label{fig:tikz:triKlein}
                    \end{figure}
                \end{proof}

            \item the double Torus: 
                \begin{proof} Please see figure \hyperref[fig:tikz:triDoubleTorus]{\ref{fig:tikz:triDoubleTorus}}, with the usual identifications. 
                    \begin{figure}[p]
                        \centering
                        \includestandalone[width=\textwidth]{doubleTorus}
                        \caption{Triangulation of the double torus.}
                        \label{fig:tikz:triDoubleTorus}
                    \end{figure}
                \end{proof}
        \end{itemize}

    \item We finish off the proof of lemma ($6.3$):
        \begin{proof} To do this, we mimic the proof theorem ($3.30$) and show that $|K|$ is path-connected; 
            \par Let 
            \[Z = \bigcup_{0 < \delta < \epsilon} (B(x,\delta) \cap |K|) = \Big(\bigcup_{0 < \delta < \epsilon}\Big) \cap |K| \]
            We claim that $|K| = Z$. Since $B(x,\delta)$ is open in $\E^n$ for each $\delta$, and $|K|$ has the subspace topology, $Z$ is open. Further, 
            \[|K| - Z = |L| = \bigcup_{y \in |L|} B(y, \delta_y)\]
            where $\delta_y = d(x,y) - \epsilon$. So, $|K| - Z$ is open. And, since $|K|$ is connected, $Z \neq \emptyset$, theorem ($3.20$) (page fifty-seven), implies $Z = |K|$. 
        \end{proof}

    \item If $|K|$ is a connected space, we show that nay two vertexes of $K$ can be connected by a path whose image is a collection of vertexes and edges of $K$:   
        \begin{proof} Let $v$ be a $0$-simplex in $K$ and let $\mathcal{P}_v$ be the subcomplex of $K$ consisting of all edge path starting with $v$, along with the edge paths they span. 
            \par Now, if $\sigma$ is a simplex of $K$ that has a vertex, call it $w$ in $\mathcal{P}_v$, then ever vertex $w'$ of $\sigma$ is in $\mathcal{P}_v$ (by definition). Thus, $\sigma \in \mathcal{P}_v$.
            \par To conclude the proof, we show that $|\mathcal{P}_v|$ is clopen in $|K|$: The fact that $|\mathcal{P}_v|$ is closed follows from the fact that each $1$-simplex is closed and there are only finitely many. Likewise, $|\mathcal{P}_v|$ is open, as each $1$-simplex is closed. 
            \par Thus, as $v \in \mathcal{P}_v \neq \emptyset$, we have by theorem ($3.20$) that $\mathcal{P}_v = K$ and the result follows. 
        \end{proof}

    \item We check that $|CK|$ and $C|K|$ are homeomorphic spaces: 
        \begin{proof} Let $K$ be a complex in $\E^n$. Then, $CK$ consists of the simplices of $K$ and the join of each of them to some vertex, $v$ via $1$-simplices and the $0$-simplex $v$ itself. So, $|CK|$ is the geometric cone on $|K|$. By lemma ($4.5$), page sixty-eight, 
            \[ |CK| \simeq C|K|\]
        \end{proof}

    \item We show that if $X$ and $Y$ are triangulable spaces, then $X \times Y$ is triangulable: 
        \begin{proof} Let $\kappa: |K| \rightarrow X$, and $\lambda: |L| \rightarrow Y$ be the triangulations of $X$, and $Y$. We claim that the function 
            \[\phi: |K| \times |L| \rightarrow X \times Y, \quad \phi(k,l) = (\kappa(k), \lambda(l))\]
            is the desired triangulation. 
            \par As $\kappa$, $\lambda$ are homeomorphisms, the result follows. 
        \end{proof}

    \item We show that if $K$ and $L$ are complexes in $\E^n$, that $|K| \cap |L|$ is a polyhedron: 
        \begin{proof} Unfortunately, this is another ill-posed question; what Armstrong neglects to say is that $X$ is a polyhedron iff $X$ can be triangulated (this is the canonical definition given elsewhere). Further, this question is really beyond the scope of this chapter. As such, we give a proof which assumes that every convex, compact, connected subset of $\E^n$ can be triangulated; we actually show that $|K| \cap |L|$ is the union of such subsets of $\E^n$, where connected implies a maximally connected component (we note that it need not be the case that $|K| \cap |L|$ is connected. Consider figure ):  
            \par Let 
            \[|K| = \bigcup_{\sigma \in |K|} \sigma, \quad |L| = \bigcup_{\sigma \in |L|} \sigma \]
            be the simplicial decompositions of $|K|$ and $|L|$. 
            Then, as the intersection of convex sets is convex, we have 
            \[|K| \cap |L| = \bigcup_{\substack{\sigma \in |K| \\ \sigma' \in |L|}} (\sigma \cap \sigma')\]
            is convex and each $\sigma \cap \sigma'$ is convex. 
            \par Now, as $\dim K,  \dim L < N < \infty$, by the well-ordering principle, we start with the smallest dimensional intersection of simplices, say $n \in \N$. Pick any such (non-empty) intersection and label it $\kappa^n_1$. Problem $3$, shows that $\kappa^n_1$ is edge connected and further, as each simplex is completely determined by its vertexes, $\kappa^n_1$ determines some maximally connected, convex, compact subset of $\E^n$.   
            %\par Then, let $L^m$ be the $m$-dimensional subcomplex of $L$. We proceed by induction on $m$ and show that $|K| \cap %|L^m|$ is a polyhedron, $n \leq m \leq N$:
            %\par But this is clear, as for each $m$, 
            %\[L^m = \bigcup_{\substack{\sigma \in |L| \\ \dim \sigma \leq m}} \sigma\]
            %and $K \cap L^{m+1}$ has a simplicial decomposition that agrees with the give one on $K \cap L^m$. 
            \par To conclude the proof, we extend this process to some other intersections 
            \[\kappa^n_1, \kappa^n_2, \kappa^n_2, \dots, \kappa^n_q, \quad \dim \kappa^n_i = \dim \kappa^n_j = n\]
            This method can be extended to the next higher' dimension $n+1$ to show that 
            \[|K| \cap |L| = \bigcup_{\substack{n \leq i \leq N \\ j}} |\kappa^i_j| \]
            which is indeed triangulable, and so a polyhedron. 
        \end{proof}

    \item We show that $S^n$ and $P^n$ are both triangulable: 
        \begin{proof} We give a triangulation of $S^n$ and use the antipodal map to establish a triangulation for $P^n$: 
            \par Let $(S^n)^+$ denote the northern hemisphere of $S^n$ and consider $S^n \subset \E^n$ and the canonical basis
            \[\{e_1, e_2, \dots, e_n\}, \quad e_i = (0, \dots, 0, 1_i, 0, \dots , 0)\]
            We claim that the set $X$ of $1$-simplices given by 
            \[\sigma_i = e_ie_{i+1}, \quad 2 \leq i \leq n, \quad n+1 \sim 2,\]
            along with the set of $1$-simplices defined by 
            \[\sigma_{Ni} = Ne_i, \quad N = (1, \dots, 0), \quad 2 \leq i \leq n \]
            along with the $2$-simplexes $\sigma_i$, $\sigma_{Ni}$ bound, and all their vertexes is a triangulation of the norther hemisphere of $S^n$: 
            \par Indeed; The fact that $X$ is a simplicial complex is clear from construction and the definition of $S^n$ and by radial projection, $\pi$, we have 
            \[\pi: |X| \rightarrow (S^n)^+ \]
            is a homeomorphism, and so $\pi$ is a triangulation of $(S^n)^+$. Further by considering the simplexes 
            \[\sigma_{Si} = Se_i, \quad S = (0, \dots, 1), \quad 2 \leq i \leq n \]
            we extend $\pi$ to a map 
            \[\Pi: |X \cup Y| \bij S^n\] 
            It follows that $S^n$ is triangulated by $\Pi$. 
            \par To show that $\RP^n$ is triangulable, we note that the antipodal map 
            \[i_{-1}: S^n \rightarrow S^n\]
            is a homeomorphism and that $\RP^n$ is defined as the identification map induced by $i_{-1}$, call it $\phi$. 
            \par Consequently, 
            \[  \phi\Pi : |X \cup Y| \rightarrow \RP^n \]
            is a homeomorphism, and so a triangulation. 
            %                  \begin{figure}
            %                      \centering
            %                      \includestandalone[width=\textwidth]{triangulationSphere}
            %                      \caption{Triangulation of $S^2$.}
            %                      \label{fig:tikz:triSphere}
            %                  \end{figure}
        \end{proof}

    \item We show that the 'dunce hat' is triangulable, but the 'comb space' is not: 
        \begin{proof} Please see figure \hyperref[fig:tikz:dunceHatTri]{\ref{fig:tikz:dunceHatTri}}, with the usual identifications. 
            \begin{figure}[p]
                \centering
                \includestandalone[width=\textwidth]{dunceHat}
                \caption{The second barycentric subdivision of the 'dunce cap' yeilds a triangulation.}
                \label{fig:tikz:dunceHatTri}
            \end{figure}
            \par To show that the comb space $X$ is not triangulable, we first prove that it is not locally connected: 
            \begin{Lemma}
                If $X$ were locally connected we could find a connected open neighbourhood $\Ocal$ for each each neighbourhood of $p$. But, each open neighbourhood $\Ocal$ contains infinitely many terms of the sequence 
                \[\Big\{\Big( \frac{1}{n}, \frac{1}{2} \Big)\Big\}_{n=1}^{\infty}\]
                As 
                \[ \Big\{(\frac{1}{n},\frac{1}{2}) : n \geq n_0  \Big\}\]
                is clearly disconnected, it follows that $X$ is not locally connected. 
            \end{Lemma}

            \par From lemma ($6.3$), it follows that $|K|$ is locally connected (this is shown in \cite{hatcher2002algebraic}). Now, since local connectedness is a topological property, if $X$ were triangulable then it must be locally connected. However, we have just shown that $X$ is not locally connected.  
        \end{proof}

\end{enumerate}

\section{Barycentric Subdivision}
\begin{enumerate}[(1)]
    \item We present a visualization of the first barycentric subdivision of a $3$-Simplex: 
        \begin{proof} Please see figure \hyperref[fig:baryTetra]{\ref{fig:baryTetra}}. 
            \begin{figure}[p]
                \centering
                \includegraphics[width = \textwidth]{baryTetra(1).jpg}
                \caption{The first barycentric subdivision of the standard $3$-simplex.}
                \label{fig:baryTetra}
            \end{figure}
        \end{proof}

    \item Let $\mathcal{F}$ be an open cover of $|K|$, we show the existence of a barycentric subdivision $K^r$, with the propert that given a vertex $v$ of $K^r$, there is an open set $U \in \mathcal{F}$ which contain all the simplexes of $K^r$ that have $v$ as a vertex: 
        \begin{proof} This is an application of Lebesgue's lemma. 
        \end{proof}

    \item Let $L$ be a subcomplex of $K$, and let $N$ be the following collection of simplexes of $K^2$: a simplex $B$ lies in $N$ if we can find a simplex $C$ in $L^2$ such that the vertexes of $B$ and $C$ together determine a simplex of $K^2$. We show that
        \begin{itemize}
            \item $N$ is a subcomplex of $K^2$: 
                \begin{proof} Let $v$ be a vertex in $N$. Since $B \subset N$, there exists a simplex 
                    \[C \in L^2\]
                    such that the vertexis of $B$ and $C$ together will determine a simplex of $K^2$. Then, $V$ and $C$ will determine a sub-simplex of the above simplex, $v \in N$. 
                    \par Now, let $B'$ be  a sub-simplex of $B$. Then, $B'$ and $C$ determine a sub-simplex of the simplex determined by $B$ and $C$ in $K^2$, which implies that 
                    \[B' \in N\]
                    So, $N$ is a sub-complex. 
                \end{proof}

            \item $|N|$ is a neighbourhood of $|L|$ in $|K|$: 
                \begin{proof} Let $y \in |L|$ and let $V$ be the union of simplices in $K^2$ which contain $y$. Then, $V$ contains a neighbourhood of $y$. Since every simplex in $V$ is contained in $|N|$, $|N|$ is a neighbourhood of $|L|$ in $|K|$.   
                \end{proof}
        \end{itemize}

    \item We use the construction of problem $11$ to prove that if $X$ is a triangulable space, and $Y$ is a subspace of $X$ which is triangulated by a subcomplex of some triangulation of $X$, then the space obtained from $X$ by shrinking $Y$ to a point is triangulable: 
        \begin{proof}
        \end{proof}
\end{enumerate}

\section{Simplicial Approximation}
\begin{enumerate}[(1)]
    \item We use the simplicial approximation theorem to show that the $n$-sphere is simply connected for $n \geq 2$: 
        \begin{proof} This is a special case of the next problem; consider 
            \[S^1 = [0,1]/\{0,1\}\] 
        \end{proof}

    \item We show that if $k < m,n$ that any map from $S^k$ to $S^m$ is null homotopic, and that the same is true of any map from $S^k$ to $S^m \times S^n$:   
        \begin{proof} Let 
            \[h: |K| \rightarrow S^k \quad g:|L| \rightarrow S^m\] be triangulations of $S^k$ and $S^m$, $k<m$. Suppose that 
            \[f: S^k \rightarrow S^m\] is a map. Then, we have a map of complexes, 
            \[g^{-1}\circ f \circ h : |K| \rightarrow |L|\]
            which has a simplicial approximation 
            \[s: |K| \rightarrow |L| \]
            Since 
            \[ \dim |K| < \dim |L|,\] $s$ is not onto. Further, since for any point (we think of vertexes) $p \in S^m$,
            \[|L| - p \cong S^m - p \cong \R^m\]  
            it follows that $s$ is null homotopic ($\R^m$ is convex). 
            \par But, since $s \simeq g^{-1}\circ f \circ h \simeq f$ (as $g$ and $h$ are homeomorphisms), we must have $f$ is null-homotopic. 
            \par For the second part of the proof, stereographic projection proves the result.
        \end{proof}

    \item We show that a simplicial map from $|K|$ to $|L|$ induces a simplicial from $|K^m|$ to $|L^m|$ for any $m$: 
        \begin{proof} Let $s: |K| \rightarrow |L|$ be the simplicial map. Then, for $K^1$, $L^1$, the barycenter of each simplex of $K$ and each simplex of $L$ can be associated via $s$; since the vertexes of $K$ are mapped to vertexes of $L$ and extended linearly, each simplex of $K$ is associated with a simplex in $L$. So, each barycentre living in a simplex of $K$ can be mapped to a barycenter of the associated simplex of $L$, via $s$ and extended linearly. Then, this new function, $s^1: |K^1| \rightarrow |L^1|$ is continuous. 
            \par Far any iteration, we repeat this process, extending each barycentre linearly. To obtain a simplicial map, 
            \[s^m: |K^m| \rightarrow |L^m|\]
        \end{proof}

    \item Suppose that $s: |K^m| \rightarrow |L|$ simplicially approximates $f: |K^m| \rightarrow |L|$, and $t: |L^n| \rightarrow |M|$ simplicially approximates $g: |L^n| \rightarrow M$. We determine if $ts: |K^{m+n}| \rightarrow |M|$ is always a simplicial approximation for $gf: |K^{m+n}| \rightarrow |M|$:   
        \begin{proof} This is essentially the statement that the composition of simplicial maps is simplicial. We claim that the answer is yes:
            \par A small induction argument, using lemma ($6.4$) page one-hundered-twenty-six, shows that for $m \geq 0$, $|K^{m+n}| = |K|$. Now, let $x \in |K^{m+n}| = |K|$. Then, 
            \[(g \circ f)(x) = g(f(x)) \]
            and, by definition, we have that $s(x) \in \car f(x)$. This implies that $t(s(x)) \in \car g(f(x))$, as $t$ is simplicial. However, by definition we then have $g(f(x)) \in t(A)$, for some simplex $A$ of $L^n$. But then, by definition we have that $ts$ simplicially approximates $gf$. 
        \end{proof}

    \item We use the simplicial approximation theorem to show that the set of homotopy classes of maps from one polyhedron to another is always countable: 
        \begin{proof} By the simplicial approximation theorem, we know that any map 
            \[f: |K| \rightarrow |L|\]
            is homotopic to a simplicial map, 
            \[s: |K^m| \rightarrow |L|\]
            But, by definition, there are only a finite number of such choices for $s$, yet $m \in \N$, and the result follows. 
        \end{proof}

    \item This is left to the reader as an exercise. 

\end{enumerate}

\section{The Edge Group of a Complex}
\begin{enumerate}[(1)]
    \item We use van Kampen's theorem to calculate the fundamental group of the double torus by dividing the surface into two halves (each of which is a punctured torus). And then, we do the calculation again, but this time splitting the surface into a disc and the closure of the complement of the disc: 
        \begin{proof} To provide a rigorus proof of this is really beyond the scope of this text. For a complete answer see \footnote{cite this}, \cite{hatcher2002algebraic}, and consider the following explanation given by \footnote{cite this}: 
            \par By van Kampen's theorem, what you get is actually
            $$\pi_1(T)\ast_{\pi_i(S^1)}\pi_1(T)$$
            which is an amalgamated product (a pushout in the category of groups).
            Roughly speaking if you have two groups $G_1$ and $G_2$ and
            embeddings $i_1$ and $i_2$ of a group $H$ in both then $G_1\ast_H\ast G_2$
            is the group freely generated by the elements of $G_1$ and $G_2$
            but identifying elements $i_1(h)$ and $i_2(h)$ for $h\in H$.
            \par Now $\pi_1(T)$ can be computed using the fact that $T$ deformation retracts
            to a bouquet of two circles. (Think about the standard torus; fix a point
            and look at the circles through it going round the torus in the two natural
            ways.)
        \end{proof}

    \item We show that the edge paths $E_1$ and $E_2$ introduced in the proof of theorem ($6.10$) are equivalent: 
        \begin{proof} This is, essentially, the argument on page one-hundered-thirty-three, applied again while noting that 
            \[|L^m| = |(I \times I)^m| = |L| = |I \times I| \]
        \end{proof}

    \item We prove that the 'dunce hat' is simply connected, using can Kampen's theorem:    
        \begin{proof} Consider figure \hyperref[fig:tikz:dunceCap]{\ref{fig:tikz:dunceCap}}. Let the one simplex (complex) in blue be $K$, and the $2$-simplex bounded if all edges were blue be $F$. Then, as such, we can represent $G(F,L) \cong E(J,v)$ via the generator $a$. As van Kampen's theorem suggests, that the dunce cap is generated by $a$ with the relation $a + a -a = a$, and we have that 
            \[\pi_1 (|J|, v) \times \pi_1(|K|, v) \cong \Z \otimes \{0 \} \cong \Z \]
            \begin{figure}[p]
                \centering
                \includestandalone[width=\textwidth]{dunceCapFund}
                \caption{The dunce cap as the identification space of the triangle. Here, the blue line represent positive orientation, and the red, negative; call these $a$, $a$, $-a$.}
                \label{fig:tikz:dunceCap}
            \end{figure}
            \par For a more in-depth proof, please see \cite{math205topology}. 
        \end{proof}

    \item Let $X$ be a path-connected triangulable space. We examine how attaching a disc to $X$ affects the fundamental group of $X$:  
        \begin{proof} The following proof is from \cite{attachingDisc}: 
            \par Using van Kampen's theorem, let $U=N(X)$ and let $V=N(D)$ where by $N(-)$ we mean 'take a small open neighbourhood'. By definition, it is easy to see that $U\cap V$ is homotopy equivalent to a circle and so the fundamental group of $U\cup V$ is equal to 
            \[\pi_1(U)\ast_{\pi_1(U\cap V)}\pi_1(V),\] 
            where we treat $\pi_1(U\cap V)$ as being the subgroup of $\pi_1(U)$ and $\pi_1(V)$ generated by the class of loops homotopic to the boundary circle $\partial D$.
            \par This is isomorphic to $\pi_1(X)/\langle[\partial D]\rangle$ where here $[\partial D]$ denotes the 'class of maps homotopic to usual inclusion of the circle into the boundary of $D$'.
        \end{proof}

    \item Let $G$ be a finitely presented group. We construct a compact triangulable space which has fundamental group $G$:
        \begin{proof} We present an image, which is by no way exhaustive, yet illustrates the method; Let $G$ be finitely represented: 
            \[G = \{g_1, g_2, \dots, g_n : r_1, r_2, \dots, r_m\}\]
            \par Consider the $i$-th relater 
            \[r_i = a^{n_1}b^{n_2} \dots z^{n_k}, \quad a,b, \dots, z \in G, \quad n_j \in \Z\]
            Then, we can construct a planar 
            \[ |\{a,b, \dots, z\}| \cdot \sum_{i = 1}^k |n_i| \quad \text{-gon} \]
            which represents the relater group and is triangulable. Please see figure \hyperref[fig:tikz:triCompact]{\ref{fig:tikz:triCompact}}. This is clearly a (compact) triangulation. Note that the chords drawn in represent an element of the form 
            \[s^{n_t}\] 
            in the relater expression. 
            \begin{figure}[!tbp]
                \begin{subfigure}[b]{0.4\textwidth}
                    \includestandalone[width = \textwidth]{compactTriGroup}
                    \caption{A triangulation of the $i$-th relator in the finite representation of $G$. Here, $r_i$ is a $15$-gon.}
                    \label{fig:tikz:triCompact}
                \end{subfigure}
                \hfill
                \begin{subfigure}[b]{0.4\textwidth}
                    \includestandalone[width=\textwidth]{triGen}
                    \caption{A triangulation of a group element in $G$, with the single relater.}
                    \label{fig:tikz:triGen}
                \end{subfigure}
                \caption{Compact (triangulable) representation of relaters and a 'single' group element. The colored borders represent group elements.}
            \end{figure}
            \par Now, in the case where an element $g \in G$ is \textit{not} a part of a relater's expression, we have figure \hyperref[fig:tikz:triGen]{\ref{fig:tikz:triGen}}; also triangulable, and compact. 
            \par To conclude, it follows from exercise $5$, page one-hundred-twenty-four, that the finite wedge, 
            \[\bigvee_{i} G_i,\]
            where $G_i$ is the planar representation given above for each $g \in G$, is triangulable. Thus, by construction, we have that 
            \[\pi_1 \left( \bigvee_{i} G_i \right) \cong G\]
        \end{proof}
\end{enumerate}

\section{Triangulating Orbit Spaces}
\begin{enumerate}[(1)]
    \item Suppose $\{V,S\}$ satisfies the hypotheses of the realization theorem, and label the elements of $V$ as $v_1, \dots, v_k$. Suppose that $x_i$ denotes the point 
        \[(i, i^2, \dots, i^{2m+1}) \in \E^{2m+1}\]
        We show that any $2m +2$ of the points $x_1, \dots, x_k$ are in general position, and so that the correspondence $v_i \leftrightarrow x_i$ can be used to realize $\{V,S\}$ in $\E^{2m+1}$:   
        \begin{proof} 
        \end{proof}

    \item By the previous problem the vertex scheme of any one-dimensional complex can be realized in $\E^3$. We find a one-dimensional complex whose vertex scheme cannot be realized in $\E^2$:   
        \begin{proof} Consider the complete graph on four points, figure \hyperref[fig:tikz:K4]{\ref{fig:tikz:K4}}. 
            \begin{figure}
                \centering
                \includestandalone[width = \textwidth]{K4}
                \caption{The complete graph on four vertexes.}
                \label{fig:tikz:K4}
            \end{figure}
            \par Any attempt to realize this in $\E^2$ will result in a self-intersection. Note, the cross of the $1$-simplices is not a simplex, by definition of the vertex scheme. 
        \end{proof}

    \item Consider the antipodal action on $S^2$ and the triangulation shown in fig. $6.17$. We show that the map $\psi: |K|/G \rightarrow |K^1/G|$ is a homeomorphism, and draw the resulting triangulation of the projective plane:     
        \begin{proof}
        \end{proof}

    \item We show that the map $\psi : |K|/G \rightarrow |K/G|$ is a homeomorphism iff the action of $G$ on $|K|$ satisfies:
        \begin{itemize}
            \item The vertexes of a $1$-simplex of $K$ never lie in the same orbit:
                \begin{proof}
                \end{proof}
            \item If the sets of vertexes of $v_0, \dots, v_k, a$ and $v_0, \dots, v_k, b$ span simplexes of $K$, and if $a,b$ lie in the same orbit, then there exists $g \in G$ such that $g(v_i) = v_i$ for $0 \leq i \leq k$ and $g(a) = b$. 
                \begin{proof}
                \end{proof}
        \end{itemize}

    \item We check that the previous conditions are always satisfied if we replace $K$ by its second barycentric subdivision: 
        \begin{proof}
        \end{proof}
\end{enumerate}

\section{Infinite Complexes}
\begin{enumerate}[(1)]
    \item We find the triangulations of the following spaces from the crystallographic groups shown in fig. $4.5$, page eighty-four;  
        \begin{itemize}
            \item the sphere: 
                \begin{proof} 
                \end{proof}

            \item the torus: 
                \begin{proof}
                \end{proof}

            \item the Klein bottle: 
                \begin{proof}
                \end{proof}
        \end{itemize}

    \item We check that the construction illustrated in fig. $6.21$ really can be carried out to produce a realization of $T$ in $\E^2$: 
        \begin{proof} This is left to the reader. 
        \end{proof}

    \item We show that the following collection $X$ of simplexes in $\E^2$ is not a simplicial complex; For each $n \in \N$, we have a vertical $1$-simplex joining $(1/n,0)$ to $(1/n,1)$ and a sloping $1$-simplex with vertexes $(1/n,0)$, $(1/n+1,1)$. In addition, we have a $1$-simplex on the $y$ axis joining $(0,0)$ to $(1,1)$:
        \begin{proof} Consider $0 = (0,0) \in X$. Then, for each $\epsilon > 0$ (small enough), we have that 
            \[B_\epsilon (0) \cap X \]
            is not path connected; to see this, suppose by contradiction that $B_\epsilon (0) \cap X$ was path connected; a simple sequential argument shows that we can find $\epsilon_2$, for which 
            \[B_\epsilon (0) \cap X \cap B_{\epsilon_2} (0) \cap X = \emptyset\]
            \par We now consider the case of removing the $1$-simplex on the $y$ axis: We claim that we do not obtain a simplex; Indeed, considering the sequence 
            \[\{\frac{1}{n}\}_{n=1}^\infty \subset X - Y,\]
            we have that 
            \[\frac{1}{n} \rightarrow 0 \notin X - Y\]
            Consequently, $X - Y$ is not closed. 
        \end{proof}

    \item We determine whether the comb space $C$, or the space illustrated in figure $3.4$, page sixty-three, can be triangulated by an infinite simplicial complex: 
        \begin{proof} Neither of these spaces can be triangulated: each exhibits non local-connectedness. 
        \end{proof}

    \item We show that the free group on a countable number of generators is a subgroup of $\Z \otimes \Z$, and deduce that any subgroup of this group must be free: 
        \begin{proof} We show that $G$ is isomorphic to some subgroup of $Z \otimes Z$ by Cayley's Theorem: 
            \par Suppose that $G$ is countably representable, and without loss of generality, has more than two generators;
            \[G = \innerone{g_1, g_2, \dots , g_n, \dots | r_1, r_2, \dots, r_m, \dots }\]
            Then, we claim that the mapping, 
            \[\phi: G \rightarrow Z \otimes Z, \quad \phi(g_{n_1}^{k_1}\dots g_{n_t}^{k_t}) = a^{n_1}b^{n_1} \dots a^{n_t}b^{n_t}\]
            is a homomorphism. Now, if 
            \[w = g_{n_1}^{k_1}\dots g_{n_t}^{k_t}\]
            is a word in $G$, then
            \[\phi(g_{n_1}^{k_1}\dots g_{n_t}^{k_t}) = a^{n_1}b^{n_1} \dots a^{n_t}b^{n_t} = \phi(g_{n_1}^{k_1}) \dots \phi(g_{n_t}^{k_t}) \]
            Further, $\phi$ is clearly onto. Thus, by Cayley's theorem, $G$ isomorphic to a subgroup of $\Z \otimes \Z$. 
            \par Now, we show that any subgroup of $\Z \otimes \Z$ is free: As
            \[H \leq \Z \otimes \Z = \innerone{a,b | \emptyset }\] 
            $H$ is a nonempty, closed subset of words generated by $a,b$ with identity and inverses. As such, $\text{rank } H$ is countable (we can actually count the words as $\Z \times \Z$ is countable). So, at least (this expresion may not be 'minimal'), 
            \[H = \innerone{\{g_n\} | r_k = e, \quad k = 1, 2, \dots }\]
            But then, by the previous result, 
            \[H \cong S \leq \Z \otimes \Z\]
        \end{proof}

    \item We show 
        \begin{itemize}
            \item if $X$ is the polyhedron of a finite complex, and if $h$ is simplicial, the pointwise periodic implies periodic: 
                \begin{proof}
                \end{proof}

            \item a connected infinite complex $K$ and a simplicial homeomorphism of $|K|$ which is pointwise periodic but not periodic: 
                \begin{proof}
                \end{proof}
        \end{itemize}

    \item We determine whether or not a pointwise periodic homeomorphism of a compact space need be periodic: 
        \begin{proof}
        \end{proof}

    \item Let $G$ be a group of homeomorphisms of the space $X$. We show 
        \begin{itemize}
            \item that if $N \trianglelefteq G$, $G/N$ acts in a natural way on $X/N$ and that $X/G$ is homeomorphic to $(X/N)/(G/N)$: 
                \begin{proof}
                \end{proof}

            \item if $F$ is the smallest normal subgroup of $G$ which contains all the elements that have fixed points, then $G/F$ acts freely on $X/F$ in the sense that only the identity element has fixed points: 
                \begin{proof}
                \end{proof}
        \end{itemize}

    \item Suppose that in addition to the conditions in the previous problem that $X$ is a simply connected polyhedron, $G$ acts simplicially, and $X/G$ is triangulated so that the projection $p: X \rightarrow X/G$ is simplicial. Choose a vertex $v$ of $X$ as a base point and define $\phi: G \rightarrow \pi_1(X/G, p(v))$ as follows: given $g \in G$, foin $v$ to $g(v)$ by an edge path $E$ in $X$; then $\phi(g)$ is the homotopy class of the edge loop $p(E)$. 
        \par We show that $\phi$ is a homomorphism, and that each element of $F$ goes to the identity under $\phi$, and $\phi$ is onto: 
        \begin{proof}
        \end{proof}

    \item We the assumptions of the previous problem, we show that $X/F$ is simply connected, and that the action of $G/F$ on $X/F$ satisfies the hypothesis of theorem ($5.13$): 
        \begin{proof}
        \end{proof}

    \item We deduce theorem ($6.18$): 
        \begin{proof}
        \end{proof}
\end{enumerate}

\chapter{Surfaces}
\section{Classification}
\begin{enumerate}[(1)]
    \item We show that punching a disc out of the sphere and ading a cross cap in its place gives a representation of $\RP^2$ as a surface in $\E^3$ with self-intersections:  
        \begin{proof} Again, this is problem which cannot be proved rigorously at this stage in the text. For a proof of this question, please see \cite{classificaionSurfaces}, \cite{classificationArticle}.
        \end{proof}

    \item Let $X$ consist of $S^2$ plus one extra point $p$. The neighbourhoods of the points of $S^2$ are the usual ones, and those of $p$ are sets of the form $[U - \{(0,0,1)\}] \cup \{p\}$ where $U$ is a neighbourhood of $(0,0,1)$ in $S^2$. We show that $X$ is not Hausdorff, but is locally homeomorphic to the plane: 
        \begin{proof} Let $V$ be an open neighbourhood of $(0,0,1)$ and let $W$ be an open neighbourhood of $\{p\}$. Then, 
            \[W = V - (\{(0,0,1)\} \cup \{p\})\]
            Now, as $V$ is a neighbourhood of $(0,0,1)$, $V \cap W$ must intersect, since each contain a disc or radius $\epsilon$ about $(0,0,1)$, for $\epsilon$ small enough. Clear, $X$ is not Hausdorff.  
            \par To show that $X$ is locally homeomorphic to the plane, let $W$, $V$ be as above and define 
            \[h: W \rightarrow \E^2\]
            as $h(1) = 0$, and stereographicly otherwise. Then, $h$ is a homeomorphism between $W$ and an open disc in $\E^2$.
            \par We now consider whether or not it seems reasonable to call $X$ a surface: It is not, as it will undermine the classification theorem. 
        \end{proof}

    \item We show that the connected sum of a torus with itself is a sphere with two handles, and the connect sum of a porjective plane with itself is a Klein bottle: 
        \begin{proof} Giving an explicit construction for the homeomorphism between $T \# T$ and the sphere with two handles added is tedious. Instead, we note that $T \# T$ has genus $2$. 
            \par To show that 
            \[\RP^2 \# \RP^2 \cong \KB,\]
            we note that 
            \[\RP^2 - D \cong \MB\]
            by definition. The result follows; see figure \hyperref[fig:tikz:KB]{\ref{fig:tikz:KB}}.
            \begin{figure}
                \centering
                \includestandalone[width=\textwidth]{KB}
                \caption{The Klein bottle ($\KB$) as the typical identification space.}
                \label{fig:tikz:KB}
            \end{figure}
        \end{proof}

    \item We determine what the connected sum of a torus and a projective plane is: 
        \begin{proof} To rigorusly show this is beyond this section in the text. But, 
            \[\RP^2 \# T \cong \RP^2 \# \RP^2 \# \RP^2 \cong \RP^2 \# \KB \]
        \end{proof}
\end{enumerate}

\section{Triangulation and Orientation}
I think that on page $154$, in the definition of a combinatorial surface, it should say that each vertex lies in at least \textit{two} triangles, not "three". In addition, Armstrong must have meant that any vertex which \textit{happens to} lie in three triangles fits together as in fig. $7.8$. 
\begin{enumerate}[(1)]
    \item Suppose we want to triangulate a surface which has a boundary. We examine how the definition of a combinatorial surface needs to be adjusted: 
        \begin{proof} We require that the $1$-complex contains the boundary. 
        \end{proof}

    \item Let $K$ be a combinatorial surface. We  
        \begin{itemize}
            \item show that the triangles of $K$ can be labelled $T_1, \dots, T_s$ in such a wa that $T_i$ always has an edge in common with at least one of $T_1, \dots, T_{i-1}$: 
                \begin{proof} We proceed by induction on $i$: 
                    \begin{itemize}
                        \item Base case: Let $i = 2$. Consider the triangle $T_1$, then pick an edge from $T_1$, call it $e$. And, as $K$ is combinatorial, let $T_2$ be a triangle that shares $e$ with $T_1$. The proves the base case.

                        \item Hypothesis: Suppose that this is true for $i \geq n$ for some $n \leq \dim K$. Then, if $T_1, \dots, T_i$ does not have an edge in common with $T_{i+1}$, then we have contradicted the fact that $K$ is connected.
                    \end{itemize}
                    This proves the result. 
                \end{proof}

            \item build a model for the surface of $|K|$ in the form of a regular polygon in the plane, which has an even number of side, and whose sides have to be identified in pairs in some way: 
                \begin{proof} Note quite sure what this means. Perhaps as statement that a square can be subdivided as previously suffices. 
                \end{proof}
        \end{itemize}

    \item We show that if $h: |K| \rightarrow S$ is a triangulation of a closed surface, then $K$ must be a combinatorial surface: 
        \begin{proof} We first show that that $K$ cannot have dimension $1$:
            \begin{Lemma}
                As $h$ is a triangulation of a closed surface, we have that for every $s \in S$, there exists an open neighbourhood $\Ocal \subset S$ and a homeomorphism $\phi$ such that 
                \[\phi: \Ocal \bij U, \quad U \subset \tau_{\E^2}\]
                But then, if $K$ has dimension one, then this is a contradiction to invariance of domain (as it is preserved under homeomorphism). Thus, $\dim K \neq 1$.  
            \end{Lemma}
            By a similar argument as above, it follows that the inverse mapping
            \[h^{-1}\phi^{-1}: U \bij |K|\] 
            implies $\dim K < 3$. As $K$ has dimension $2$ and is a surface, it is connected. Let $\Gamma$ be its edge graph. Further, for each $x \in \Int(\gamma) \subset \Gamma$, there exists an open neighbourhood $\Ocal_x$ which is homeomorphic (by extension) to $\E^2$; an argument as theorem ($5.23$), page one-hundred-sixteen, shows that $\gamma$ must border two $2$-simplexes. Thus, 
            $\deg v \geq 2$ for each vertex $v \in \Gamma \subset |K|$.
            \par To conclude, we show that if a vertex $v$ of $K$ is in at least three triangles, then it forms a vertex for a cone, as in figure $7.8$; But, considering the open start of $v$, this is clear; the $1$ skeleton of the open star defines a tree, and consequently, 
            \[\overline{\Star}(v, K) - \Star(v,k)\]
            is a simple closed polygonal curve by induction on the edge set. 
        \end{proof}

    \item Let $G$ be a finite group which acts as a group of homeomorphisms of a closed surface $S$ in such a way that the only element with any fixed points is the identity. We show 
        \begin{itemize}
            \item that the orbit space $S/G$ is a closed surface: 
                \begin{proof} The following proof is credited to \cite{groupActions}: 
                    \par Since the group acts freely, for any point $x \in S$, the elements $g \cdot x$ are pairwise distinct for $g \in G$. Because $S$ is Hausdorff, this means that for every $g$, there are neighborhoods $U_g$ and $V_g$ respectively of $x$ and $g \cdot x$ that are disjoint. Now $G$ is finite so $\bigcap_{g \in G} U_g$ is still a neighborhood of $x$. $G$ acts by homeomorphism, so if you restrict the neighborhoods you can then prove that the quotient map is a local homeomorphism. Thus $S/G$ is locally homeomorphic to a plane, just like $S$.
                    \par Similarly if $x$ and $y$ have different orbits, then $g \cdot x \neq h \cdot y$ for all $g,h \in G$ and so you can find neighborhoods $U_g$, $V_h$ that don't intersect. The intersections $\bigcap_g U_g$ and $\bigcap_h V_h$ are still neighborhoods because $G$ is finite, and map to neighborhoods of $[x]$ and $[y]$ that don't intersect in $S/G$. So $S/G$ is Hausdorff too.
                \end{proof}

            \item $S$ may be orientable, yet $S/G$ non-orientable: 
                \begin{proof} We know that the canonical group action of $\Z_2$ on $S^2$ yeilds the M\"obius strip, which is non-orientable. 
                \end{proof}

            \item that if $S/G$ is orientable, then $S$ does not have to be: 
                \begin{proof} The example above works, by considering the group action of $\Z_2$ on $\RP^2$ to give $S^2$ back, as $\RP^2/\Z_2$.
                \end{proof}
        \end{itemize}

    \item Let $K$ be an orientable combinatorial surface, orient all its triangles in a compatible manner, and suppose that $h: |K| \rightarrow |K|$ is a simplicial homeomorphism. Suppose, further, that there is a triangle $A$, oriented by the ordering $(uvw)$ whose image $h(A)$ occurs with the orientation $(h(u)h(v)h(w))$ induced by $h$. We prove that the same must hold for any other triangle of $K$: 
        \begin{proof}
            \par We give an example of an orientable combinatorial surface and a simplicial homeomorphism which is not orientation-preserving:
        \end{proof}

    \item Let $K$ be an orientable combinatorial surface. Suppose that $G$ acts simplicially on $|K|$. We show that if the action is fixed-point free and each element of $G$ is orientation-preserving that the complex $K^2/G$ is an orientable combinatorial surface:
        \begin{proof} Pick an orientation on $K$. Then, every triangle $T \in K^2$ has an orientation, since $G$ is orientation preserving. Thus, 
            \[\bigcup_{g \in G} gT\]
            has consistent orientation. But this implies that 
            \[[T] - \bigcup_{g \in G} gT = T/G\]
            has well-defined orientation and hence 
            \[K^2/G\]
            is an orientable combinatorial surface. 
        \end{proof}
\end{enumerate}

\section{Euler Characteristics}
\begin{enumerate}[(1)]
    \item We prove lemma ($7.8$): 
        \begin{proof} This is an application of \hyperref[sec:splitexact]{\ref{sec:splitexact}}. This was inspired by \cite{eulerChar}, \cite{maunder}: 
            \par Consider the split exact sequence (as $\chi$ is addative) 
            \[0 \rightarrow \mathcal{F}_{K\cup L} \rightarrow \mathcal{F}_{K} \oplus \mathcal{F}_{L} \rightarrow \mathcal{F}_{K\cap L} \rightarrow 0\]
            via maps $a \mapsto (a,a)$ and $(a,b) \mapsto a -b$. Then by  \hyperref[sec:splitexact]{\ref{sec:splitexact}}, the result follows. 
        \end{proof}

    \item We prove lemma ($7.9$): 
        \begin{proof} As each barycentric subdivision of some simplex $\sigma_k$ adds one more vertex, namely 
            \[\hat{A} =  \frac{1}{k+1} (v_0 + \dots + v_k)\]
            the result follows. 
        \end{proof}

    \item We prove from the previous problem that the Euler characteristic of a graph $\Gamma$ is a topological invariant of $|\Gamma|$: 
        \begin{proof} Please see \cite{kinsey}, page one-hundered-five, theorem ($5.13$). 
        \end{proof}

    \item Let $K$ be a finite complex. Suppose that $G$ acts simplicially on $|K|$, and that the action is fixed-point free. We show that 
        \[\chi(K) = |G| \chi(K^2/G)\]
        where $|G| = \card G$:
        \begin{proof} Suppose that $K$ if finite. Then, it follows from a previous exercise that 
            \[K/G \simeq K^2/G\]
            Thus, for every simplex $\sigma$ in $K^2/G$, there corresponds $|G| \sigma$  simplexes in $K^2$. Thus, 
            \[\chi(K^2) = |G| (|\sigma_0| - |\sigma_1| + \dots + (-1)^d|\sigma_d| ) = |G| \cdot \chi(K^2/G)\]
            But, as $\chi$ is topological invariant, and $|K| = |K^2|$, we have 
            \[\chi(K ) = \chi(K^2) = |G|\chi(K^2/G)\]
        \end{proof}

    \item Let $K$ be a combinatorial surface such that $|K|$ is in the plane as in problem $6$, and let $J$ denote the boundary curve of the resulting regular polygon. We show that 
        \[\chi(K) = \chi(\Gamma) + 1\]: 
        \begin{proof} As $K$ is combinatorial, theorem ($7.7$) implies that $\chi(K) = 2$. In otherwords, we show that 
            \[\chi(\Gamma) = 1\]
            But, as $\Gamma$ is a tree, lemma ($7.5$) says that 
            \[\chi(\Gamma) = 1\]
            The result follows. 
        \end{proof}

    \item Continuing from the previous problem, suppose that $\Gamma$ has an edge, one end of which is not joined to any other edges. We show that $\chi(K) = 2$ implies 
        \[|K| \cong S^2:\]
        \begin{proof} We omit this proof. 
        \end{proof}
\end{enumerate}

\section{Surgery}
\begin{enumerate}[(1)]
    \item We thicken each curve in fig. $7.16$ and decide whether the result is a cylinder or a M\"obius strip and describe the effect of doing surgery along the curve: 
        \begin{proof} This is left to the reader as an exercise. 
        \end{proof}

    \item We show that the surface illustrated in fig. $7.17$ is homeomorphic to one of the standard ones using the procedure of theorem ($7.14$): 
        \begin{proof} As 
            \[T \# T\]
            is a sphere with two handles attached, we have that figure $7.17$ is 
            \[\KB \# S_3^2,\]
            where $S_3^2$ denotes a sphere with three handles added. By a previous exercise, we have that figure $7.17$ is 
            \[\RP^2 \# \RP^2 \# S_3^2\]
            and the result follows. 

        \end{proof} 

    \item Let $Z \subset Y \subset Z$ be three concentric discs in the plane. We find a homeomorphism from $X$ to itself which is the identity on the boundary circle of $X$ and which throws $Y$ onto $Z$: 
        \begin{proof} Suppose, without loss of generality, that 
            \[Z \subsetneq Y \subsetneq X\]
            As such, there exists a retraction $r: Y \hookrightarrow Z$, $r|_Z \equiv 1_Z$. Now, we extend $r$ to some $R: X \rightarrow X$, so that
            \[ R(x) =     
            \begin{cases}
                x & x \in X - Y \\
                r(x) & x \in Y 
            \end{cases}
            \]
            Then, clearly $R = 1_{\partial X}$, and $R: Y \hookrightarrow Z$. It is left to show that $R$ is continuous. Without loss of generality, let $\Ocal \subset \Image R$ be open, such that 
            \[\Ocal \cap (X - Z) \cap Z \neq \emptyset\]
            Then, 
            \[R^{-1}(\Ocal) = R^{-1}(\Out Z \cap \Ocal) \cup R^{-1}(\overline{\Int} Z \cap \Ocal)\]
            by the Jordan curve separation theorem(s), which is clearly open $X$. A similar argument show that $R^{-1}$ is continuous. The result follows. 
        \end{proof}

    \item Suppose that we have two discs in the plane, both of which are bounded by polygonal curves, and one which lies in the interior of the other. We prove that the region between them is homeomorphic to an annulus: 
        \begin{proof} Please see \cite{jordanCurve}.
        \end{proof}

    \item with the notation of lemma ($7.13$) and problem $19$, we find a homeomorphism $h: D^1 \rightarrow X$ such that $h(D) = Y$ and 
        \[h(\overline{\Star} (\hat{A}, K^2)) = Z\]
        and prove lemma ($7.13$): 
        \begin{proof} As $A$ is a triangle of $D$, $A$ is homeomorphic to a disc, via t, and we have the following inclusions:
            \[ A \subset D \subset K\]
        \end{proof}
\end{enumerate}

\section{Surface Symbols}
\begin{enumerate}[(1)]
    \item We show that the two surfaces shown in fig. $7.22$ are not homeomorphic: 
        \begin{proof}
        \end{proof}

    \item We determine what happens if we remove the interiors of two disjoint discs from a closed surface, then glue the two resulting boundary circles together: 
        \begin{proof}
        \end{proof}

    \item We use the classification theorem to show that the operation of connected sum is well-defined: 
        \begin{proof}
        \end{proof}

    \item Assume that every compact surface can be triangulated. We show that if the boundary component of a surface is non-empty, then it consists of a finite number of disjoint circles: 
        \begin{proof}
        \end{proof}

    \item We show that nay compact connect surface is homeomorphic to a closed surface from which the interiors of a finite number of disjoint discs have been removed: 
        \begin{proof}
        \end{proof}

    \item We determine the fundamental group of the space obtained by punching $k$ holes in the sphere: 
        \begin{proof}
        \end{proof}

    \item Write $H(p,r)$ for $H(p)$ with the interiors of $r$ disjoint discs removed, and write $M(q,s)$ for $M(q)$ with $s$ discs similarly removed. We show that $H(p,r)$ can be obtained from a $(4p + 3r)$-sided polygonal region in the plane by gluing up its edges according to the surface symbol 
        \[a_1 b_1 a^{-1}_1 b^{-1}_1 \dots a_p b_p a^{-1}_p b^{-1}_p x_1 y_1 x_1^{-1} \dots x_r y_r x^{-1}_r:\]
        \begin{proof}
        \end{proof}

    \item We find a surface symbol for $M(q,s)$, as defined in the previous problem: 
        \begin{proof}
        \end{proof}

    \item We calculate the fundamental groups of $H(p,r)$ and $M(q,s)$: 
        \begin{proof}
        \end{proof}

    \item We show
        \begin{itemize}
            \item that $H(p,r) \cong H(p',r')$ implies $p = p'$ and $r = r'$: 
                \begin{proof}
                \end{proof}

            \item that $M(q,s) \cong M(q',s')$ implies $q = q'$ and $s = s'$: 
                \begin{proof}
                \end{proof}

            \item that there are no values of $p,q,r,s$ for which $M(p,r) \cong H(q,s)$: 
                \begin{proof}
                \end{proof}
        \end{itemize}

    \item Define the genus of a compact connect surface to be the genus of the closed surface obtained by capping off each boundary circle with a disc. We show that a compact connect surface is completely determined by wheter or not it is orientable, together with its genus and its number of boundary circles: 
        \begin{proof}
        \end{proof}

    \item We posit a general result from the two picture in fig. $7.24$: 
        \begin{proof}
        \end{proof}

\end{enumerate}

\chapter{Simplicial Homology}
\section{Cycles and Boundaries}
\begin{enumerate}[(1)]
    \item 
\end{enumerate}

\section{Homology Groups}
\begin{enumerate}[(1)]
    \item 
\end{enumerate}

\section{Examples}
\begin{enumerate}[(1)]
    \item 
\end{enumerate}

\section{Simplicial Maps}
\begin{enumerate}[(1)]
    \item 
\end{enumerate}

\section{Stellar Subdivision}
\begin{enumerate}[(1)]
    \item 
\end{enumerate}

\section{Invariance}
\begin{enumerate}[(1)]
    \item 
\end{enumerate}


\nocite{*}
\bibliographystyle{apa}
\bibliography{bib}

\end{document}
